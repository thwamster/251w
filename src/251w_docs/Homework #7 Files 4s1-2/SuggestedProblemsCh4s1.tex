\documentclass{article}
\usepackage[dvips]{graphicx}
\usepackage{a4wide}
\usepackage{amsmath}
\usepackage{euscript}
\usepackage{amssymb}
\usepackage{amsthm}
\usepackage{amsopn}

\theoremstyle{definition}
\newtheorem*{definition}{Definition}
\newtheorem{theorem}{Theorem}

\newcommand{\vv}{\ensuremath{\vec{v}}}
\newcommand{\vu}{\ensuremath{\vec{u}}}
\newcommand{\vw}{\ensuremath{\vec{w}}}
\newcommand{\vx}{\ensuremath{\vec{x}}}
\newcommand{\vy}{\ensuremath{\vec{y}}}
\newcommand{\vb}{\ensuremath{\vec{b}}}
\newcommand{\vo}{\ensuremath{\vec{0}}}
\newcommand{\va}{\ensuremath{\vec{a}}}
\newcommand{\ve}{\ensuremath{\vec{e}}}

\newcommand{\R}{\mathbb{R}}
\newcommand{\Z}{\mathbb{Z}}
\newcommand{\C}{\mathbb{C}}
\newcommand{\N}{\mathbb{N}}
\newcommand{\Q}{\mathbb{Q}}

%%%%%%%%%%%%%%%%%%%%%%%%%%%%%%%%%%%%%%%%%%%%%%%


\begin{document}
    \begin{center}
        \Large{Math 251W: Foundations of Advanced Mathematics}

        \normalsize Solutions to problems from section 4.1 \hrule
        \vspace{0.4cm}

    \end{center}


    \noindent Problem 4.1.1  \vspace{0.3cm}

    \begin{enumerate}
        \item Function
        \item Not a function:  $a$ is paired with two distinct elements of
        the codomain.
        \item Function
        \item Not a function: Not every element of the domain is represented
        by an ordered pair
        \item Not a function:  $c$ is paired with two distinct elements of
        the codomain
        \item Function
    \end{enumerate}

    \vspace{1cm}

%%%%%%%%%%%%%%%%%%%%%%%%%%%%%%%%%%%%%%%%%%%%%%%%%%%%%%%%%%%%%%%%%%%%%%%%%%%%

    \noindent Problem 4.1.2  \vspace{0.3cm}

    \begin{enumerate}
        \item Function
        \item Not a function
        \item Not a function
        \item Function
        \item Not a function
    \end{enumerate}

    \vspace{1cm}

%%%%%%%%%%%%%%%%%%%%%%%%%%%%%%%%%%%%%%%%%%%%%%%%%%%%%%%%%%%%%%%%%%%%%%%%

    \noindent Problem 4.1.3  \vspace{0.3cm}

    \begin{enumerate}
        \item[i] Function
        \item[ii] Not a function
        \item[iii] Not a function
        \item[iv] Function
    \end{enumerate}

    \vspace{1cm}

%%%%%%%%%%%%%%%%%%%%%%%%%%%%%%%%%%%%%%%%%%%%%%%%%%%%%%%%%%%%%%%%%%%%%%%%%%%%%%


%%%%%%%%%%%%%%%%%%%%%%%%%%%%%%%%%%%%%%%%%%%%%%%%%%%%%%%%%%%%%%%%%%%%%%%%%%%%%

    \noindent Problem 4.1.4  \vspace{0.3cm}

    \begin{enumerate}
        \item Needs to specify domain and codomain
        \item Describes a function
        \item Does not define a function. The logarithm is not defined for
        all real numbers
        \item I would say this is good enough, but technically it should say
        $g(x)=e^x$ for all $x\in\R$
    \end{enumerate}

    \vspace{1cm}

%%%%%%%%%%%%%%%%%%%%%%%%%%%%%%%%%%%%%%%%%%%%%%%%%%%%%%%%%%%%%%%%%%%%%%%%%%%%


    \noindent Problem 4.1.5  \vspace{0.3cm}

    \begin{enumerate}
        \item[5] Not a function: The formula is not defined for the subset
        of the domain $(0,1)$.
        \item[6] Not a function: The formula assigns the element $1$ of the
        domain two possible outputs, $1$ and $-1$
    \end{enumerate}

    \vspace{1cm}

%%%%%%%%%%%%%%%%%%%%%%%%%%%%%%%%%%%%%%%%%%%%%%%%%%%%%%%%%%%%%%%%%%%%%%%%%%%%%%%%

    \noindent Problem 4.1.6  \vspace{0.3cm}

    \begin{enumerate}
        \item[2] $X=[-1,1]$
        \item[3] Let $U_i$ be the set $(2i\pi,(2i+1)\pi)$.
        $X=\{U_i\}_{i\in\Z}$
        \item[4] You can let $X=(0,\infty)$, or $X=(-\infty,0)$.

    \end{enumerate}

    \vspace{1cm}

%%%%%%%%%%%%%%%%%%%%%%%%%%%%%%%%%%%%%%%%%%%%%%%%%%%%%%%%%%%%%%%%%%%%%%%%%%%%%

%%%%%%%%%%%%%%%%%%%%%%%%%%%%%%%%%%%%%%%%%%%%%%%%%%%%%%%%%%%%%%%%%%%%%

    \noindent Problem 4.1.7  \vspace{0.3cm}

    Not necessarily. The functions $g$ and $f$ will act the same on the
    subset $S$, but they could map the set $A-S$ in completely different
    ways.

    Counterexample: Let $A=\{a, b, c\}$, $S=\{a,c\}$, $B=\{1,2,3,4\}$.
    Thus, $f=\{(a,1), (b,2), (c,3)\}$ and $g=\{(a,1),(b,4),(c,3)\}$ are
    two function such that $g$ is an extension of $f|_S$, but $f\not=g$.

    \vspace{1cm}

%%%%%%%%%%%%%%%%%%%%%%%%%%%%%%%%%%%%%%%%%%%%%%%%%%%%%%%%%%%%%%%%%%%%%%%%%%



\end{document}