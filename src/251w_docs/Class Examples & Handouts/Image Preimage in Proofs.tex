\documentclass{article}
\usepackage{xypic}
\usepackage{graphicx}
\usepackage{amsmath, amsthm, amssymb}
\usepackage{url}
\usepackage[usenames,dvipsnames]{color}
\setlength{\textheight}{22.5cm} \setlength{\textwidth}{16cm}
\setlength{\topmargin}{-1cm} \setlength{\oddsidemargin}{0cm}
\setlength{\evensidemargin}{0cm}

\newcommand{\R}{\mathbb{R}}
\newcommand{\Z}{\mathbb{Z}}
\newcommand{\C}{\mathbb{C}}
\newcommand{\N}{\mathbb{N}}
\newcommand{\Q}{\mathbb{Q}}

\newcommand{\tcG}{\textcolor{Green}}
\newcommand{\tcR}{\textcolor{Red}}
\newcommand{\tcB}{\textcolor{Blue}}
\newcommand{\tcP}{\textcolor{Purple}}

%%%%%%%%%%%%%%%%%%%%%%%%%%%%%%%%%%%%%%%%%%%%%%%


\begin{document}

    \newpage
    \hfill\tiny{Math 251W: Foundations of Advanced Mathematics, McNicholas}

    \vspace{0.3cm}
    \centerline{\Large{Tips \& Techniques}}
    \vspace{0.3cm}
    \hrule
    \vspace{0.3cm}
    \vspace{.5cm}

    \normalsize
    \fbox{Using the definition of image and pre-image in a proof}
    \medskip

    The application of these definitions can be a little tricky at first. The following list shows how you would apply the definition in each possible direction.
    \begin{enumerate}
        \item If you know an \textbf{element $y$ is in the image of some set $P$ under the function $f$}, i.e. $$y\in f_*(P),$$

        you can conclude \underline{by the definition of image}  that there is some element in $P$ that gets mapped to $y$, i.e. $$\exists x\in P\mbox{ such that } f(x)=y$$
        \item Conversely, if you know an \textbf{element $x$ is in the set $P,$} $$x\in P,$$ you can conclude \underline{by the definition of image} that $f(x)$ is in the image of $P$ under $f,$ i.e.
        $$x\in P \Rightarrow f(x)\in f_*(P).$$
        \item If you know an \textbf{element $x$ is in the pre-image of some set $Q$ under the function $f$}, symbolically $$x\in f^*(Q)$$

        you can conclude \underline{by the definition of pre-image}  that $f(x)$ is an element of $Q$, i.e. $$f(x)\in Q.$$
        \item Conversely, if you know an \textbf{element $f(x)$ is in some subset $Q$ of the codomain}, i.e. $$f(x)\in Q,$$ you can conclude \underline{by the definition of pre-image} that
        $$x\in f^*(Q).$$
    \end{enumerate}

    \vspace{.25cm}
    \hrule
%****************************************

    \vspace{.5cm}

    \fbox{Proving Function Equality, $f=g$}
    \begin{itemize}
        \item Verify domains are equal (if this is not obvious is amounts to a set equality proof - but usually it's obvious from the definitions of the functions)
        \item Verify codomains are equal (if this is not obvious is amounts to a set equality proof - but usually it's obvious from the definitions of the functions)
        \item Prove the functions map each element of the domain in the same way. There are two options for this:
        \begin{itemize}
            \item (Option 1) Let $x$ be an arbitrary element of ``the domain'' (we can refer to the domain without it being ambiguous once we have established that the domains of the two functions are the same). Show that $f(x)$ is the same as $g(x).$
            \item (Option 2) Prove the sets $F\subseteq\mbox{Domain}\times\mbox{Codomain}$ and $G\subseteq\mbox{Domain}\times\mbox{Codomain}$ associated with $f$ and $g$ respectively are equal (so a set equality proof)
        \end{itemize}
    \end{itemize}

    \vspace{.25cm}
    \hrule


\end{document}