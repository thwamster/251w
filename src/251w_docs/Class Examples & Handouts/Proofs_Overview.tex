\documentclass{article}
\usepackage{xypic}
\usepackage{graphicx}
\usepackage{amsmath, amsthm, amssymb}
\usepackage{url}
\usepackage[usenames,dvipsnames]{color}
\setlength{\textheight}{22.5cm} \setlength{\textwidth}{16cm}
\setlength{\topmargin}{-1cm} \setlength{\oddsidemargin}{0cm}
\setlength{\evensidemargin}{0cm}

\newcommand{\R}{\mathbb{R}}
\newcommand{\Z}{\mathbb{Z}}
\newcommand{\C}{\mathbb{C}}
\newcommand{\N}{\mathbb{N}}
\newcommand{\Q}{\mathbb{Q}}

\newcommand{\tcG}{\textcolor{Green}}
\newcommand{\tcR}{\textcolor{Red}}
\newcommand{\tcB}{\textcolor{Blue}}
\newcommand{\tcP}{\textcolor{Purple}}

%%%%%%%%%%%%%%%%%%%%%%%%%%%%%%%%%%%%%%%%%%%%%%%


\begin{document}

    \newpage
    \hfill\tiny{Math 251W: Foundations of Advanced Mathematics, McNicholas}

    \vspace{0.3cm}
    \centerline{\Large{Types of Proofs/What We Need to Prove}}
    \vspace{0.3cm}
    \hrule
    \vspace{0.3cm}
    \vspace{.5cm}

    \normalsize
    \fbox{Steps to Writing a Proof}
    \begin{enumerate}
        \item Fully understand the proposition
        \begin{itemize}
            \item Simplify any negations
            \item Review relevant definitions
            \item Identify the structure of the statement (convert to symbolic notation, recognize hidden quantifiers, etc.)
        \end{itemize}
        \item Determine the type of proof (Contradiction, $\forall$, $\exists$, $!$, Contrapositive, Direct, etc.)
        \begin{itemize}
            \item Start to scaffold the proof: State the premises,
            Identify the Conclusion
        \end{itemize}
        \item Formulate the proof
        \begin{itemize}

            \item Create a chain of logical equivalences and implications linking the premises to the conclusion and make explicit calls to \textbf{mathematical justifications}.
        \end{itemize}
    \end{enumerate}

    \vspace{.25cm}
    \hrule
%****************************************

    \vspace{.5cm}

    \fbox{Tips \& Techniques}
    \begin{itemize}
        \item Go back to definitions!
        \item Determine \emph{what} needs to be proved (See types of proofs).
        \item Use scratch for explorations, discovery of proof, working backwards (in the case of existence proofs).
        \item Write in proper grammatical English.
    \end{itemize}

    \vspace{.25cm}
    \hrule
%****************************************
    \vspace{.5cm}

    \tcB{Format:
    \underline{Scratch}}
    \begin{quote}
        \textcolor{Green}{Experimentation, exploration, scratch work, backward construction}
    \end{quote}
    \vspace{.25cm}
    \tcB{\underline{proposition:}} \textcolor{Green}{Statement of proposition}

    \vspace{.25cm}

    \tcB{\fbox{proof}}\textcolor{Green}{(Type)}
    \begin{itemize}
        \item \textcolor{Green}{Start with assumptions}
        \item \textcolor{Green}{proof: complete sentences, grammatically correct sentences; define variables; refer back to definitions, thms, etc.}
        \item \textcolor{Green}{Concluding statement $\blacksquare$ or Q.E.D. (quod erat demonstrandum ``that which was to be demonstrated'')}
    \end{itemize}
    \vspace{.25cm}
    \hrule
%****************************************
    \vspace{.5cm}
    \newpage
    \normalsize
    Many proofs involve combinations of the following techniques.
    \begin{enumerate}
        \item Contradiction: $R\Leftrightarrow \neg(\neg R)$
        $$\neg R\Rightarrow?\Rightarrow \cdots\Rightarrow ?\Rightarrow  \ddagger$$
        \item For all (\underline{prop:} $(\forall x\in U) P(x))$
        \begin{itemize}
            \item Let $a$ be an arbitrary element of $U.$  Show $P(a)$ is true.
        \end{itemize}
        \item Existence (\underline{prop:} $(\exists x\in U) P(x)$)
        \begin{itemize}
            \item Use scratch to construct instance/example (often backwards)
            \item In proof, verify this instance satisfies all requirements \tcR{(\emph{Beware of backwards proofs!})}
        \end{itemize}
        \item Uniqueness (\underline{prop:} $(\exists ! x \in U) P(x)$)
        \begin{itemize}
            \item Have to prove \emph{existence} and \emph{uniqueness}
            \item Proving uniqueness:
            \begin{itemize}
                \item Assume there exist two such elements, i.e. $P(x_0)$ and $P(y_0)$ are both true. Show $x_0=y_0.$ OR
                \item Assume $x_0\not= y_0, P(x_0),$ and $P(y_0)$ are all true. Show $\ddagger.$
            \end{itemize}
        \end{itemize}

        \item  If...Then...($P\Rightarrow Q$)
        \begin{itemize}
            \item \underline{Direct Proof:} Start with premises (assumptions) $P$, use logical implications and equivalences to deduce $Q.$  $$P\Rightarrow ?\Rightarrow ?\Rightarrow\dots\Rightarrow ?\Rightarrow Q$$
            \item \underline{Contrapositive:} $(P\Rightarrow Q)\Leftrightarrow$ \fbox{$\neg Q\Rightarrow \neg P$} \newline
            Start with $\neg Q$, use logical implications to get to $\neg P.$ $$\neg Q\Rightarrow ?\Rightarrow\dots\Rightarrow ?\Rightarrow \neg P.$$
            \item \underline{Contradiction:} $R\Leftrightarrow \neg(\neg R),\quad\therefore (P\Rightarrow Q)\Leftrightarrow$ \fbox{$\neg(P\wedge\neg Q)$} \newline
            Show $P\wedge\neg Q$ is a contradiction, (which makes $\neg(P\wedge\neg Q)$, i.e. $P\Rightarrow Q$, a tautology).  $$(P\wedge\neg Q)\Rightarrow ?\Rightarrow\dots\Rightarrow ?\Rightarrow \ddagger$$
            \item \underline{Cases:}
            \begin{itemize}
                \item Equivalence: $(P\Leftrightarrow Q)\Leftrightarrow$ \fbox{$(P\Rightarrow Q)\wedge(Q\Rightarrow P)$} (2 proofs in one), or \newline
                $P\Leftrightarrow ?\Leftrightarrow \dots \Leftrightarrow ?\Leftrightarrow Q$ (Use only equivalences not implications. Faster, but often much more difficult as you have to proceed in the same way in each direction)
                \item Chain of equivalences: $(P\Leftrightarrow Q\Leftrightarrow R\Leftrightarrow S)\Leftrightarrow$ \fbox{$(P\Rightarrow Q)\wedge(Q\Rightarrow R)\wedge(R\Rightarrow S)\wedge (S\Rightarrow P)$} (Many proofs in one, have to close the circle.)
                \item OR's
                \begin{itemize}
                    \item $[(A\vee B)\Rightarrow Q]\Leftrightarrow$ \fbox{$(A\Rightarrow Q)\wedge (B\Rightarrow Q)$} (2 proofs in one)
                    \item $[P\Rightarrow (A\vee B)]\Leftrightarrow$ \fbox{$(\neg A\wedge\neg B)\Rightarrow \neg P$} (Contrapositive. ``Ands'' are often easier to work with)
                    \item $[P\Rightarrow (A\vee B)]\Leftrightarrow$ \fbox{$(P\wedge\neg A)\Rightarrow B$} $\Leftrightarrow$ \fbox{$(P\wedge\neg B)\Rightarrow A$} (Whichever is easier to prove)
                \end{itemize}
                \item AND's
                \begin{itemize}
                    \item $[(A\wedge B)\Rightarrow Q].$  $$(A\wedge B)\Rightarrow ?\Rightarrow \dots\Rightarrow ?\Rightarrow Q$$
                    \item $[P\Rightarrow(A\wedge B)]\Leftrightarrow$ \fbox{$(P\Rightarrow A)\wedge (P\Rightarrow B)$} (2 proofs in one)
                \end{itemize}
            \end{itemize}
        \end{itemize}


    \end{enumerate}

    \hrule

%****************************************
% \vspace{.5cm}

% \fbox{Steps to Writing a Proof}
% \begin{enumerate}
% \item Fully understand the proposition: simplify any negations, identify hidden quantifiers, etc, convert (partially) to symbolic notation.
% \item Determine the type of proof and outline (scaffold) the structure imposed by this proof type.
% \begin{itemize}
%     \item State the premises/assumptions and desired conclusion
% \end{itemize}
% \item Formulate the proof
% \begin{itemize}
%     \item Create chain of logical implications/equivalences linking the proposition to the conclusion and making explicit calls to mathematical justifications.
% \end{itemize}
% \end{enumerate}

% \vspace{.25cm}
% \hrule
% %****************************************
% \newpage
% %\vspace{.5cm}

% \fbox{Tips \& Techniques}
% \begin{itemize}
% \item Go back to definitions!
% \item Determine \emph{what} needs to be proved (See types of proofs).
% \item Use scratch for explorations, discovery of proof, working backwards (in the case of existence proofs).
% \item Write in proper grammatical English.
% \end{itemize}

% \vspace{.25cm}
% \hrule
% %****************************************
% \vspace{.5cm}

% \tcB{Format:
% \underline{Scratch}}
% \begin{quote}
% \textcolor{Green}{Experimentation, exploration, scratch work, backward construction}
% \end{quote}
% \vspace{.25cm}
% \tcB{\underline{proposition:}} \textcolor{Green}{Statement of proposition}

% \vspace{.25cm}

% \tcB{\fbox{proof}}\textcolor{Green}{(Type)}
% \begin{itemize}
% \item \textcolor{Green}{Start with assumptions}
% \item \textcolor{Green}{proof: complete sentences, grammatically correct sentences; define variables; refer back to definitions, thms, etc.}
% \item \textcolor{Green}{Concluding statement $\blacksquare$ or Q.E.D. (quod erat demonstrandum ``that which was to be demonstrated'')}
% \end{itemize}
\end{document}