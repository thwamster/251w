\documentclass{article}
\usepackage[dvips]{graphicx}
\usepackage{a4wide}
\usepackage{amsmath}
\usepackage{euscript}
\usepackage{amssymb}
\usepackage{amsthm}
\usepackage{amsopn}

\theoremstyle{definition}
\newtheorem*{definition}{Definition}
\newtheorem{theorem}{Theorem}

\newcommand{\vv}{\ensuremath{\vec{v}}}
\newcommand{\vu}{\ensuremath{\vec{u}}}
\newcommand{\vw}{\ensuremath{\vec{w}}}
\newcommand{\vx}{\ensuremath{\vec{x}}}
\newcommand{\vy}{\ensuremath{\vec{y}}}
\newcommand{\vb}{\ensuremath{\vec{b}}}
\newcommand{\vo}{\ensuremath{\vec{0}}}
\newcommand{\va}{\ensuremath{\vec{a}}}
\newcommand{\ve}{\ensuremath{\vec{e}}}

\newcommand{\R}{\mathbb{R}}
\newcommand{\Z}{\mathbb{Z}}
\newcommand{\C}{\mathbb{C}}
\newcommand{\N}{\mathbb{N}}
\newcommand{\Q}{\mathbb{Q}}

%%%%%%%%%%%%%%%%%%%%%%%%%%%%%%%%%%%%%%%%%%%%%%%


\begin{document}
    \begin{center}
        \Large{Math 251W: Foundations of Advanced Mathematics}

        \normalsize Solutions to problems from section 5.1 \hrule
        \vspace{0.4cm}

    \end{center}


    \noindent Problem 5.1.1  \vspace{0.3cm}

    \begin{enumerate}
        \item[2] $[3]=\{3n|n\in \Z\}$, $[-3]=\{-3n|n\in\Z\}$, and
        $[6]=\{6n|n\in\Z\}$
        \item[3] $[3]=\{1,3,-1,-3\}=[-3]$, $[6]=\{1,-1,2,-2,3,-3,6,-6\}$
    \end{enumerate}

    \vspace{1cm}

%%%%%%%%%%%%%%%%%%%%%%%%%%%%%%%%%%%%%%%%%%%%%%%%%%%%%%%%%%%%%%%%%%%%%%%%%%%%

    \noindent Problem 5.1.2  \vspace{0.3cm}

    \begin{enumerate}
        \item[1] $[(0,0)]=x$-axis, $[(3,4)]=$ the line $y=\frac{4}{3}$
        \item[2] $[(0,0)]=\{(0,0)\}$, $[(3,4)]=$ points on the ellipse
        $57=7x^2+y^2$
        \item[3] $[(0,0)]=$ the $x$-axis and $y$-axis, $[(3,4)]=$ the line
        $x=3$ and the line $y=4$.
    \end{enumerate}

    \vspace{1cm}

%%%%%%%%%%%%%%%%%%%%%%%%%%%%%%%%%%%%%%%%%%%%%%%%%%%%%%%%%%%%%%%%%%%%%%%%

    \noindent Problem 5.1.4  \vspace{0.3cm}

    \begin{enumerate}
        \item[2] ${\cal R}$ is symmetric, but not reflexive nor transitive
        \item[4] ${\cal M}$ is symmetric, reflexive, and transitive
        \item[6] ${\cal D}$ is reflexive, and transitive, but not symmetric

    \end{enumerate}

    \vspace{1cm}

%%%%%%%%%%%%%%%%%%%%%%%%%%%%%%%%%%%%%%%%%%%%%%%%%%%%%%%%%%%%%%%%%%%%%%%%%%%%%%


    \noindent Problem 5.1.5  \vspace{0.3cm}

    \begin{enumerate}
        \item If ${\cal R}$ is reflexive, ${\cal R'}$ is not reflexive
        \item If ${\cal R}$ is symmetric, ${\cal R'}$ is symmetric
        \item If ${\cal R}$ is transitive, ${\cal R'}$ is not necessarily
        transitive
    \end{enumerate}

    \vspace{1cm}

%%%%%%%%%%%%%%%%%%%%%%%%%%%%%%%%%%%%%%%%%%%%%%%%%%%%%%%%%%%%%%%%%%%%%%%%%%%%%%

%%%%%%%%%%%%%%%%%%%%%%%%%%%%%%%%%%%%%%%%%%%%%%%%%%%%%%%%%%%%%%%%%%%%%%%%%%%%%

    \noindent Problem 5.1.6  \vspace{0.3cm}

    Problem is in the line "Choose $y\in A$ such that $x{\cal R}y.$"
    There is no guarantee that given a particular $x$ in $A$ there will
    be an ordered pair $(x,y)\in {\cal R}$. Unlike a function, a
    relation does not have to have an ordered pair for every element of
    $A$.

    \vspace{1cm}

%%%%%%%%%%%%%%%%%%%%%%%%%%%%%%%%%%%%%%%%%%%%%%%%%%%%%%%%%%%%%%%%%%%%%%%%%%%%


    \noindent Problem 5.1.7  \vspace{0.3cm}

    The relation defined in example 5.1.1(5) is reflexive and
    transitive, but not symmetric.


    \vspace{1cm}


    \newpage
%%%%%%%%%%%%%%%%%%%%%%%%%%%%%%%%%%%%%%%%%%%%%%%%%%%%%%%%%%%%%%%%%%%%%%%%%%


    \noindent Problem 5.1.13  \vspace{0.3cm}

    If there is an \underline{injective} map which respects the
    relation, the relation must only contain ordered pairs of the form
    $(x,x)$, i.e. $x{\cal R}y$ iff $x=y$.  {\it You should be able to
    prove this observation}

    \vspace{1cm}


%%%%%%%%%%%%%%%%%%%%%%%%%%%%%%%%%%%%%%%%%%%%%%%%%%%%%%%%%%%%%%%%%%%%%%%%%%





\end{document}