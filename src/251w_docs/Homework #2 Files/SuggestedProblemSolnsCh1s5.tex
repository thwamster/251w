\documentclass{article}
\usepackage[dvips]{graphicx}
\usepackage{a4wide}
\usepackage{amsmath}
\usepackage{euscript}
\usepackage{amssymb}
\usepackage{amsthm}
\usepackage{amsopn}

\theoremstyle{definition}
\newtheorem*{definition}{Definition}
\newtheorem{theorem}{Theorem}

\newcommand{\vv}{\ensuremath{\vec{v}}}
\newcommand{\vu}{\ensuremath{\vec{u}}}
\newcommand{\vw}{\ensuremath{\vec{w}}}
\newcommand{\vx}{\ensuremath{\vec{x}}}
\newcommand{\vy}{\ensuremath{\vec{y}}}
\newcommand{\vb}{\ensuremath{\vec{b}}}
\newcommand{\vo}{\ensuremath{\vec{0}}}
\newcommand{\va}{\ensuremath{\vec{a}}}
\newcommand{\ve}{\ensuremath{\vec{e}}}

\newcommand{\R}{\mathbb{R}}
\newcommand{\Z}{\mathbb{Z}}
\newcommand{\C}{\mathbb{C}}
\newcommand{\N}{\mathbb{N}}
\newcommand{\Q}{\mathbb{Q}}

%%%%%%%%%%%%%%%%%%%%%%%%%%%%%%%%%%%%%%%%%%%%%%%


\begin{document}
    \begin{center}
        \Large{Math 251W: Foundations of Advanced Mathematics}

        \normalsize \normalsize Hints \& Solutions to SUGGESTED PROBLEMS from Section
        1.5
        \vspace{0.25cm}
        \hrule
        \vspace{0.75cm}
    \end{center}

    \noindent {\it Note: The following are brief solutions or proofs for
    selected problems. Remember, the answer is the least important
    part. It's understanding how to get the answer and how to explain
    your process which is important.} \vspace{0.5cm}




    \noindent \fbox{1.5.2}

    \begin{enumerate}
        \item There exists a car x for all cars y such that x is as fast as
        y.
        \item For all cars x, there exists a car y, such that x is as
        expensive as y.
        \item There exists a car y for all cars x, such that car x is as
        fast as car y or car x is as old as car y.
        \item For all cars y, there exists a car x, such that if car x is
        not as expensive as car y, then car x is as fast as car y.
    \end{enumerate}

    \vspace{0.6cm}



    \noindent \fbox{1.5.4}

    \begin{enumerate}
        \item $(\exists q)(\forall p)A(p,q)$
        \item $(\forall q)(\exists p)B(p,q)$
        \item $(\exists q)(\forall p)A(p,q)\wedge\neg B(p,q)$
        \item $(\forall q)(\exists p)C(p,q)\wedge \neg A(p,q)$
    \end{enumerate}

    \vspace{0.6cm}

    \noindent \fbox{1.5.6}

    \begin{enumerate}
        \item There is a boy who is not good.
        \item Every bat weighs less than 50 lbs.
        \item There is a real number $x$ such that $x^2-2x\le0$.
        \item There is a parent who doesn't have to change diapers.
        \item There is a flying saucer which isn't aiming to conquer any
        galaxy.
        \item For all integers $n$, $n^2$ is not a perfect number.
        \item For all houses in Kansas, there exists a person who can enter
        the house without going blind.
        \item There exists a house which does not have any white doors.
        \item Every person in New York City doesn't own at least one book
        published in 1990.
    \end{enumerate}

    \vspace{0.6cm}

    \noindent \fbox{1.5.7}\vspace{0.2cm}

    The symbol $\Z$ represents the integers and the symbol $\R$
    represents the real numbers. Let $\Z^+$ represent the set of
    positive integers and $\R^+$ represent the set of positive real
    numbers. Let $P(Q,x)$ be the statement $\ln(Q-x)>5$, $T(x,k)$ be
    the statement $x\le k$, and $S(Q)$ be the statement "$Q$ is
    cacophonous."  Thus, the given statement can be written:
    $$(\exists Q \in \Z)(\forall x \in \R^+)(\exists k
    \in
    \Z^+)(P(Q,x)\wedge(T(x,k)\rightarrow S(Q)).$$

    Negating this statement we find:
    $$\neg[(\exists Q \in \Z)(\forall x \in \R^+)(\exists k \in
    \Z^+)(P(Q,x)\wedge(T(x,k)\rightarrow S(Q))]$$$$\Leftrightarrow (\forall
    Q \in \Z)(\exists x \in \R^+)(\forall k \in \Z^+)(\neg
    P(Q,x)\vee(T(x,k)\wedge\neg S(Q)).$$

    Translating this negation back into words, we have: for all integers
    $Q$, there exists a real number $x>0$ for all positive integers $k$,
    such that $\ln(Q-x)\le 5$, or $x\le k$ and $Q$ is not cacophonous.

    \vspace{0.6cm}

    \noindent \fbox{1.5.10}\vspace{0.2cm}

    The flaw occurs in line (5). When you use existential
    instantiation, you have to use a new variable than those already
    defined. Thus, we can't use existential instantiation to conclude
    $M(a)$, but rather $M(b)$. This is not the case with universal
    instantiation however. Since the variable in universal
    instantiation is arbitrary, you can chose to use a variable already
    used to represent an element of $U$.

    The argument in this problem is invalid. You can't conclude
    $(\exists x \in U)[M(x)\wedge Q(x)]$ from $(\exists x \in
    U)[P(x)\wedge Q(x)]$ and $(\exists x \in U)[M(x)]$. These two
    predicate logic statements say that there is some $x$ in $U$ such
    that $P(x)\wedge Q(x)$ is true, and some $x$ in $U$ such that $M(x)$
    is true. There is no reason the same $x$ should necessarily work in
    both cases.

    \vspace{0.6cm}


\end{document}