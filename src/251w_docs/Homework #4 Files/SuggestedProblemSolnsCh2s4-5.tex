\documentclass{article}
\usepackage[dvips]{graphicx}
\usepackage{a4wide}
\usepackage{amsmath}
\usepackage{euscript}
\usepackage{amssymb}
\usepackage{amsthm}
\usepackage{amsopn}

\theoremstyle{definition}
\newtheorem*{definition}{Definition}
\newtheorem{theorem}{Theorem}

\newcommand{\vv}{\ensuremath{\vec{v}}}
\newcommand{\vu}{\ensuremath{\vec{u}}}
\newcommand{\vw}{\ensuremath{\vec{w}}}
\newcommand{\vx}{\ensuremath{\vec{x}}}
\newcommand{\vy}{\ensuremath{\vec{y}}}
\newcommand{\vb}{\ensuremath{\vec{b}}}
\newcommand{\vo}{\ensuremath{\vec{0}}}
\newcommand{\va}{\ensuremath{\vec{a}}}
\newcommand{\ve}{\ensuremath{\vec{e}}}

\newcommand{\R}{\mathbb{R}}
\newcommand{\Z}{\mathbb{Z}}
\newcommand{\C}{\mathbb{C}}
\newcommand{\N}{\mathbb{N}}
\newcommand{\Q}{\mathbb{Q}}

%%%%%%%%%%%%%%%%%%%%%%%%%%%%%%%%%%%%%%%%%%%%%%%


\begin{document}
    \begin{center}
        \Large{Math 251W: Foundations of Advanced Mathematics}

        \normalsize Solutions or Hints to problems from sections 2.4 $\&$ 2.5 \hrule
        \vspace{0.75cm}
    \end{center}


    \noindent Problem 2.4.1  \vspace{0.3cm}

    \begin{enumerate}
        \item[1] \underline{proposition:} If an integer is combustible, then
        it is even or prime.

        This a proposition of the form $P\rightarrow(A\vee B)$. Thus, to
        prove this statement, it suffices to prove $(P\wedge\neg
        A)\rightarrow B$ or $(P\wedge\neg B)\rightarrow A$. Choosing the
        first option, the outline of the proof would be as follows:

        Assume $x$ is an odd, combustible integer. From this, somehow show
        $x$ must be prime.

        \item[3] \underline{proposition:} For an integer to be putrid, it is
        necessary and sufficient that it is both odd and divisible by 50.

        This is an equivalence statement. Thus we have to prove both that
        if an integer is putrid it is odd and divisible by 50, and that if
        an integer is odd and divisible by 50 it is putrid.

        For the first proof, assume $x$ is putrid. Use this assumption to
        somehow show that $x$ is odd. Use this assumption to show $x$ is
        also divisible by 50.

        For the second proof, assume $x$ is odd and divisible by 50. Use
        these two assumptions to prove that $x$ is putrid.
    \end{enumerate}

    \vspace{1cm}

%%%%%%%%%%%%%%%%%%%%%%%%%%%%%%%%%%%%%%%%%%%%%%%%%%%%%%%%%%%%%%%%%%%%%%%%%%%%

    \noindent Problem 2.4.9  \vspace{0.3cm}

    \begin{enumerate}
        \item[3] \underline{proposition:} Given real numbers $x$ and $y$,
        $|x-y|=|y-x|$.

        \vspace{.2cm}

        \fbox{proof} (Direct, Cases)

        Let $x$ and $y$ be real numbers. There are two cases to consider:
        (i) when $x\ge y$, and (ii) when $x<y$.

        \underline{Case (i):}  Suppose $x\ge y$. Thus, $x-y\ge 0$. By
        definition of absolute value, this implies $|x-y|=x-y$. Furthermore,
        since $x-y\ge 0$, $-(x-y)=y-x<0$. Again, by definition of absolute
        value, this implies $|y-x|=-(y-x)=x-y$. Thus, if $x\ge y$,
        $|x-y|=|y-x|$.

        \underline{Case (ii):} Suppose $x<y$. It follows that $x-y<0$ and
        $y-x=-(x-y)>0$. By definition of absolute value, $|x-y|=-(x-y)=y-x$
        and $|y-x|=y-x$. Thus, if $x<y$, then $|x-y|=|y-x|$.

        Since in all cases we have $|x-y|=|y-x|$, it follows that
        $|x-y|=|y-x|$ for all real numbers $x$ and $y$.  $\blacksquare$
    \end{enumerate}

    \vspace{1cm}

%%%%%%%%%%%%%%%%%%%%%%%%%%%%%%%%%%%%%%%%%%%%%%%%%%%%%%%%%%%%%%%%%%%%%%%%

    \noindent Problem 2.4.11  \vspace{0.3cm}

    \begin{enumerate}
        \item[2] \underline{proposition:} Given real numbers $x$ and $y$,
        $\lfloor x+y \rfloor=\lfloor x\rfloor + \lfloor y \rfloor$ or
        $\lfloor x+y \rfloor=\lfloor x\rfloor + \lfloor y \rfloor + 1$.

        \vspace{0.4cm}

        \fbox{proof} (Direct Proof)

        \vspace{0.2cm}

        Let $x$ and $y$ be real numbers. Let $s=x-\lfloor x\rfloor$ and
        $t=\lfloor y\rfloor$. By definition of the floor function, $s$ and
        $t$ are real numbers with the property $0\le s<1$ and $0\le t<1$.
        Thus, $0=0+0\le s+t < 1+1=2$. We will consider the two possible
        cases: (i) $0\le s+t<1$, and (ii) $1\le s+t<2$.

        \underline{Case (i):}  Suppose $0\le s+t<1$. By our definition of
        $s$ and $t$, $x=\lfloor x\rfloor +s$, and $y=\lfloor y\rfloor +t$.
        Thus, $x+y=\lfloor x\rfloor +s + \lfloor y\rfloor +t$. Let $r=s+t$,
        since $0\le s+t<1$ it follows that $x+y=\lfloor x\rfloor +\lfloor
        y\rfloor +r$, where $\lfloor x\rfloor$ and $\lfloor y\rfloor$ are
        integers, and $r$ is a real number such that $0\le r<1$. By
        definition of the floor function if follows that $\lfloor
        x+y\rfloor=\lfloor\lfloor x\rfloor + \lfloor y\rfloor
        +r\rfloor=\lfloor x\rfloor + \lfloor y\rfloor$.

        \underline{Case (ii):}  Suppose $1\le s+t<2$. By our definition of
        $s$ and $t$, $x=\lfloor x\rfloor +s$, and $y=\lfloor y\rfloor +t$.
        Thus, $x+y=\lfloor x\rfloor +s + \lfloor y\rfloor +t$. Let
        $r=s+t-1$, since $1\le s+t<2$ it follows that $x+y=\lfloor x\rfloor
        +\lfloor y\rfloor 1+r$, where $\lfloor x\rfloor$ and $\lfloor
        y\rfloor$ are integers, and $r$ is a real number such that $0\le
        r<1$. By definition of the floor function if follows that $\lfloor
        x+y\rfloor=\lfloor\lfloor x\rfloor + \lfloor y\rfloor
        +1+r\rfloor=\lfloor x\rfloor + \lfloor y\rfloor+1$.


        Thus, for real numbers $x$ and $y$, $\lfloor x+y \rfloor=\lfloor
        x\rfloor + \lfloor y \rfloor$ or $\lfloor x+y \rfloor=\lfloor
        x\rfloor + \lfloor y \rfloor + 1. \blacksquare$
    \end{enumerate}

    \vspace{1cm}

%%%%%%%%%%%%%%%%%%%%%%%%%%%%%%%%%%%%%%%%%%%%%%%%%%%%%%%%%%%%%%%%%%%%%%%%%%%%%%


%%%%%%%%%%%%%%%%%%%%%%%%%%%%%%%%%%%%%%%%%%%%%%%%%%%%%%%%%%%%%%%%%%%%%%%%%%%%%

    \noindent Problem 2.5.1  \vspace{0.3cm}

    converting the statements to statements with explicit quantifiers,
    we find:

    \begin{enumerate}
        \item For all 5x5 matrices, if the matrix has positive determinant,
        then the matrix is bouncy.
        \item There exists an integer $x$ such that $x$ is crusty and $x>7$.
        \item For all integers $k$, there exists an integer $w$ such that
        $w$ is opulent and $k|w$.
        \item For all $m$ in the set of ribbed integers, there exists a 2x2
        matrix $P$ such that $P$ is fibrous and $\det(P)>m$.
        \item There exists a 2x2 matrix $M$ for all subtle integers $x$ such
        that $x|tr(M)$.
    \end{enumerate}

    \vspace{1cm}

%%%%%%%%%%%%%%%%%%%%%%%%%%%%%%%%%%%%%%%%%%%%%%%%%%%%%%%%%%%%%%%%%%%%%%%%%%%%


%%%%%%%%%%%%%%%%%%%%%%%%%%%%%%%%%%%%%%%%%%%%%%%%%%%%%%%%%%%%%%%%%%%%%%%%%%%%%

    \noindent Problem 2.5.4  \vspace{0.3cm}

    \begin{enumerate}
        \item[3] \underline{proposition:} For each integer $b$, there exists
        an integer $a$ such that $a|b$.

        \vspace{.2cm}

        \fbox{proof} (Direct Constructive)

        Let $b$ be an arbitrary integer, and set $a=b$. Thus, $a*1=b$,
        which by definition of divisor implies $a|b$. Thus, since $b$ was
        arbitrary, it follows that for all integers $b$, there exists and
        integer $a$ such that $a|b$. $\blacksquare$

        \item[4] \underline{proposition:} There exists an integer $a$ such
        that for all integers $b$, $a|b$.

        \vspace{.2cm}

        \fbox{proof} (Direct Constructive)

        Let $a$ be the integer 1. Let $b$ be any arbitrary integer. Since
        $1*b=b$, it follows by the definition of divisor that $1|b$. Thus,
        since $b$ was arbitrary, it follows that $a=1$ is a divisor of all
        integers. Thus, there exists and integer $a$ for all integers $b$,
        such that $a|b$.  $\blacksquare$

    \end{enumerate}

    \vspace{1cm}
    \newpage
%%%%%%%%%%%%%%%%%%%%%%%%%%%%%%%%%%%%%%%%%%%%%%%%%%%%%%%%%%%%%%%%%%%%%

    \noindent Problem 2.5.5  \vspace{0.3cm}

    \begin{enumerate}
        \item
        \underline{proposition:} For each real number $x$, there exists a
        real number $y$ such that $e^{x}-y>0$.

        \vspace{0.2cm}

        \fbox{proof} (Direct Constructive)

        \vspace{0.2cm}

        Let $x$ be an arbitrary real number. Let $y=-e^{x}$. Since the
        exponential function maps real numbers to real numbers, clearly $y$
        is a real number. Furthermore,
        $e^{x}-y=e^{x}+e^{x}=2e^{x}$. Since $\forall x\in\R$,
        $e^{x}>0$, it follows that $2e^{x}>0$. Thus, since $x$ was an
        arbitrary real number, for all real numbers $x$ there exists a real
        number $y$ such that $e^{x}-y>0$. $\blacksquare$

        \item
        \underline{proposition:} There exists a real number $y$ such that
        for all real numbers $x$, $e^{x}-y>0$.

        \vspace{0.2cm}

        \fbox{proof} (Direct Constructive)

        \vspace{0.2cm}

        Let $y=0$ and let $x$ be an arbitrary real number. Thus,
        $e^{x}-y=e^{x}-0=e^{x}$. Since $\forall x\in\R$, $e^{x}$ is
        positive, it follows that $e^{x}>0$. Thus, since $x$ was an
        arbitrary real number, there exists a real number $y$ for all real
        numbers $x$ such that $e^{x}-y>0$. $\blacksquare$

    \end{enumerate}

    \vspace{1cm}

%%%%%%%%%%%%%%%%%%%%%%%%%%%%%%%%%%%%%%%%%%%%%%%%%%%%%%%%%%%%%%%%%%%%%%%%%%%%%


%%%%%%%%%%%%%%%%%%%%%%%%%%%%%%%%%%%%%%%%%%%%%%%%%%%%%%%%%%%%%%%%%%%%%%%%%%


\end{document}