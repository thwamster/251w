\documentclass{article}
\usepackage[dvips]{graphicx}
\usepackage{a4wide}
\usepackage{amsmath}
\usepackage{euscript}
\usepackage{amssymb}
\usepackage{amsthm}
\usepackage{amsopn}

\theoremstyle{definition}
\newtheorem*{definition}{Definition}
\newtheorem{theorem}{Theorem}

\newcommand{\vv}{\ensuremath{\vec{v}}}
\newcommand{\vu}{\ensuremath{\vec{u}}}
\newcommand{\vw}{\ensuremath{\vec{w}}}
\newcommand{\vx}{\ensuremath{\vec{x}}}
\newcommand{\vy}{\ensuremath{\vec{y}}}
\newcommand{\vb}{\ensuremath{\vec{b}}}
\newcommand{\vo}{\ensuremath{\vec{0}}}
\newcommand{\va}{\ensuremath{\vec{a}}}
\newcommand{\ve}{\ensuremath{\vec{e}}}

\newcommand{\R}{\mathbb{R}}
\newcommand{\Z}{\mathbb{Z}}
\newcommand{\C}{\mathbb{C}}
\newcommand{\N}{\mathbb{N}}
\newcommand{\Q}{\mathbb{Q}}

%%%%%%%%%%%%%%%%%%%%%%%%%%%%%%%%%%%%%%%%%%%%%%%


\begin{document}
    \begin{center}
        \Large{Math 251W: Foundations of Advanced Mathematics}

        \normalsize Portfolio problems from sections 3.2
        $\&$ 3.3

        \hfill Name:
        \hrule \vspace{0.4cm}


        \vspace{0.75cm}
    \end{center}


    \noindent Problem 3.2.13  \vspace{0.3cm}


    \underline{proposition:} Given sets $A$ and $B$, If $A\subseteq B$,
    then ${\cal P}(A)\subseteq {\cal P}(B)$.

    \vspace{.2cm}

    \fbox{proof} () \vspace{.15cm}



    \vspace{1cm}

%%%%%%%%%%%%%%%%%%%%%%%%%%%%%%%%%%%%%%%%%%%%%%%%%%%%%%%%%%%%%%%%%%%%%%%%%%

    \noindent Problem 3.3.7  \vspace{0.3cm}

    \begin{enumerate}
        \item[iv]
        \underline{proposition:} Given sets $A$ and $B$, $B-(B-A)=A$ iff
        $A\subseteq B$.

        \vspace{.2cm}

        \fbox{proof} () \vspace{.15cm}



        \item[vi]
        \underline{proposition:} Given sets $A$ and $B$, if $A\subseteq B$
        then $C-A\supseteq C-B$.

        \vspace{.2cm}

        \fbox{proof} ( ) \vspace{.15cm}


    \end{enumerate}
    \vspace{1cm}


%%%%%%%%%%%%%%%%%%%%%%%%%%%%%%%%%%%%%%%%%%%%%%%%%%%%%%%%%%%%%%%%%%%%%%%%%%

    \noindent Problem 3.3.17  \vspace{0.3cm}

    Prove or give a counterexample

    \begin{enumerate}
        \item[1]
        Given sets $A$ and $B$, ${\cal P}(A\cup B)$ is necessarily equal
        to ${\cal P}(A)\cup{\cal P}(B)$.

        \vspace{.2cm}

        \fbox{Proof/Counterexample}



        \item[2] \underline{proposition:} Given sets $A$ and $B$, ${\cal
        P}(A\cap B)={\cal P}(A)\cap{\cal P}(B)$

        \vspace{.2cm}

        \fbox{Proof/Counterexample}


    \end{enumerate}

    \vspace{1cm}

%%%%%%%%%%%%%%%%%%%%%%%%%%%%%%%%%%%%%%%%%%%%%%%%%%%%%%%%%%%%%%%%%%%%%%%%%%




\end{document}