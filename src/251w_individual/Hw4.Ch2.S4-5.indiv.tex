% Program: PortfolioCh2S4-5.ltx
% Author: Teddy Wachtler
%
% Description: Portfolio Assignment 4: Sections 2.4 & 2.5
% *********************************************************

\documentclass{article}
\usepackage[dvips]{graphicx}
\usepackage{a4wide}
\usepackage{amsmath}
\usepackage{euscript}
\usepackage{amssymb}
\usepackage{amsthm}
\usepackage{amsopn}

\theoremstyle{definition}
\newtheorem*{definition}{Definition}
\newtheorem{theorem}{Theorem}

\renewcommand{\qed}{\blacksquare}

\newcommand{\vv}{\ensuremath{\vec{v}}}
\newcommand{\vu}{\ensuremath{\vec{u}}}
\newcommand{\vw}{\ensuremath{\vec{w}}}
\newcommand{\vx}{\ensuremath{\vec{x}}}
\newcommand{\vy}{\ensuremath{\vec{y}}}
\newcommand{\vb}{\ensuremath{\vec{b}}}
\newcommand{\vo}{\ensuremath{\vec{0}}}
\newcommand{\va}{\ensuremath{\vec{a}}}
\newcommand{\ve}{\ensuremath{\vec{e}}}

\newcommand{\R}{\mathbb{R}}
\newcommand{\Z}{\mathbb{Z}}
\newcommand{\C}{\mathbb{C}}
\newcommand{\N}{\mathbb{N}}
\newcommand{\Q}{\mathbb{Q}}

%%%%%%%%%%%%%%%%%%%%%%%%%%%%%%%%%%%%%%%%%%%%%%%%%%%%%%%%%%%

\begin{document}
	\begin{center}
		\Large{Math 251W: Foundations of Advanced Mathematics}

		\normalsize{Portfolio Assignment 4: \S 2.4 \& 2.5}

		\vspace{0.2cm}

		\hfill {\textbf Name:} Teddy Wachtler

		\vspace{0.25cm}
		\hrule
	\end{center}

	\vspace{0.3cm}

%%%%%%%%%%%%%%%%%%%%%%%%%%%%%%%%%%%%%%%%%%%%%%%%%%%%%%%%%%%

	\noindent Problem 2.4.8 [INDIVIDUAL]
	Prove the following
	\vspace{0.3cm}

	\underline{proposition:}
	If $n$ is an odd integer, then there exists an integer $k$ such that $n^2=8k+1$.

	\vspace{0.4cm}
	\fbox{proof} (Direct)
	\vspace{0.2cm}

	Let $n$ be an arbitrary odd integer.
	By the definition of odd integers, let $c$ be a particular integer so that $n = 2c + 1$.
	By taking both sides to the power of two, we see $n^2 = (2c + 1)^2$.
	By simplification and the distributive law, we see \[n^2 = 4c^2 + 4c + 1 = n^2 = 4(c^2 + c) + 1.\]
	Consider the parity of $c$ and $c^2 + c$.
	Consider $c_{even}$ to represent $c$ the case that $c$ is even.
	By the properties of even numbers, let $a$ be an integer so that $c_{even} = 2a$.
	By substitution, simplification, and the distributive law we see \[c_{even}^2 + c_{even} = (2a)^2 + 2a = 4a^2 + 2a = 2(a^2 + 2a).\]
	Consider $c_{odd}$ to represent $c$ in the case that $c$ is odd.
	By the properties of odd numbers, let $b$ be an integer so that $c_{odd} = 2b + 1$.
	By substitution, simplification, and the distributive law we see \[c_{odd}^2 + c_{odd} = (2b + 1)^2 + 2b + 1 = 4b^2 + 4b + 1 + 2b + 1 = 4b^2 + 6b + 2 = 2(b^2 + 3b + 1).\]
	Since addition is closed, let $k$ be an integer so that $k = a^2 + 2a$ or $k = b^2 + 3b + 1$.
	By substitution, we see $c_{odd}^2 + c_{odd} = 2k$ or $c_{even}^2 + c_{even} = 2k$.
	Since an integer must be even or odd, and since the value of $c^2 + c$ is the same irrespective of the parity of $c$, we see $c^2 + c = 2k$.
	By substitution, we see $n^2 = 4(2k) + 1$.
	By simplification, we see $n^2 = 8k + 1$.
	Therefore, for all odd integers $n$, there exists an integer $k$ so that $n^2 = 8k + 1$.
	$\qed$.

	\vspace{1cm}

\end{document}