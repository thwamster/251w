\documentclass{article}
\usepackage[dvips]{graphicx}
\usepackage{a4wide}
\usepackage{amsmath}
\usepackage{euscript}
\usepackage{amssymb}
\usepackage{amsthm}
\usepackage{amsopn}

\theoremstyle{definition}
\newtheorem*{definition}{Definition}
\newtheorem{theorem}{Theorem}

\newcommand{\vv}{\ensuremath{\vec{v}}}
\newcommand{\vu}{\ensuremath{\vec{u}}}
\newcommand{\vw}{\ensuremath{\vec{w}}}
\newcommand{\vx}{\ensuremath{\vec{x}}}
\newcommand{\vy}{\ensuremath{\vec{y}}}
\newcommand{\vb}{\ensuremath{\vec{b}}}
\newcommand{\vo}{\ensuremath{\vec{0}}}
\newcommand{\va}{\ensuremath{\vec{a}}}
\newcommand{\ve}{\ensuremath{\vec{e}}}

%%%%%%%%%%%%%%%%%%%%%%%%%%%%%%%%%%%%%%%%%%%%%%%


\begin{document}
\begin{center}
\Large{Math 251W: Foundations of Advanced Mathematics}

\normalsize Hints \& Solutions to SUGGESTED PROBLEMS from Section
1.4
\vspace{0.25cm}
\hrule
 \vspace{0.75cm}
\end{center}

\noindent {\it Note: The following are brief solutions or proofs for
selected problems.  Remember, the answer is the least important
part. It's understanding how to get the answer and how to explain
your process which is important.} \vspace{0.5cm}




\noindent \fbox{1.4.1(3)} \vspace{0.2cm}

$\begin{array}{llr} 1 & E\rightarrow F&\\
2 & \neg G\rightarrow \neg F&\\
3 & H\rightarrow I & \\
4 & E \vee H & \\\hline
5 & F\rightarrow G& 2, \mbox{Contrapositive}\\
6 & E\rightarrow G & 1, 5, \mbox{Hypothetical Syllogism}\\
7 & G\vee I & 3, 4, 6, \mbox{Constructive Dilema}
\end{array}$

\vspace{0.25cm}

\noindent \fbox{1.4.1(5)}\vspace{0.2cm}

Invalid argument.  Take the case where:
\begin{itemize}
\item $Q$ is False
\item $S$ is False
\item $P$ is False
\item $R$ is False
\item $T$ is True
\end{itemize}
Under these conditions, the four premises are all true, but $Q\vee
S$ is false.  Thus the premises do not imply the conclusion $Q\vee
S$.

\vspace{0.6cm}



\noindent \fbox{1.4.2(2)}  \vspace{0.2cm}

Let $L$ be the statement "the new CD by the Geeks is loud," $T$ be
the statement "the new CD by the Geeks is Tedious," $O$ be the
statement "the new CD by the Geeks is long," and $C$ be the
statement "the new CD by the Geeks is cacophonous." Thus the given
argument can be written symbolically as:

$$\begin{array}{llr}  & (L\vee T)\rightarrow (\neg O \wedge \neg C) &\\
 & T&\\\hline
 & \neg O &
\end{array}$$

We demonstrate this argument is valid with the following derivation.

$$\begin{array}{llr} 1 & (L\vee T)\rightarrow (\neg O \wedge \neg C) &\\
2 & T&\\\hline 3 & L\vee T & 2, \mbox{addition}\\
4 & \neg O \wedge \neg C & 1, 3, \mbox{modus ponens}\\
5 & \neg 0 & 4, \mbox{simplification}
\end{array}$$

\vspace{0.25cm}

\noindent \fbox{1.4.2(4)} \vspace{0.2cm}

Let $F$ be the statement "Susan likes fish," $O$ be the statement
"Susan likes onions," $G$ be the statement "Susan likes garlic," $A$
be the statement "Susan likes guava," and $C$ be the statement
"Susan likes cilantro." Thus the given argument can be written
symbolically as:

$$\begin{array}{llr}  & F\rightarrow O &\\
 & \neg G \rightarrow \neg O&\\
 & G\rightarrow A& \\
 & F\vee C& \\
 & \neg A &\\\hline
 & C &
\end{array}$$

We demonstrate this argument is valid with the following derivation.

$$\begin{array}{llr}1  & F\rightarrow O &\\
1 & \neg G \rightarrow \neg O&\\
3 & G\rightarrow A& \\
4 & F\vee C& \\
5 & \neg A &\\\hline 6 & \neg G & 5, 3, \mbox{modus tollens}\\
7 & \neg 0 & 6, 2, \mbox{modus ponens}\\
8 & \neg F & 7, 1, \mbox{modus tollens}\\
9 & C & 8, 4, \mbox{modus tollendo ponens}
\end{array}$$

\vspace{0.25cm}

\noindent \fbox{1.4.2(5)}  \vspace{0.2cm}

Let $G$ be the statement "Fred plays the guitar," $F$ be the
statement "Fred plays the flute," $O$ be the statement "Fred plays
the organ," and $H$ be the statement "Fred plays the harp." Thus the
given argument can be written symbolically as:

$$\begin{array}{llr}  & \neg(G\wedge F) &\\
 & (\neg G \wedge \neg F)\rightarrow (O\wedge H)&\\
 & H\rightarrow O &\\\hline
 & O &
\end{array}$$



This is an invalid argument.  Take the case where:
\begin{itemize}
\item $O$ is False
\item $H$ is False
\item $G$ is False
\item $F$ is True
\end{itemize}
Under these conditions, the three premises are all true, but $O$ is
false.  Thus the premises do not imply the conclusion.

\vspace{0.25cm}

\noindent \fbox{1.4.2(6)}  \vspace{0.2cm}

Let $R$ be the statement "you rob a bank," $F$ be the statement "you
have fun," $V$ be the statement "you go on vacation," and $J$ be the
statement "you go to jail." The given argument is valid as
demonstrated by the following derivation.

$$\begin{array}{llr}1  & R\rightarrow J &\\
2 & J \rightarrow \neg F&\\
3 & V\rightarrow F& \\
4 & R\vee V& \\\hline 5 & J\vee F & 1, 3, 5, \mbox{constructive
dilema}
\end{array}$$


\vspace{0.6cm}

\noindent \fbox{1.4.4:}

\begin{enumerate}
\item Fallacy of unwarranted assumptions
\item Fallacy of the inverse
\item Fallacy of the converse
\item Fallacy of the converse
\item Fallacy of unwarranted assumptions
\item Fallacy of the inverse
\end{enumerate}



\end{document}