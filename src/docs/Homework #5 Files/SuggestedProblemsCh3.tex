\documentclass{article}
\usepackage[dvips]{graphicx}
\usepackage{a4wide}
\usepackage{amsmath}
\usepackage{euscript}
\usepackage{amssymb}
\usepackage{amsthm}
\usepackage{amsopn}

\theoremstyle{definition}
\newtheorem*{definition}{Definition}
\newtheorem{theorem}{Theorem}

\newcommand{\vv}{\ensuremath{\vec{v}}}
\newcommand{\vu}{\ensuremath{\vec{u}}}
\newcommand{\vw}{\ensuremath{\vec{w}}}
\newcommand{\vx}{\ensuremath{\vec{x}}}
\newcommand{\vy}{\ensuremath{\vec{y}}}
\newcommand{\vb}{\ensuremath{\vec{b}}}
\newcommand{\vo}{\ensuremath{\vec{0}}}
\newcommand{\va}{\ensuremath{\vec{a}}}
\newcommand{\ve}{\ensuremath{\vec{e}}}

\newcommand{\R}{\mathbb{R}}
\newcommand{\Z}{\mathbb{Z}}
\newcommand{\C}{\mathbb{C}}
\newcommand{\N}{\mathbb{N}}
\newcommand{\Q}{\mathbb{Q}}

%%%%%%%%%%%%%%%%%%%%%%%%%%%%%%%%%%%%%%%%%%%%%%%


\begin{document}
    \begin{center}
        \Large{Math 251W: Foundations of Advanced Mathematics}

        \normalsize Solutions to problems from sections 3.2, 3.3, $\&$ 3.4
        \hrule \vspace{0.4cm}

        I'm a bit rushed to get these solutions done, so they are not
        proof-read. Nor are they necessarily good examples of how to write
        a complete proof. They are intended to be rough solutions for you
        to check your reasoning against.

        \vspace{0.75cm}
    \end{center}


    \noindent Problem 3.2.1  \vspace{0.3cm}

    \begin{enumerate}
        \item False
        \item True
        \item True
        \item True
        \item False
        \item False
        \item False
        \item True
        \item True
    \end{enumerate}

    \vspace{1cm}

%%%%%%%%%%%%%%%%%%%%%%%%%%%%%%%%%%%%%%%%%%%%%%%%%%%%%%%%%%%%%%%%%%%%%%%%%%%%

    \noindent Problem 3.2.2  \vspace{0.3cm}

    \begin{enumerate}
        \item Even integers
        \item Composite numbers
        \item Rational numbers, $\Q$
    \end{enumerate}

    \vspace{1cm}

%%%%%%%%%%%%%%%%%%%%%%%%%%%%%%%%%%%%%%%%%%%%%%%%%%%%%%%%%%%%%%%%%%%%%%%%

    \noindent Problem 3.2.3  \vspace{0.3cm}

    \begin{enumerate}
        \item Set of all fathers
        \item Set of all grandparents
        \item Set of all husbands
        \item Set of all people with siblings
        \item Set of all people except the oldest person alive
        \item Empty set, $\emptyset$
    \end{enumerate}

    \vspace{1cm}

%%%%%%%%%%%%%%%%%%%%%%%%%%%%%%%%%%%%%%%%%%%%%%%%%%%%%%%%%%%%%%%%%%%%%%%%%%%%%%


%%%%%%%%%%%%%%%%%%%%%%%%%%%%%%%%%%%%%%%%%%%%%%%%%%%%%%%%%%%%%%%%%%%%%%%%%%%%%

    \noindent Problem 3.2.4  \vspace{0.3cm}

    \begin{enumerate}
        \item $S=\{x\in\R|x>0\}$
        \item $S=\{x\in\Z|x=2m+1 \mbox{ for some } m\in\Z\}$
        \item $S=\{x\in\Q|\exists m,n\in\Z \mbox{ such that } n\not=0, \mbox{
            and } x=\frac{m}{5n}\}$
        \item $S=\{x\in\Z|\exists i\in\Z\mbox{ such that } -4\le i\le 4,
        \mbox{ and } x=i^3\}$
        \item $S=\{x\in\Z|x=1+4(n-1) \mbox{ for some } n\in\N\}$.
    \end{enumerate}

    \vspace{1cm}

%%%%%%%%%%%%%%%%%%%%%%%%%%%%%%%%%%%%%%%%%%%%%%%%%%%%%%%%%%%%%%%%%%%%%%%%%%%%


%%%%%%%%%%%%%%%%%%%%%%%%%%%%%%%%%%%%%%%%%%%%%%%%%%%%%%%%%%%%%%%%%%%%%%%%%%%%%

    \noindent Problem 3.2.9  \vspace{0.3cm}

    $A=\{a,b,\{a,b\}\}$ has 3 elements.

    \vspace{1cm}

%%%%%%%%%%%%%%%%%%%%%%%%%%%%%%%%%%%%%%%%%%%%%%%%%%%%%%%%%%%%%%%%%%%%%%%%%%%%

    \noindent Problem 3.2.12  \vspace{0.3cm}

    Given $A=\{x,y,z,w\}$,
    \begin{eqnarray}
        {\cal
        P}(A)&=&\{\{x\},\{y\},\{z\},\{w\},\{x,y\},\{x,z\},\{x,w\},\{y,z\},\{y,w\},\nonumber\\
        &
        &\{z,w\},\{x,y,z\},\{x,y,w\},\{x,z,w\},\{y,z,w\},\{x,y,z,w\},\emptyset\}\nonumber
    \end{eqnarray}
    \vspace{1cm}

%%%%%%%%%%%%%%%%%%%%%%%%%%%%%%%%%%%%%%%%%%%%%%%%%%%%%%%%%%%%%%%%%%%%%

    \noindent Problem 3.2.15  \vspace{0.3cm}

    \begin{enumerate}
        \item False
        \item True
        \item True
        \item True
        \item False
        \item True
        \item False
        \item False
        \item True

    \end{enumerate}

    \vspace{1cm}

%%%%%%%%%%%%%%%%%%%%%%%%%%%%%%%%%%%%%%%%%%%%%%%%%%%%%%%%%%%%%%%%%%%%%%%%%%%%%

%%%%%%%%%%%%%%%%%%%%%%%%%%%%%%%%%%%%%%%%%%%%%%%%%%%%%%%%%%%%%%%%%%%%%

    \noindent Problem 3.3.2  \vspace{0.3cm}

    Given $C=\{a,b,c,d,e,f\}$, $D=\{a,c,e\}$, $E=\{d,e,f\}$, and
    $F=\{a,b\}$

    \begin{enumerate}
        \item[1] $C-(D\cup E)=\{b\}$
        \item[2] $(C-D)\cup E=\{b,d,e,f\}$
        \item[3] $F-(C-E)=\emptyset$
        \item[4] $F\cap (D\cup E)=\{a\}$
        \item[5] $(F\cap D)\cup E=\{a,d,e,f\}$
        \item[6] $(C-D)\cup (F\cap E)=\{b,d,f\}$

    \end{enumerate}

    \vspace{1cm}

%%%%%%%%%%%%%%%%%%%%%%%%%%%%%%%%%%%%%%%%%%%%%%%%%%%%%%%%%%%%%%%%%%%%%%%%%%%%%

    \noindent Problem 3.3.3  \vspace{0.3cm}

    Given $X=[0,5)$, $Y=[2,4]$, $Z=(1,3]$, and $W=(3,5)$

    \begin{enumerate}
        \item[1] $Y\cup Z=(1,4]$
        \item[2] $Z\cap W=\emptyset$
        \item[3] $Y-W=[2,3]$
        \item[4] $X\times W=\{(x,y)\in\R^2|0\le x<5\mbox{ and } 3<y<5\}$
        \item[5] $(X\cap Y)\cup Z=(1,4]$
        \item[6] $X-(Z\cup W)=[0,1]$

    \end{enumerate}

    \vspace{1cm}

%%%%%%%%%%%%%%%%%%%%%%%%%%%%%%%%%%%%%%%%%%%%%%%%%%%%%%%%%%%%%%%%%%%%%%%%%%%%%
    \noindent Problem 3.3.7  \vspace{0.3cm}

    \begin{enumerate}
        \item[ii]
        \underline{proposition:} Given sets $A$ and $B$, $(A-B)\cap
        B=\emptyset$

        \vspace{.2cm}

        \fbox{proof} (Contradiction)

        Let $A$ and $B$ be sets. Suppose $(A-B)\cap B\not=\emptyset$. Thus,
        by definition of intersection, there exists an element $x$ such that
        $x\in (A-B)$ and $x\in B$. By definition of the difference of sets,
        $x\in (A-B)$ implies $x\in A$ and $x\not\in B$. However, this
        contradicts $x\in B$. Thus, there does not exist an element $x\in
        (A-B)\cap B$.  $(A-B)\cap B=\emptyset$.  $\blacksquare$

    \end{enumerate}
    \vspace{1cm}

%%%%%%%%%%%%%%%%%%%%%%%%%%%%%%%%%%%%%%%%%%%%%%%%%%%%%%%%%%%%%%%%%%%%%%%%%%

    \noindent Problem 3.3.8  \vspace{0.3cm}

    \begin{enumerate}
        \item[i]
        \underline{proposition:} Given sets $A$, $B$, $C$, and $D$, if
        $A\subseteq B$ and $C\subseteq D$, then $A\times C\subseteq B\times
        D$

        \vspace{.2cm}

        \fbox{proof} (Direct)

        Let $A$, $B$, $C$, and $D$ be sets such that $A\subseteq B$ and
        $C\subseteq D$. Let $(x,y)$ be an arbitrary element of $A\times C$.
        By definition of the cartesian product, since $(x,y)\in A\times C$
        it follows that $x\in A$ and $y\in C$. Since $A\subseteq B$ and
        $x\in A$, it follows by the definition of subset that $x\in B$.
        Similarly, since $y\in C\subseteq D$, $y\in D$. Thus, by definition
        of the cartesian product, $(x,y)\in B\times D$. Thus, since $(x,y)$
        was an arbitrary element of $A\times C$, if follows that for all
        $(x,y)\in A\times C$, $(x,y)$ is also an element of $B\times D$. By
        definition of subset, $A\times C \subseteq B\times D$.
        $\blacksquare$
    \end{enumerate}
    \vspace{1cm}

%%%%%%%%%%%%%%%%%%%%%%%%%%%%%%%%%%%%%%%%%%%%%%%%%%%%%%%%%%%%%%%%%%%%%%%%%%


%%%%%%%%%%%%%%%%%%%%%%%%%%%%%%%%%%%%%%%%%%%%%%%%%%%%%%%%%%%%%%%%%%%%%%%%%%

    \noindent Problem 3.3.19  \vspace{0.3cm}

    \underline{proposition:} If $A$ and $B$ be are sets such that
    $B\subseteq A$, then $A\times A-B\times B=[(A-B)\times A]\cup
    [A\times (A-B)]$.

    \vspace{.3cm}

    \fbox{Proof} (Direct)

    \vspace{.2cm}

    Let $(x,y)$ be an arbitrary element of the set $A\times A-B\times
    B$. By definition of the difference of sets, $(x,y)\in A\times A$
    and $(x,y)\not\in B\times B$. By the definition of the cartesian
    product, $(x,y)\in A\times A$ implies $x\in A$ and $y\in A$,
    $(x,y)\not\in B\times B$ implies $x\not\in B$ OR $y\not\in B$. Thus
    we have $x \in A-B$ and $y \in A$, or $x \in A$ and $y\in A-B$. By
    definition of the cartesian product, this implies $(x,y)\in
    (A-B)\times A$ or $(x,y)\in A\times (A-B)$. Thus, by the definition
    of the union of sets, $(x,y)\in [(A-B)\times A]\cup [A\times
    (A-B)]$. Since $(x,y)$ was an arbitrary element of $A\times
    A-B\times B$, it follows that $A\times A-B\times
    B\subseteq[(A-B)\times A]\cup [A\times (A-B)]$.

    \vspace{.2cm}

    Let $(x,y)$ be an arbitrary element of $[(A-B)\times A]\cup [A\times
    (A-B)]$. Thus, $(x,y)\in (A-B)\times A$, or $(x,y)\in A\times
    (A-B)$.

    \underline{Case 1:}  Suppose $(x,y)\in (A-B)\times A$, thus $x\in
    (A-B)$ and $y\in A$.  $x\in (A-B)$ implies $x\in A$ and $x\not\in
    B$. Since both $x$ and $y$ are in $A$, it follows that $(x,y)\in
    A\times A$. Furthermore, since $x\not\in B$, $(x,y)\not\in B\times
    B$. Thus, $(x,y)\in A\times A-B\times B$.

    \underline{Case 2:}  Suppose $(x,y)\in A\times(A-B)$, thus $x\in A$
    and $y\in (A-B)$. Since $y\in (A-B)$, it follows that $y\in A$ and
    $y\not\in B$. Since $x$ and $y$ are both in $A$, $(x,y)\in A\times
    A$. Furthermore, since $y\not\in B$, $(x,y)\not\in B\times B$.
    Thus, $(x,y)\in A\times A-B\times B$.

    Since $(x,y)$ was an arbitrary element of $[(A-B)\times A]\cup
    [A\times (A-B)]$, and in either case $(x,y)\in A\times A-B\times B$,
    it follows that $A\times A-B\times B\supseteq[(A-B)\times A]\cup
    [A\times (A-B)]$.

    Therefore, since $A\times A-B\times B\subseteq[(A-B)\times A]\cup
    [A\times (A-B)]$ and $A\times A-B\times B\supseteq[(A-B)\times
    A]\cup [A\times (A-B)]$, $A\times A-B\times B=[(A-B)\times A]\cup
    [A\times (A-B)]$. $\blacksquare$

    \vspace{1cm}

%%%%%%%%%%%%%%%%%%%%%%%%%%%%%%%%%%%%%%%%%%%%%%%%%%%%%%%%%%%%%%%%%%%%%%%%%%


    \noindent Problem 3.4.1  \vspace{0.3cm}


    \begin{enumerate}
%\item[1] $\bigcup_{k\in\N} B_k=\{0\}\cup \N$ and $\bigcap_{k\in\N}
%B_k=\{0,1,2\}$
        \item[2] $\bigcup_{k\in\N} B_k=\N$ and $\bigcap_{k\in\N}
        B_k=\emptyset$
        \item[3] $\bigcup_{k\in\N} B_k=(0,7)\cup\{11, 12, \ldots\}$ and $\bigcap_{k\in\N}
        B_k=[3,5]$
        \item[4] $\bigcup_{k\in\N} B_k=[-1,4]\cup[5,6)$ and $\bigcap_{k\in\N}
        B_k=[-1,3]\cup\{5\}$
        \item[5] $\bigcup_{k\in\N} B_k=(-1,1]\cup(2,3)$ and $\bigcap_{k\in\N}
        B_k=(0,1]\cup\{2\}$
        \item[6] $\bigcup_{k\in\N} B_k=[0,1)\cup[7,8)$ and $\bigcap_{k\in\N}
        B_k=[0,\frac{2}{3}]\cup\{7\}$

    \end{enumerate}

    \vspace{1cm}

%%%%%%%%%%%%%%%%%%%%%%%%%%%%%%%%%%%%%%%%%%%%%%%%%%%%%%%%%%%%%%%%%%%%%%%%%%%%%


    \noindent Problem 3.4.2  \vspace{0.3cm}


    \begin{enumerate}
%\item[3] $B_k=[3-(k-1),3+(k-1)]$
        \item[4] $B_k=(3-\frac{1}{k},6+\frac{2}{k})$
        \item[5] $B_k=[1-\frac{1}{k},k]\cap[2,2+k)$
    \end{enumerate}

    \vspace{1cm}

%%%%%%%%%%%%%%%%%%%%%%%%%%%%%%%%%%%%%%%%%%%%%%%%%%%%%%%%%%%%%%%%%%%%%%%%%%%%%



\end{document}