\documentclass{article}
\usepackage[dvips]{graphicx}
\usepackage{a4wide}
\usepackage{amsmath}
\usepackage{euscript}
\usepackage{amssymb}
\usepackage{amsthm}
\usepackage{amsopn}

\theoremstyle{definition}
\newtheorem*{definition}{Definition}
\newtheorem{theorem}{Theorem}

\newcommand{\vv}{\ensuremath{\vec{v}}}
\newcommand{\vu}{\ensuremath{\vec{u}}}
\newcommand{\vw}{\ensuremath{\vec{w}}}
\newcommand{\vx}{\ensuremath{\vec{x}}}
\newcommand{\vy}{\ensuremath{\vec{y}}}
\newcommand{\vb}{\ensuremath{\vec{b}}}
\newcommand{\vo}{\ensuremath{\vec{0}}}
\newcommand{\va}{\ensuremath{\vec{a}}}
\newcommand{\ve}{\ensuremath{\vec{e}}}

\newcommand{\R}{\mathbb{R}}
\newcommand{\Z}{\mathbb{Z}}
\newcommand{\C}{\mathbb{C}}
\newcommand{\N}{\mathbb{N}}
\newcommand{\Q}{\mathbb{Q}}

%%%%%%%%%%%%%%%%%%%%%%%%%%%%%%%%%%%%%%%%%%%%%%%


\begin{document}
\begin{center}
\Large{Math 251W: Foundations of Advanced Mathematics}

\normalsize Solutions to problems from section 4.3 
\hrule
\vspace{0.4cm}

\end{center}


\noindent Problem 4.3.2  \vspace{0.3cm}

\begin{enumerate}
\item $g(x)=x+7$, $h(x)=\sqrt{|x|}$   (Note: there should be an
absolute value in the function $f$ since otherwise the domain of $f$
is not $\R$
\item $g(x)=\sqrt{|x|}$, $h(x)=x+7$
\item $g(x)=x^2$, $h(x)=\left\{\begin{array}{cc} x^3 & \mbox{ if
}x\ge 0\\ x^2 & \mbox{ if } x<0\end{array}\right.$
\item $g(x)=\left\{\begin{array}{cc} x^6 & \mbox{ if } x\ge0\\ x &
\mbox{ if } x<0\end{array}\right.$, $h(x)=\left\{\begin{array}{cc} \sqrt{x}
& \mbox{ if } x\ge0\\ x & \mbox{ if } x<0\end{array}\right.$
\end{enumerate}

\vspace{1cm}

%%%%%%%%%%%%%%%%%%%%%%%%%%%%%%%%%%%%%%%%%%%%%%%%%%%%%%%%%%%%%%%%%%%%%%%%%%%%

\noindent Problem 4.3.4  \vspace{0.3cm}

\begin{enumerate}
\item $h(x)=\left\{\begin{array}{cc} \frac{1}{5}
& \mbox{ if } x\ge0\\ -2 & \mbox{ if } x<0\end{array}\right.$,
$k(x)=\left\{\begin{array}{cc} 10x & \mbox{ if } x\ge0\\
\frac{x}{-1} & \mbox{ if } x<0\end{array}\right.$
\item $s(x)=\frac{x}{5}$, $t(x)=5x$
\end{enumerate}

\vspace{1cm}

%%%%%%%%%%%%%%%%%%%%%%%%%%%%%%%%%%%%%%%%%%%%%%%%%%%%%%%%%%%%%%%%%%%%%%%%

\noindent Problem 4.3.5  \vspace{0.3cm}

Let $f:A\rightarrow B$ and $g:B\rightarrow C$ be functions, and let
$U\subseteq A$ and $V\subseteq C$.

\vspace{0.3cm}

\underline{proposition:} $(g\circ f)_*(U)=g_*(f_*(U))$

\vspace{0.2cm}

\fbox{proof}

\vspace{0.15cm}

 Let $y$ be an arbitrary element of $(g\circ f)_*(U)$.  Thus, by
 definition of image, $\exists x\in U$ such that $(g\circ f)(x)=y$.
 By definition of composition, $(g\circ f)(x)=g(f(x)).$  Since $x\in
 U$, $f(x)\in f_*(U)$, and since $f(x)\in f_*(U)$, $g(f(x))\in
 g_*(f_*(U))$.  Thus $y=g(f(x))\in g_*(f_*(U))$, which implies $(g\circ
 f)_*(U)\subseteq g_*(f_*(U))$.

Let $y$ be an arbitrary element of $g_*(f_*(U))$.  By definition of
image, there exists an element $b\in f_*(U)$ such that $y=g(b)$.
Since $b\in f*(U)$, there must exist an element $x$ in $U$ such that
$f(x)=b$.  Thus, $y=g(b)=g(f(x))=(g\circ f)(x)$.  By definition of
image, since $x\in U$ and $y=(g\circ f)(x)$, $y\in (g\circ f)_*(U)$.
Thus, $(g\circ
 f)_*(U)\supseteq g_*(f_*(U))$.

Since we've shown $(g\circ
 f)_*(U)\subseteq g_*(f_*(U))$ and $(g\circ
 f)_*(U)\supseteq g_*(f_*(U))$, $(g\circ
 f)_*(U)= g_*(f_*(U))$.  $\blacksquare$


\vspace{0.3cm}

\underline{proposition:} $(g\circ f)^*(V)=f^*(g^*(V))$

\vspace{0.2cm}

\fbox{proof}

\vspace{0.15cm}

Let $x$ be an arbitrary element of $(g\circ f)^*(V)$.  Thus, by
definition of inverse image, $(g\circ f)(x)=g(f(x))\in V$.  Since
$g(f(x))\in V$ it follows that $f(x)\in g^*(V)$, which implies $x\in
f^*(g^*(V))$, again by definition of inverse image.  Thus, since $x$
was an arbitrary element, it follows that $(g\circ f)^*(V)\subseteq
f^*(g^*(V))$.

Let $x$ be an arbitrary element of $f^*(g^*(V))$, i.e. $f(x)\in
g^*(V)$.  By definition of inverse image, $f(x)\in g^*(V)$ implies
$g(f(x))=(g\circ f)(x)\in V$.  Again, by definition of inverse
image, this implies $x\in (g\circ f)^*(V)$.  Thus, since $x$ was an
arbitrary element, $(g\circ f)^*(V)\supseteq f^*(g^*(V))$.

Having showed $(g\circ f)^*(V)\subseteq f^*(g^*(V))$ and $(g\circ
f)^*(V)\supseteq f^*(g^*(V))$, it follows that $(g\circ f)^*(V)=
f^*(g^*(V))$.  $\blacksquare$

\vspace{1cm}

%%%%%%%%%%%%%%%%%%%%%%%%%%%%%%%%%%%%%%%%%%%%%%%%%%%%%%%%%%%%%%%%%%%%%%%%%%%%%%


%%%%%%%%%%%%%%%%%%%%%%%%%%%%%%%%%%%%%%%%%%%%%%%%%%%%%%%%%%%%%%%%%%%%%%%%%%%%%

\noindent Problem 4.3.6  \vspace{0.3cm}

\underline{proposition:}  Let $f:A\rightarrow B$ and $g:B\rightarrow
C$ be functions with inverse functions $f^{-1}:B\rightarrow A$ and
$g^{-1}:C\rightarrow B$.  The inverse of the function $g\circ f$ is
$f^{-1}\circ g^{-1}$.

\vspace{0.2cm}

\fbox{proof}

\vspace{0.15cm}

By definition of composition, $f^{-1}\circ g^{-1}$ is a function
mapping $C\rightarrow A$.  Consider the composition $(f^{-1}\circ
g^{-1})\circ (g\circ f)$.  By the associativity of composition,
$(f^{-1}\circ g^{-1})\circ (g\circ f)=f^{-1}\circ(g^{-1}\circ
g)\circ f$.  By the definition of inverse functions $g^{-1}\circ
g=1_B$, $g\circ g^{-1}=1_C$, $f\circ f^{-1}=1_B$, and $f^{-1}\circ
f=1_A$. Thus, by the identity law and the definition of inverse
functions, $f^{-1}\circ (g^{-1}\circ g)\circ f=f^{-1}\circ (1_B\circ
f)=f^{-1}\circ f=1_A$. Since $(f^{-1}\circ g^{-1})\circ (g\circ
f)=1_A$, $f^{-1}\circ g^{-1}$ is the left inverse of $g\circ f$.

Consider the composition $(g\circ f)\circ(f^{-1}\circ g^{-1})$.  By
the associativity of composition, the definition of inverses, and
the identity law $(g\circ f)\circ(f^{-1}\circ g^{-1})=g\circ (f\circ
f^{-1})\circ g^{-1}=g\circ(1_B\circ g^{-1})=g\circ g^{-1}=1_C$.  By
definition or right inverse, this implies $f^{-1}\circ g^{-1}$ is
the right inverse of $g\circ f$.

Thus, since $f^{-1}\circ g^{-1}$ is the left and right inverse, it
follows by definition of inverse that $f^{-1}\circ g^{-1}$ is the
inverse of $g\circ f$. $\blacksquare$

\vspace{1cm}

%%%%%%%%%%%%%%%%%%%%%%%%%%%%%%%%%%%%%%%%%%%%%%%%%%%%%%%%%%%%%%%%%%%%%%%%%%%%


\noindent Problem 4.3.7  \vspace{0.3cm}

\begin{enumerate}
\item[1]  $g_1:[0,\infty)\rightarrow \R$ defined by $g_1(x)=x$ for all $x\in [0,\infty)$, and
$g_2:[0,\infty)\rightarrow \R$ defined by $g_2(x)=-x$ for all $x\in
[0,\infty)$.
\end{enumerate}

\vspace{1cm}

%%%%%%%%%%%%%%%%%%%%%%%%%%%%%%%%%%%%%%%%%%%%%%%%%%%%%%%%%%%%%%%%%%%%%%%%%%%%%%%%

\noindent Problem 4.3.8  \vspace{0.3cm}

\begin{enumerate}
\item[2] $f_1:\R\rightarrow \R$ defined by $f_1(x)=ln|x|$ for all
$x\in \R$, and $f_2:\R\rightarrow \R$ defined by
$$f_2(x)=\left\{\begin{array}{cc} ln(x) & \mbox{ if } x>0\\ x &
\mbox{ if }x\le 0\end{array}\right.$$ for all $x\in \R$.
\end{enumerate}

\vspace{1cm}


%%%%%%%%%%%%%%%%%%%%%%%%%%%%%%%%%%%%%%%%%%%%%%%%%%%%%%%%%%%%%%%%%%%%%%%%%%



\end{document}