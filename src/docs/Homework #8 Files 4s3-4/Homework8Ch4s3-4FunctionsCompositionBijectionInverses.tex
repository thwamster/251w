% Program: PortfolioCh4s3-4.tex
% Author: Erin McNicholas
%
%
% Note: Comments made after a % sign are not read by the compiler.
%       are notational comments in the code.
% *********************************************************************

% *********************************************************************
%     Header Commands: These are commands that format the document and
%         define new command shortcuts.  You can use the \newcommand
%         function to define shortcuts for commonly used commands.
%         If you are just learning LaTeX, you should not need to 
%         modify this portion of the code
% *********************************************************************


\documentclass{article}
\usepackage[dvips]{graphicx}
\usepackage{a4wide}
\usepackage{amsmath}
\usepackage{euscript}
\usepackage{amssymb}
\usepackage{amsthm}
\usepackage{amsopn}

\theoremstyle{definition}
\newtheorem*{definition}{Definition}
\newtheorem{theorem}{Theorem}

\newcommand{\vv}{\ensuremath{\vec{v}}}
\newcommand{\vu}{\ensuremath{\vec{u}}}
\newcommand{\vw}{\ensuremath{\vec{w}}}
\newcommand{\vx}{\ensuremath{\vec{x}}}
\newcommand{\vy}{\ensuremath{\vec{y}}}
\newcommand{\vb}{\ensuremath{\vec{b}}}
\newcommand{\vo}{\ensuremath{\vec{0}}}
\newcommand{\va}{\ensuremath{\vec{a}}}
\newcommand{\ve}{\ensuremath{\vec{e}}}
%%%%%%%%%%%%%%%%%%%%%%%%%%%%%%%%%%%%%%%%%%%%%%%

% *********************************************************************
%     \begin{document} starts the portion of the code that will 
%         be translated into text.  This is the portion of the code
%         you will modify to insert your answers in
% *********************************************************************
\begin{document}
\begin{center}
\Large{Math 251W: Foundations of Advanced Mathematics}

\normalsize  Portfolio problems from section 4.3 \& 4.4

\hfill Name: 

\vspace{0.4cm}
\hrule \vspace{0.4cm}


\vspace{0.75cm}
\end{center}




%%%%%%%%%%%%%%%%%%%%%%%%%%%%%%%%%%%%%%%%%%%%%%%%%%%%%%%%%%%%%%%%%%%%%%%%%%


\noindent Problem 4.3.10  \vspace{0.3cm}


\underline{proposition:} Let $f:A\rightarrow B$ be a function.
\begin{itemize}
\item[i] If $f$ has two distinct left inverses, it has no right inverse.
\end{itemize}

\vspace{.2cm}

\fbox{proof} ( ) \vspace{.15cm}



\vspace{1cm}


%%%%%%%%%%%%%%%%%%%%%%%%%%%%%%%%%%%%%%%%%%%%%%%%%%%%%%%%%%%%%%%%%%%%%%%%%%

\noindent Problem 4.3.11  \vspace{0.3cm}

\underline{proposition:} Let $f:A\rightarrow A_1\times A_2\times
\dots \times A_k$ be a function, and let $U_i\subseteq A_i$ for each
$i\in\{1,\ldots,k\}$, then $$f^*(U_1\times U_2\times \dots \times
U_k)=\bigcap_{i=1}^k f_i^*(U_i)$$
where $f_i$ is the $i^{th}$ coordinate function, $f_i=\pi_i\circ f$

\vspace{.2cm}

\fbox{proof} ( ) \vspace{.15cm}



\vspace{1cm}



%%%%%%%%%%%%%%%%%%%%%%%%%%%%%%%%%%%%%%%%%%%%%%%%%%%%%%%%%%%%%%%%%%%%%%%%%%

%%%%%%%%%%%%%%%%%%%%%%%%%%%%%%%%%%%%%%%%%%%%%%%%%%%%%%%%%%%%%%%%%%%%%%%%%%




\noindent Problem 4.4.7  \vspace{0.3cm}

\underline{proposition:}  The function $\phi :{\cal P}(A)\rightarrow
{\cal P}(A)$, defined by $\phi (X)=A-X$ for all $X\in {\cal P}(A)$,
is bijective.

\vspace{.2cm}

\fbox{proof} ( ) \vspace{.15cm}



\vspace{1cm}

%%%%%%%%%%%%%%%%%%%%%%%%%%%%%%%%%%%%%%%%%%%%%%%%%%%%%%%%%%%%%%%%%%%%%%%%%%%%


\noindent Problem 4.4.14  \vspace{0.3cm}

\begin{enumerate}
\item[3] \underline{proposition:} Given functions $f:A\rightarrow B$, and $g:B\rightarrow C$, if $g\circ f$ is bijective, then
$f$ must be injective and $g$ must be surjective.

\end{enumerate}
\vspace{0.2cm}

\fbox{proof} ( )



\vspace{1cm}

%%%%%%%%%%%%%%%%%%%%%%%%%%%%%%%%%%%%%%%%%%%%%%%%%%%%%%%%%%%%%%%%%%%%%%%%%%%%%%%%



\noindent Problem 4.4.17  \vspace{0.3cm}

\underline{proposition:} Let $f:A\rightarrow B$ be a map.  $f$ is
surjective iff $B-f_*(X)\subseteq f_*(A-X)$ for all $X\subseteq A$.

\vspace{0.2cm}

\fbox{proof}

\vspace{0.2cm}




\vspace{1cm}

%%%%%%%%%%%%%%%%%%%%%%%%%%%%%%%%%%%%%%%%%%%%%%%%%%%%%%%%%%%%%%%%%%%%%%%%%%



\end{document}