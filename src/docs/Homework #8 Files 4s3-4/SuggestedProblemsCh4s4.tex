\documentclass{article}
\usepackage[dvips]{graphicx}
\usepackage{a4wide}
\usepackage{amsmath}
\usepackage{euscript}
\usepackage{amssymb}
\usepackage{amsthm}
\usepackage{amsopn}

\theoremstyle{definition}
\newtheorem*{definition}{Definition}
\newtheorem{theorem}{Theorem}

\newcommand{\vv}{\ensuremath{\vec{v}}}
\newcommand{\vu}{\ensuremath{\vec{u}}}
\newcommand{\vw}{\ensuremath{\vec{w}}}
\newcommand{\vx}{\ensuremath{\vec{x}}}
\newcommand{\vy}{\ensuremath{\vec{y}}}
\newcommand{\vb}{\ensuremath{\vec{b}}}
\newcommand{\vo}{\ensuremath{\vec{0}}}
\newcommand{\va}{\ensuremath{\vec{a}}}
\newcommand{\ve}{\ensuremath{\vec{e}}}

\newcommand{\R}{\mathbb{R}}
\newcommand{\Z}{\mathbb{Z}}
\newcommand{\C}{\mathbb{C}}
\newcommand{\N}{\mathbb{N}}
\newcommand{\Q}{\mathbb{Q}}

%%%%%%%%%%%%%%%%%%%%%%%%%%%%%%%%%%%%%%%%%%%%%%%


\begin{document}
    \begin{center}
        \Large{Math 251W: Foundations of Advanced Mathematics}

        \normalsize Solutions to suggested problems from section 4.4 \hrule
        \vspace{0.4cm}

    \end{center}


    \noindent Problem 4.4.1  \vspace{0.3cm}

    \begin{enumerate}
        \item $t$ is injective, but not surjective
        \item $s$ is neither surjective, nor injective
        \item $g$ is bijective
        \item $k$ is neither injective, nor surjective
        \item $Q$ is injective, but not surjective
    \end{enumerate}

    \vspace{1cm}

%%%%%%%%%%%%%%%%%%%%%%%%%%%%%%%%%%%%%%%%%%%%%%%%%%%%%%%%%%%%%%%%%%%%%%%%%%%%

    \noindent Problem 4.4.2  \vspace{0.3cm}

    \begin{enumerate}
        \item injective, but not surjective
        \item bijective
        \item injective, but not surjective
        \item bijective
    \end{enumerate}

    \vspace{1cm}

%%%%%%%%%%%%%%%%%%%%%%%%%%%%%%%%%%%%%%%%%%%%%%%%%%%%%%%%%%%%%%%%%%%%%%%%

    \noindent Problem 4.4.3  \vspace{0.3cm}

    \begin{enumerate}
        \item[3] Yes, $f|_S$ is necessarily injective
        \item[4] No, $g|_S$ is not necessarily surjective. Counterexample:
        $A=\{a,b,c,d\}$, $B=\{1,2,3\}$, $S=\{a,b\}$, and
        $g=\{(a,1),(b,2),(c,3),(d,3)\}$. Thus, $g|_S=\{(a,1),(b,2)\}$ is
        not surjective as there is no element $x\in S$ such that
        $g|_S(x)=3$.
        \item[5] No
        \item[6] \underline{proposition:} If $k:S\rightarrow B$ is
        surjective, then the extension $K:A\rightarrow B$ is also
        surjective.

        \vspace{0.2cm}

        \fbox{proof}

        \vspace{0.2cm}

        Let $b$ be an arbitrary element of $B$. Since $k$ is surjective,
        $\exists s\in S$ such that $k(s)=b$. Since $S\subseteq A$, $s\in
        A$. Furthermore, by definition of the extension, $K(s)=k(s)$ for
        all $s\in S$. Thus, $K(s)=b$. Therefore, since $b$ was arbitrary
        it follows that for all $b\in B$, $\exists s\in A$ such that
        $K(s)=b$. Thus, by definition, $K$ is surjective.  $\blacksquare$

        \item[7] The projection maps are not injective.

    \end{enumerate}

    \vspace{1cm}

%%%%%%%%%%%%%%%%%%%%%%%%%%%%%%%%%%%%%%%%%%%%%%%%%%%%%%%%%%%%%%%%%%%%%%%%%%%%%%

    \newpage

%%%%%%%%%%%%%%%%%%%%%%%%%%%%%%%%%%%%%%%%%%%%%%%%%%%%%%%%%%%%%%%%%%%%%%%%%%%%%

    \noindent Problem 4.4.9  \vspace{0.3cm}

    \underline{proposition:}  (Thm 4.4.3(ii)) Given $A$ and $B$ are
    non-empty sets and $f:A\rightarrow B$ is a function, $f$ has a left
    inverse iff $f$ is injective.

    \vspace{0.2cm}

    \fbox{proof}

    \vspace{0.15cm}

    ($\Rightarrow$)  Suppose $f$ has a left inverse $f_L^{-1}$. Let $x$
    and $y$ be two arbitrary elements of $A$ such that $f(x)=f(y)$.
    Since the left inverse is a function, and functions map equal inputs
    to equal outputs, it follows that $f_L^{-1}(f(x))=f_L^{-1}(f(y))$.
    Furthermore, by definition of left inverse $f_L^{-1}\circ f=1_A$.
    Thus  $$f_L^{-1}(f(x))=(f_L^{-1}\circ f)(x)=1_A(x)=x,$$ and
    $$f_L^{-1}(f(y))=(f_L^{-1}\circ f)(y)=1_A(y)=y.$$  Therefore, $f_L^{-1}(f(x))=f_L^{-1}(f(y))$ implies
    $x=y$. Since $x$ and $y$ were arbitrary elements of the domain for
    which $f(x)=f(y)$ it follows that $f$ is injective.

    \vspace{0.15cm}

    ($\Leftarrow$)  Suppose $f$ is injective. By definition of
    injectivity, $\forall y\in f_*(A)$ there exists only one element
    $x\in A$ such that $f(x)=y$. Let $a$ be a particular element of
    $A$. Define the function $h:B\rightarrow A$ by
    $h(y)=\left\{\begin{array}{cc}
                     x,\mbox{ where }f(x)=y & \forall y\in
                     f_*(A)\\ a & \forall y\in B-f_*(A)
    \end{array}\right\}$. Since $f$ is
    injective, this definition of $h$ is well defined. Let $x$ be an
    arbitrary element of $A$, $(h\circ f)(x)=h(f(x))=h(y)$ where $y\in
    f_*(A)$. Thus, by definition $h(y)=x$. Since $x$ was arbitrary it
    follows that for all $x\in A$, $(h\circ f)(x)=x$. By definition of
    the left inverse, this implies $h$ is the left inverse of $f$.
    $\blacksquare$

    \vspace{1cm}

%%%%%%%%%%%%%%%%%%%%%%%%%%%%%%%%%%%%%%%%%%%%%%%%%%%%%%%%%%%%%%%%%%%%%%%%%%%%


    \noindent Problem 4.4.13  \vspace{0.3cm}

    \begin{enumerate}
%\item[2] \underline{proposition:} Given functions $f:A\rightarrow B$, and $g:B\rightarrow C$, if $g\circ f$ is surjective, then
%$g$ must be surjective.
%
%\vspace{0.2cm}
%
%\fbox{proof} (By Contradiction)
%
%\vspace{0.2cm}
%
%Suppose $g\circ f$ is surjective, but $g$ is not surjective. Thus,
%there exists an element $y\in C$ for which there is no element $b\in
%B$ such that $g(b)=y$.  If $g\circ f$ is surjective, then there
%exists an element $x\in A$ such that $(g\circ f)(x)=g(f(x))=y$. By definition of
%the function $f$, $f(x)=b$ for some $b\in B$.  Thus,
%$g(f(x))=g(b)=y$. This contradicts our earlier conclusion that there
%did not exist any $b\in B$ such that $g(b)=y$.  Thus, if $g\circ f$
%is surjective, $g$ must be surjective.  $\blacksquare$

        \item[4] Let $A=\{a,b,c\}, B=\{1,2,3,4\}, C=\{x,y,z\}$,
        $f:A\rightarrow B$ be the function $f=\{(a,1),(b,2),(c,3)\}$, and
        $g:B\rightarrow C$ be the function $g=\{(1,x),(2,y),(3,z),(4,z)\}$.
        Thus, $g\circ f:A\rightarrow C$ is the function $g\circ f=\{(a,x),
        (b,y),(c,z)]\}$.  $f$ is not surjective, and $g$ is not injective,
        yet $g\circ f$ is bijective.

    \end{enumerate}

    \vspace{1cm}

%%%%%%%%%%%%%%%%%%%%%%%%%%%%%%%%%%%%%%%%%%%%%%%%%%%%%%%%%%%%%%%%%%%%%%%%%%%%%%%%

    \noindent Problem 4.4.14  \vspace{0.3cm}

    \underline{proposition:} (Thm 4.4.4(ii)) Let $A$ and $B$ be
    non-empty sets, and $f:A\rightarrow B$ be a function, $f$ is
    injective iff $f\circ g=f\circ h$ implies $g=h$ for all function
    $g,h:Y\rightarrow A$.

    \vspace{0.2cm}

    \fbox{proof}

    \vspace{0.2cm}

    ($\Rightarrow$)  Suppose $f$ is injective, and that
    $g,h:Y\rightarrow A$ are arbitrary functions with the property that
    $f\circ g = f\circ h$. By theorem 4.4.3(ii), since $f$ is
    injective, $f$ has a left inverse $f_L^{-1}$. Thus, since $f\circ
    g= f\circ h$, it follows that $f_L^{-1}\circ(f\circ
    g)=f_L^{-1}\circ(f\circ h)$. By the associativity of composition,
    the definition of left inverse, and the identity law
    $f_L^{-1}\circ(f\circ g)=(f_L^{-1}\circ f)\circ g=1_A\circ g=g$.
    Similarly, $f_L^{-1}\circ(f\circ h )=(f_L^{-1}\circ f)\circ
    h=1_A\circ h=h$. Thus, since $f_L^{-1}\circ(f\circ
    g)=f_L^{-1}\circ(f\circ h)$, $g=h$. Since $g$ and $h$ were
    arbitrary functions, it follows that for all functions
    $g,h:Y\rightarrow A$, if $f$ has a left inverse and $f\circ g=
    f\circ h$, then $g=h$.

    ($\Leftarrow$ Proof by Contradiction)  Suppose $f$ is not injective,
    and that for all sets $Y$ and functions $g,h:Y\rightarrow A$, if
    $f\circ g=f\circ h$ then $g=h$. By definition of injective, if $f$
    is not injective, then there exist two distinct points $x_1$ and
    $x_2$ in $A$ such that $f(x_1)=f(x_2)$. Let $Y=A$ and let $h=1_A$.
    Define the function $g:A\rightarrow A$ by
    $$g(x)=\left\{\begin{array}{cc}
                      1_A(x) & \forall x\in
                      A-\{x_1,x_2\}\\ x_2 & x=x_1\\
                      x_1 & x=x_2
    \end{array}\right\}.$$ Thus $h\not= g$, yet $f\circ h=f=f\circ
    g$. This contradicts our assumption that if $f\circ g=f\circ h$
    then $g=h$. Thus, if for all sets $Y$ and functions
    $g,h:Y\rightarrow A$, $f\circ g=f\circ h$ implies $g=h$, then $f$ is
    injective.  $\blacksquare$

    \vspace{1cm}


%%%%%%%%%%%%%%%%%%%%%%%%%%%%%%%%%%%%%%%%%%%%%%%%%%%%%%%%%%%%%%%%%%%%%%%%%%



\end{document}