% Program: PortfolioCh1S2-3.ltx
% Author: Erin McNicholas
%
% Note: Comments made after a % sign are not read by the compiler.
%       are notational comments in the code.
% *********************************************************************

% *********************************************************************
%     Header Commands: These are commands that format the document and
%         define new command shortcuts.  You can use the \newcommand
%         function to define shortcuts for commonly used commands.
%         If you are just learning LaTeX, you should not need to 
%         modify this portion of the code
% *********************************************************************


\documentclass{article}
\usepackage[dvips]{graphicx}
\usepackage{a4wide}
\usepackage{amsmath}
\usepackage{euscript}
\usepackage{amssymb}
\usepackage{amsthm}
\usepackage{amsopn}

\theoremstyle{definition}
\newtheorem*{definition}{Definition}
\newtheorem{theorem}{Theorem}

\newcommand{\vv}{\ensuremath{\vec{v}}}
\newcommand{\vu}{\ensuremath{\vec{u}}}
\newcommand{\vw}{\ensuremath{\vec{w}}}
\newcommand{\vx}{\ensuremath{\vec{x}}}
\newcommand{\vy}{\ensuremath{\vec{y}}}
\newcommand{\vb}{\ensuremath{\vec{b}}}
\newcommand{\vo}{\ensuremath{\vec{0}}}
\newcommand{\va}{\ensuremath{\vec{a}}}
\newcommand{\ve}{\ensuremath{\vec{e}}}
%%%%%%%%%%%%%%%%%%%%%%%%%%%%%%%%%%%%%%%%%%%%%%%

% *********************************************************************
%     \begin{document} starts the portion of the code that will 
%         be translated into text.  This is the portion of the code
%         you will modify to insert your answers in
% *********************************************************************
\begin{document}
\begin{center}
\Large{Math 251W: Foundations of Advanced Mathematics}

\normalsize{Portfolio Assignment 1: \S 1.2-3}

\vspace{0.2cm}

\hfill {\bf Name:} Your Name Here

\vspace{0.25cm}
\hrule
\end{center}

\vspace{0.3cm}

\noindent Problem 1.2.6:  Let $E=$ ``The house is blue,'' let $F=$ ``The house is 30 years old,'' and $G=$ ``The house is ugly.''  Translate the following statements into symbols.

\begin{itemize}
\item[(1)] If the house is 30 years old, then it is ugly.
\item[(2)] If the house is blue, then it is ugly or it is 30 years old.
\item[(3)] If the house is blue then it is ugly, or it is 30 years old.
\item[(4)] The house is not ugly if and only if it is 30 years old.
\item[(5)] The house is 30 years old if it is blue, and it is not ugly if it is 30 years old.  
\item[(6)] For the house to be ugly, it is necessary and sufficient that it be ugly and 30 years old.
\end{itemize}
\vspace{0.25cm}

\fbox{Solution}
\begin{itemize}
\item[(1)] $F\rightarrow G$
\item[(2)] $E\rightarrow(G\vee F)$
\item[(3)]
\item[(4)]
\item[(5)]
\item[(6)]
\end{itemize}

\vspace{.5cm}

%*****************************************************************

\noindent Problem 1.2.15: Let $P$ be a statement, let $TA$ be a tautology, and let $CO$ be a contradiction.
\begin{itemize}
\item[(1)] Show that $P\vee TA$ is a tautology.
\item[(2)] Show that $P\wedge CO$ is a contradiction.
\end{itemize}

\vspace{0.25cm}
\fbox{Solution}
\begin{itemize}
\item[(1)] Since $TA$ is a tautology, by construction it is always
true. Similarly, since $CO$ is a contradiction, it is always false.
Thus, we can verify that for any statement $P$, $P\vee TA$ is a
tautology using the truth table
below.

\vspace{0.25cm}

$\begin{array}{c|c|ccc}P&TA&P&\vee&TA\\\hline
T & T & T & T & T\\
F & T & F & T & T\end{array}$

\vspace{0.25cm}

\item[(2)]
\end{itemize}

\vspace{.5cm}

%*****************************************************************

\noindent Problem 1.3.8: State the {\it inverse, converse,} and {\it contrapositive} of each of the following statements. \emph{Hint: First write the original statement in If... then... form}.
\begin{itemize}
\item[(1)] If it's Tuesday, it must be Belgium.
\item[(2)] I will go home if it is after midnight.
\item[(3)] Good fences make good neighbors.
\item[(4)] Lousy food is sufficient for a quick meal.
\end{itemize}

\vspace{0.25cm}
\fbox{Solution}
\begin{itemize}
\item[(1)]
\begin{itemize}
\item Inverse: If it's not Tuesday, it must not be Belgium.
\item Converse: If it's Belgium, it must be Tuesday.
\item Contrapositive:  If it's not Belgium, it must not be Tuesday.
\end{itemize}
\item[(2)]
\item[(3)]
\item[(4)] 
\end{itemize}

\vspace{.5cm}

%*****************************************************************

\noindent Problem 1.3.10: Negate each of the following statements
\begin{itemize}
\item[(1)] $e^5>0.$
\item[(2)] $3<5$ or $7\ge 8.$
\item[(3)] $\sin(\pi/2)<0$ and $\tan(0)\ge 0.$
\item[(4)] If $y=3$ then $y^2=7.$
\item[(5)] $w-3>0$ implies $w^2+9>6w.$
\item[(6)] $a-b=c$ iff $a=b+c$
\end{itemize}

\vspace{0.25cm}
\fbox{Solution}
\begin{itemize}
\item[(1)] $e^5\le 0.$
\end{itemize}

\vspace{.5cm}

%*****************************************************************

\noindent Problem 1.3.12: Simplify the following statements (making use of any equivalences of statements given so far in the text or exercises).
\begin{itemize}
\item[(1)] $\neg(P\rightarrow \neg Q).$
\item[(2)] $A\rightarrow (A\wedge B).$
\item[(3)] $(X\wedge Y)\rightarrow X.$
\item[(4)] $\neg(M\vee L)\wedge L.$
\item[(5)] $(P\rightarrow Q)\vee Q.$
\item[(6)] $\neg(X\rightarrow Y)\vee Y.$
\end{itemize}

\vspace{0.25cm}
\fbox{Solution}
\begin{itemize}
\item[(1)] $\neg(P\rightarrow \neg Q)\Leftrightarrow P\wedge Q$
\end{itemize}

\vspace{.5cm}

%*****************************************************************


\noindent Challenge Problem: 
For each statement, determine whether the second statement is the inverse, converse, or contrapostive of the first statement, or none of these.  If none of these, write what the inverse, converse, and contrapositive of the first statement would be.
\begin{itemize}
    \item[(1)] ``If I am cold, I  put on a jacket;'' and ``I am not cold if I do not wear a jacket."
    \item[(2)] ``A warm house is necessary for a warm heart"; and ``A cold heart is sufficient for a cold house."
    \item[(3)] ``Going to the beach is sufficient for me to have fun"; and ``Not going to the beach is insufficient for me to have fun"
\end{itemize}
\vspace{0.25cm}
\fbox{Solution}
\begin{itemize}
\item[(1)]
\item[(2)]
\item[(3)]
\end{itemize}
\end{document}
