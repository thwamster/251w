\documentclass{article}
\usepackage{xypic}
\usepackage{graphicx}
\usepackage{amsmath, amsthm, amssymb}
\usepackage{url}
\usepackage[usenames,dvipsnames]{color}
\setlength{\textheight}{22.5cm} \setlength{\textwidth}{16cm}
\setlength{\topmargin}{-1cm} \setlength{\oddsidemargin}{0cm}
\setlength{\evensidemargin}{0cm}

\newcommand{\R}{\mathbb{R}}
\newcommand{\Z}{\mathbb{Z}}
\newcommand{\C}{\mathbb{C}}
\newcommand{\N}{\mathbb{N}}
\newcommand{\Q}{\mathbb{Q}}

\newcommand{\tcG}{\textcolor{Green}}
\newcommand{\tcR}{\textcolor{Red}}
\newcommand{\tcB}{\textcolor{Blue}}
\newcommand{\tcP}{\textcolor{Purple}}

%%%%%%%%%%%%%%%%%%%%%%%%%%%%%%%%%%%%%%%%%%%%%%%


\begin{document}

\newpage
\hfill\small{Math 251W: Foundations of Advanced Mathematics, McNicholas}
\vspace{0.5cm}
\hrule
\vspace{0.5cm}
\large{\fbox{Axioms / Definitions / Theorems at our disposal}
\normalsize
\begin{itemize}
\item[] \fbox{Defn} Number Systems
	\begin{itemize}
	\item[]  Integers, $\Z:=\{\ldots, -3, -2, -1, 0, 1, 2, 3, \ldots\}$
		\begin{itemize}
		\item Addition and Multiplication are defined and \emph{closed} on the integers, i.e. if $a,b\in \Z$ then $a+b\in \Z$ and $ab\in\Z$.
		\item Multiplication distributes over addition, i.e. $\forall a,b,c\in\Z$ $a(b+c)=ab+ac.$
		\item Addition and multiplication are both associative, i.e. $a+(b+c)=(a+b)+c$ and $a(bc)=a(bc)$ forall $a,b,c\in\Z$
		\item Addition and multiplication are both commutative, i.e. $a+b=b+a$ and $ab=ba$ for all $a,b\in\Z$
		\item Every integer has an additive inverse, i.e. for all $a\in\Z$ there exists $-a\in\Z$ such that $a+(-a)=0$
		\item Not every element in the integers has a multiplicative inverse, i.e. given $a\in\Z$, there does NOT exist an $a^{-1}\in\Z$ such that $aa^{-1}=1$ unless $a=\pm 1.$
		\end{itemize}
	\item[]  Rational numbers, $\Q:=$\{set of numbers of the form $\frac{n}{m},$ where $m,n\in\Z,$ and $m\not=0,$ i.e. fractions\} 
		\begin{itemize}
		\item Addition and Multiplication are defined, associative, commutative, and closed on the rationals.
		\item Multiplication distributes over addition
		\item Every rational number has an additive inverse, i.e. for all $\frac{m}{n}\in\Q$ there exists $\frac{-m}{n}\in\Q$ such that $\frac{m}{n}+\frac{-m}{n}=0$
		\item Every non-zero element of the rationals has a multiplicative inverse, in particular if  ($\frac{m}{n}\in\Q$ and $\frac{m}{n}\not=0$, then $\frac{n}{m}\in\Q$ and $(\frac{m}{n})(\frac{n}{m})=1$).
		\end{itemize}
	\item[] Real numbers $\R:=$\{the set of rational and irrational numbers, i.e. everything on the number line\}
		\begin{itemize}	
		\item Addition and Multiplication are defined, associative, commutative, and closed on the reals.
		\item Multiplication distributes over addition
		\item Every real number has an additive inverse, and every \emph{non-zero} real number has a multiplicative inverse in the reals.
		\end{itemize}
	\end{itemize}
\end{itemize}
\bigskip

Definitions \& Theorems about Number Systems
\begin{itemize}
\item[] \fbox{Defn} Let $a$ and $b$ be numbers in some Number System $U$.  We say $a$ divides $b$ in $U$, denoted $a|b,$ iff there exists an element $q$ in $U$ such that $b=aq.$  If $a$ divides $b$, we say $a$ is a \emph{divisor} of $b.$
\item[] \textcolor{Red}{Check your understanding \begin{itemize}
    \item State the definition of divides in the context of the integers
    \item What are the divisors of $6$ in $\Z$?
    \item What are the divisors of $6$ in $\Q$?
    \item What are the divisors of $6$ in $\R$?
\end{itemize}}
\item[] \fbox{Thm}  There are no zero divisors in $\R,$ i.e. if $ab=0$ for some $a,b\in R$ then $a=0$ or $b=0.$ 
\item[] \textcolor{Red}{Check your understanding: Are there zero divisors in $\Q$ or $\Z$?}
\end{itemize}
\newpage

Definitions \& Theorems involving Integers $\Z$
\begin{itemize}
\item[] \fbox{Defn} Let $n$ be an integer.  We say $n$ is \emph{even} if there is some integer $k$ such that $n=2k.$  We say $n$ is \emph{odd} if there is some integer $j$ such that $n=2j+1.$
\item[] \fbox{Defn} Let $p$ be a positive integer greater than $1.$  We say $p$ is \emph{prime} iff the only positive integers which divide $p$ are $1$ and $p.$  A positive integer greater than $1$  is \emph{composite} if it is not prime.
\item[] \fbox{Defn} Integers $a$ and $b$ are said to be \emph{relatively prime} iff their only common factors are $\pm 1.$
\item[] \fbox{Thm} Every integer can be \emph{uniquely} expressed,  up to an ordering of the factors and multiplication by $\pm 1$, as a product of primes.
\item[] \fbox{Cor} Every integer greater than $1$ has a prime divisor.
\item[] \fbox{Cor to Div Alg} For all integers $a$ and non-zero integers $b$, there exist integers $q$ and $r$ such that $a=bq+r$ and $0\leq r< b$.
\end{itemize}

\vfill

Definitions \& Theorems about Real ($\R$) and Rational ($\Q$) Numbers
\begin{itemize}
\item[] \fbox{Defn} Let $x\in\R.$  We say $x$ is a \emph{rational number} if there exist integers $n$ and $m$ such that $m\not=0$ and $x=n/m.$ If $x$ is not rational, we say it is an \emph{irrational number.}

\item[] \fbox{Thm} Every rational number $r$ can be expressed in the form $\frac{p}{q}$ where $p$ and $q$ are relatively prime integers.
\end{itemize}
\vfill
\end{document}