\documentclass{article}
\usepackage[dvips]{graphicx}
\usepackage{a4wide}
\usepackage{amsmath}
\usepackage{euscript}
\usepackage{amssymb}
\usepackage{amsthm}
\usepackage{amsopn}

\theoremstyle{definition}
\newtheorem*{definition}{Definition}
\newtheorem{theorem}{Theorem}

\newcommand{\vv}{\ensuremath{\vec{v}}}
\newcommand{\vu}{\ensuremath{\vec{u}}}
\newcommand{\vw}{\ensuremath{\vec{w}}}
\newcommand{\vx}{\ensuremath{\vec{x}}}
\newcommand{\vy}{\ensuremath{\vec{y}}}
\newcommand{\vb}{\ensuremath{\vec{b}}}
\newcommand{\vo}{\ensuremath{\vec{0}}}
\newcommand{\va}{\ensuremath{\vec{a}}}
\newcommand{\ve}{\ensuremath{\vec{e}}}

\newcommand{\R}{\mathbb{R}}
\newcommand{\Z}{\mathbb{Z}}
\newcommand{\C}{\mathbb{C}}
\newcommand{\N}{\mathbb{N}}
\newcommand{\Q}{\mathbb{Q}}
\newcommand{\D}{\ensuremath{{\cal D}}}

%%%%%%%%%%%%%%%%%%%%%%%%%%%%%%%%%%%%%%%%%%%%%%%


\begin{document}
    \begin{center}
        \Large{Math 251W: Foundations of Advanced Mathematics}

        \normalsize Hints to Portfolio problems from sections 6.1, 6.2,
        $\&$ 6.3 \hrule \vspace{0.4cm}


        \vspace{0.75cm}
    \end{center}




    \noindent Problem 6.1.6: Prove the following \vspace{0.3cm}

    \underline{proposition:} Let $A$ and $B$ be finite sets such that $|A|=|B|$. Given a function $f:A\rightarrow B$, the following are equivalent:
    \begin{itemize}
        \item[i.] $f$ is bijective
        \item[ii.] $f$ is injective
        \item[iii.] $f$ is surjective
    \end{itemize}
    \vspace{0.2cm}

    \fbox{Hint:}
    \begin{itemize}
        \item Recall that for finite sets $X$ and $B$, if $X\subseteq B$ and $|X|=|B|$ then $X=B$. Thus if you show $|f_*(A)|=B$ you can conclude $f_*(A)=B.$
        \item Also, if you have already proved that $f$ being injective implies $f$ is surjective, then you might be able to apply that result to the left or right (you should figure out which is the right one to use) inverse of $f$ to get that surjectivity implies injectivity without having to prove injectivity directly.
    \end{itemize}

    \vspace{1cm}


%%%%%%%%%%%%%%%%%%%%%%%%%%%%%%%%%%%%%%%%%%%%%%%%%%%%%%%%%%%%%%%%%%%%%%%%%%

    \noindent Problem 6.1.13: Prove the following \vspace{0.3cm}



    \underline{proposition:} Given countable sets $A$ and $B$, $A\times
    B$ is countable.

    \vspace{.2cm}

    \fbox{Hint}
    \begin{itemize}
        \item Define the set $B_a=\{(a,x)|x\in B\}$ for all $a\in A$.
        \item It is not too hard to prove that the cardinality of $B_a=B$ for all $a\in A$
        \item It is also not too hard to prove that $A\times B=\bigcup_{a\in A}B_a$.
        \item From there, the proposition follows from our theorem about the union of countably many countable sets.
    \end{itemize}
    \vspace{1cm}


%%%%%%%%%%%%%%%%%%%%%%%%%%%%%%%%%%%%%%%%%%%%%%%%%%%%%%%%%%%%%%%%%%%%%%%%%%


%%%%%%%%%%%%%%%%%%%%%%%%%%%%%%%%%%%%%%%%%%%%%%%%%%%%%%%%%%%%%%%%%%%%%%%%%%

    \noindent Problem Irreducible Polynomials: Prove the following  \vspace{0.3cm}

    \underline{proposition:} Every reducible polynomial can be written as a product of irreducible polynomials.

    \vspace{.2cm}

    \fbox{Hint}
    \begin{itemize}
        \item Some useful bits of knowledge from H.S. Algebra:
        \begin{itemize}
            \item A \emph{reducible} polynomial is one which can be written as the product of two polynomials of smaller degree
            \item Degree 1 polynomials are all irreducible
            \item Degree 3 and above are all reducible
        \end{itemize}
        \item The proposition can be reworded to make the dependence on $\N$ explicit by saying: Every polynomial of degree $n$ can be written as a product of reducible polynomials for all $n\in\N$.
    \end{itemize}
    \vspace{1cm}

\end{document}