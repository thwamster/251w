\documentclass{article}
\usepackage[dvips]{graphicx}
\usepackage{a4wide}
\usepackage{amsmath}
\usepackage{euscript}
\usepackage{amssymb}
\usepackage{amsthm}
\usepackage{amsopn}

\theoremstyle{definition}
\newtheorem*{definition}{Definition}
\newtheorem{theorem}{Theorem}

\newcommand{\vv}{\ensuremath{\vec{v}}}
\newcommand{\vu}{\ensuremath{\vec{u}}}
\newcommand{\vw}{\ensuremath{\vec{w}}}
\newcommand{\vx}{\ensuremath{\vec{x}}}
\newcommand{\vy}{\ensuremath{\vec{y}}}
\newcommand{\vb}{\ensuremath{\vec{b}}}
\newcommand{\vo}{\ensuremath{\vec{0}}}
\newcommand{\va}{\ensuremath{\vec{a}}}
\newcommand{\ve}{\ensuremath{\vec{e}}}

\newcommand{\R}{\mathbb{R}}
\newcommand{\Z}{\mathbb{Z}}
\newcommand{\C}{\mathbb{C}}
\newcommand{\N}{\mathbb{N}}
\newcommand{\Q}{\mathbb{Q}}

%%%%%%%%%%%%%%%%%%%%%%%%%%%%%%%%%%%%%%%%%%%%%%%


\begin{document}
\begin{center}
\Large{Math 251W: Foundations of Advanced Mathematics}

\normalsize Solutions to some of the suggested problems from section 5.3\hrule
\vspace{0.4cm}


\end{center}


\noindent Problem 5.3.1  \vspace{0.3cm}

\begin{enumerate}
\item No, not transitive
\item No, not symmetric or reflexive
\item Equivalence relation
\item No, not transitive
\item Equivalence relation
\item Equivalence relation
\end{enumerate}

\vspace{1cm}

%%%%%%%%%%%%%%%%%%%%%%%%%%%%%%%%%%%%%%%%%%%%%%%%%%%%%%%%%%%%%%%%%%%%%%%%%%%%

\noindent Problem 5.3.2  \vspace{0.3cm}

\begin{enumerate}
\item $[0]=\{0\}, \qquad [3]=\{3, -3\}$
\item $[0]=\{n\pi|n\in\Z\}, \qquad [3]=\{b\R|\sin(b)=\sin(3)\}$
\item $[0]=\{0\}, \qquad [3]=\{\ldots,\frac{3}{4},\frac{3}{2},3,6,12,\ldots\}=\{2^n3|n\in\Z\}$
\end{enumerate}

\vspace{1cm}

%%%%%%%%%%%%%%%%%%%%%%%%%%%%%%%%%%%%%%%%%%%%%%%%%%%%%%%%%%%%%%%%%%%%%%%%

\noindent Problem 5.3.3  \vspace{0.3cm}

\begin{enumerate}
\item $[(0,0)]=\{(0,0)\}$, and $[(3,4)]$ equals the set of points on the circle of radius $5$ centered at the origin.
\item $[(0,0)]=\{(0,0)\}$, and $[(3,4)]$ equals the set of points on diamond centered at the origin with corners at the points $(7,0), (0,7), (-7,0),$ and $(0,-7)$
\item $[(0,0)]=\{(0,0)\}$, and $[(3,4)]$ equals the set of points on the square of sides of length 8 traced out by the line segments $y=4$ and $y=-4$ for $-4\le x\le 4$, and the segments $x=4$ and $x=-4$ for $-4\le y\le4$.
\end{enumerate}

\vspace{1cm}

%%%%%%%%%%%%%%%%%%%%%%%%%%%%%%%%%%%%%%%%%%%%%%%%%%%%%%%%%%%%%%%%%%%%%%%%%%%%%%


\noindent Problem 5.3.10  \vspace{0.3cm}

\begin{enumerate}
\item Partition of $[0,\infty)$
\item Not a partition, sets have non-empty intersection and the union includes more than the desired set
\item Partition
\item Not a partition, sets have non-empty intersection
\item Not a partition, sets have non-empty intersection
\item Partition
\end{enumerate}

\vspace{1cm}

%%%%%%%%%%%%%%%%%%%%%%%%%%%%%%%%%%%%%%%%%%%%%%%%%%%%%%%%%%%%%%%%%%%%%%%%%%%%%%

%%%%%%%%%%%%%%%%%%%%%%%%%%%%%%%%%%%%%%%%%%%%%%%%%%%%%%%%%%%%%%%%%%%%%%%%%%%%%

\noindent Problem 5.3.11  \vspace{0.3cm}

Outline in the back of the text

\vspace{1cm}

%%%%%%%%%%%%%%%%%%%%%%%%%%%%%%%%%%%%%%%%%%%%%%%%%%%%%%%%%%%%%%%%%%%%%%%%%%%%


\noindent Problem 5.3.12  \vspace{0.3cm}

\begin{enumerate}
\item Partition sets are sets of siblings and half-siblings related through the maternal line
\item $\mathcal{D}=\{\mbox{positive numbers}, \mbox{negative numbers}\}$
\item $\mathcal{D}$ equals the set of all circles centered at the point $(1,0)$ in the Cartesian plane
\end{enumerate}


\vspace{1cm}



%%%%%%%%%%%%%%%%%%%%%%%%%%%%%%%%%%%%%%%%%%%%%%%%%%%%%%%%%%%%%%%%%%%%%%%%%%


\noindent Problem 5.3.15  \vspace{0.3cm}

\begin{enumerate}
\item Yes, for an arbitrary set we have seen relations which are not equivalence relations, and sets of sets which have non-empty intersection
\item $\mathcal{R}(A)=\mathcal{P}(A\times A)$ and $\mathcal{S}_A=\mathcal{P}(\mathcal{P}(A))$
\item $\mathcal{R}(A)=\{\emptyset,\{(1,1)\},\{(2,2)\},\{(1,2)\}, \{(2,1)\},\{(1,1),(1,2)\},\{(1,1),(2,1)\},\{(1,1),(2,2)\},\{(2,2),(1,2)\},\newline\{(2,2),(2,1)\},\{(1,1),(1,2),(2,1)\},\{(1,1),(1,2),(2,2)\},\{(1,1),(2,1),(2,2)\},\{(1,2),(2,1),(2,2)\},\newline\{(1,1),(1,2),(2,1),(2,2)\}\}$ \newline and $\mathcal{S}_A=\{\{\emptyset\},\{\{1\}\},\{\{2\}\},\{\{1,2\}\},\{\emptyset,\{1\}\},\{\emptyset,\{2\}\},\{\emptyset,\{1,2\}\},\{\{1\},\{2\}\},\{\{1\},\{1,2\}\},\{\{2\},\{1,2\}\},\newline\{\emptyset,\{1\},\{2\}\},\{\emptyset,\{1\},\{1,2\}\},\{\emptyset,\{2\},\{1,2\}\},\{\{1\},\{2\},\{1,2\}\},\{\emptyset,\{1\},\{2\},\{1,2\}\}\}$
\item $|\mathcal{R}(A)|=|\mathcal{S}(A)|=2^{2^n}-1$
\end{enumerate}

\vspace{1cm}


%%%%%%%%%%%%%%%%%%%%%%%%%%%%%%%%%%%%%%%%%%%%%%%%%%%%%%%%%%%%%%%%%%%%%%%%%%


%%%%%%%%%%%%%%%%%%%%%%%%%%%%%%%%%%%%%%%%%%%%%%%%%%%%%%%%%%%%%%%%%%%%%%%%%%



\end{document}