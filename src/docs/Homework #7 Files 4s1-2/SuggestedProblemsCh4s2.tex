\documentclass{article}
\usepackage[dvips]{graphicx}
\usepackage{a4wide}
\usepackage{amsmath}
\usepackage{euscript}
\usepackage{amssymb}
\usepackage{amsthm}
\usepackage{amsopn}

\theoremstyle{definition}
\newtheorem*{definition}{Definition}
\newtheorem{theorem}{Theorem}

\newcommand{\vv}{\ensuremath{\vec{v}}}
\newcommand{\vu}{\ensuremath{\vec{u}}}
\newcommand{\vw}{\ensuremath{\vec{w}}}
\newcommand{\vx}{\ensuremath{\vec{x}}}
\newcommand{\vy}{\ensuremath{\vec{y}}}
\newcommand{\vb}{\ensuremath{\vec{b}}}
\newcommand{\vo}{\ensuremath{\vec{0}}}
\newcommand{\va}{\ensuremath{\vec{a}}}
\newcommand{\ve}{\ensuremath{\vec{e}}}

\newcommand{\R}{\mathbb{R}}
\newcommand{\Z}{\mathbb{Z}}
\newcommand{\C}{\mathbb{C}}
\newcommand{\N}{\mathbb{N}}
\newcommand{\Q}{\mathbb{Q}}

%%%%%%%%%%%%%%%%%%%%%%%%%%%%%%%%%%%%%%%%%%%%%%%


\begin{document}
    \begin{center}
        \Large{Math 251W: Foundations of Advanced Mathematics}

        \normalsize Solutions to problems from section 4.2 \hline
        \vspace{0.4cm}

    \end{center}


    \noindent Problem 4.2.3  \vspace{0.3cm}

    \begin{enumerate}
        \item[2] $$f_*([-5,2])=[0,16],$$
        $$f^*([-5,2])=[-1-\sqrt{2},-1+\sqrt{2}],$$ $$f_*(f^*([-5,2]))=[0,2],$$
        $$f^*(f_*([-5,2]))=[-5,3]$$
        \item[4] $$f_*(T)=\{0,1,2,6\},$$ $$f^*(T)=[0,3)\cup[6,7),$$
        $$f_*(f^*(T))=\{0,1,2,6\},$$ $$f^*(f_*(T))=[0,3)\cup[6,7)$$
    \end{enumerate}

    \vspace{1cm}

%%%%%%%%%%%%%%%%%%%%%%%%%%%%%%%%%%%%%%%%%%%%%%%%%%%%%%%%%%%%%%%%%%%%%%%%%%%%

    \noindent Problem 4.2.4  \vspace{0.3cm}

    \begin{enumerate}
        \item $g^*(\{3\})$ is the set of all ordered pairs $(x,y)$ such that
        $xy=3$. Thus, $g^*(\{3\})$ would be the set of points on the graph
        of $y=\frac{3}{x}$
        \item $g^*([-1,1])$ would be the portion of the cartesian plane
        between the $x$-axis and the graph of $y=\frac{1}{x}$ and the
        portion of the cartesian plane between the $x$-axis and the graph of
        $y=\frac{-1}{x}$, including the $x$-axis and the points on the graph
        of $y=\frac{1}{x}$ and $y=\frac{-1}{x}$.
    \end{enumerate}

    \vspace{1cm}

%%%%%%%%%%%%%%%%%%%%%%%%%%%%%%%%%%%%%%%%%%%%%%%%%%%%%%%%%%%%%%%%%%%%%%%%

    \noindent Problem 4.2.5  \vspace{0.3cm}

    \begin{enumerate}
        \item[3] Not necessarily. Counterexample: Let $X=\{a,b,c,d\},
        Y=\{1,2,3,4\}$, and $P=\{(a,2), (a,3), (b,1)\}$. Clearly
        $P\subseteq X\times Y$, $\pi_1(P)=\{a,b\}$, and
        $\pi_2(P)=\{1,2,3\}$. Thus,
        $\pi_1(P)\times\pi_2(P)=\{(a,1),(a,2),(a,3),(b,1),(b,2),(b,3)\}\not=P$.
    \end{enumerate}

    \vspace{1cm}

%%%%%%%%%%%%%%%%%%%%%%%%%%%%%%%%%%%%%%%%%%%%%%%%%%%%%%%%%%%%%%%%%%%%%%%%%%%%%%


%%%%%%%%%%%%%%%%%%%%%%%%%%%%%%%%%%%%%%%%%%%%%%%%%%%%%%%%%%%%%%%%%%%%%%%%%%%%%

    \noindent Problem 4.2.7  \vspace{0.3cm}

    $f_*(f^*(\bigcup_{j\in J}V_j))$ is not necessarily equal to
    $\bigcup_{j\in J}V_j$. See problem 4.2.3 for examples of when the
    image does not cancel out the inverse image.

    \vspace{1cm}

%%%%%%%%%%%%%%%%%%%%%%%%%%%%%%%%%%%%%%%%%%%%%%%%%%%%%%%%%%%%%%%%%%%%%%%%%%%%


    \noindent Problem 4.2.8  \vspace{0.3cm}

    \begin{enumerate}
        \item  Take the function $f:\R\rightarrow\R$ defined by $f(x)=x^2$
        for all $x\in\R$. Thus, $f_*([0,2])=[0,4]$, but $f^*([0,4])=[-2,2]$
        not $[0,2]$.
        \item Again, take $f:\R\rightarrow\R$ defined by $f(x)=x^2$.
        $f^*([-10,4])=[-2,2]$, but $f_*([-2,2])=[0,4]$.
    \end{enumerate}

    \vspace{1cm}

%%%%%%%%%%%%%%%%%%%%%%%%%%%%%%%%%%%%%%%%%%%%%%%%%%%%%%%%%%%%%%%%%%%%%%%%%%%%%%%%


%%%%%%%%%%%%%%%%%%%%%%%%%%%%%%%%%%%%%%%%%%%%%%%%%%%%%%%%%%%%%%%%%%%%%%%%%%



\end{document}