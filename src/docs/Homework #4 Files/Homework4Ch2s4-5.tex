% Program: PortfolioCh2S4-5.ltx
% Author: Erin McNicholas
%
% Description: Portfolio Assignment 4: Sections 2.4 & 2.5
%
% Note: Comments made after a % sign are not read by the compiler.
%       are notational comments in the code.
% *********************************************************************

% *********************************************************************
%     Header Commands: These are commands that format the document and
%         define new command shortcuts.  You can use the \newcommand
%         function to define shortcuts for commonly used commands.
%         If you are just learning LaTeX, you should not need to
%         modify this portion of the code
% *********************************************************************


\documentclass{article}
\usepackage[dvips]{graphicx}
\usepackage{a4wide}
\usepackage{amsmath}
\usepackage{euscript}
\usepackage{amssymb}
\usepackage{amsthm}
\usepackage{amsopn}

\theoremstyle{definition}
\newtheorem*{definition}{Definition}
\newtheorem{theorem}{Theorem}

\newcommand{\vv}{\ensuremath{\vec{v}}}
\newcommand{\vu}{\ensuremath{\vec{u}}}
\newcommand{\vw}{\ensuremath{\vec{w}}}
\newcommand{\vx}{\ensuremath{\vec{x}}}
\newcommand{\vy}{\ensuremath{\vec{y}}}
\newcommand{\vb}{\ensuremath{\vec{b}}}
\newcommand{\vo}{\ensuremath{\vec{0}}}
\newcommand{\va}{\ensuremath{\vec{a}}}
\newcommand{\ve}{\ensuremath{\vec{e}}}

\newcommand{\R}{\mathbb{R}}
\newcommand{\Z}{\mathbb{Z}}
\newcommand{\C}{\mathbb{C}}
\newcommand{\N}{\mathbb{N}}
\newcommand{\Q}{\mathbb{Q}}

%%%%%%%%%%%%%%%%%%%%%%%%%%%%%%%%%%%%%%%%%%%%%%%

% *********************************************************************
%     \begin{document} starts the portion of the code that will
%         be translated into text.  This is the portion of the code
%         you will modify to insert your answers in
% *********************************************************************
\begin{document}
\begin{center}
\Large{Math 251W: Foundations of Advanced Mathematics}

\normalsize{Portfolio Assignment 4: \S 2.4 \& 2.5}

\vspace{0.2cm}

\hfill {\bf Name:} Your name here

\vspace{0.25cm}
\hrule
\end{center}

\vspace{0.3cm}
%%%%%%%%%%%%%%%%%%%%%%%%%%%%%%%%%%%%%%%%%%%%%%%%%%%%%%%%%%%%%%%%%%%%%%%%%%%%%%

\noindent Problem 2.4.3  \vspace{0.3cm}

For the following proposition, give a list of conditional statements you would prove to prove the entire stated equivalence.  Then prove \textbf{one} of those conditional statements.

\medskip

\underline{proposition:} Given $a$ and $b$ are positive integers,
the following are equivalent:
\begin{itemize}
\item[i] $a$ and $b$ are relatively prime.
\item[ii] $a+b$ and $b$ are relatively prime.
\item[iii] $a$ and $a+b$ are relatively prime.
\end{itemize}

\vspace{0.4cm}

\fbox{proof}()

\vspace{0.2cm}



\vspace{1cm}

%%%%%%%%%%%%%%%%%%%%%%%%%%%%%%%%%%%%%%%%%%%%%%%%%%%%%%%%%%%%%%%%%%%%%%%%%%%%%


%%%%%%%%%%%%%%%%%%%%%%%%%%%%%%%%%%%%%%%%%%%%%%%%%%%%%%%%%%%%%%%%%%%%%%%%%%%%

\noindent Problem 2.4.6  \vspace{0.3cm} Fill in, and then prove, the following proposition.



\medskip

\underline{proposition} The only triple primes (i.e. primes of the form $p, p+2,$ and
$p+4$ for some integer $p$) are ???

\vspace{0.4cm}

\fbox{proof} ()



\vspace{1cm}

%%%%%%%%%%%%%%%%%%%%%%%%%%%%%%%%%%%%%%%%%%%%%%%%%%%%%%%%%%%%%%%%%%%%%%%%%%%%%

%%%%%%%%%%%%%%%%%%%%%%%%%%%%%%%%%%%%%%%%%%%%%%%%%%%%%%%%%%%%%%%%%%%%%%%%%%%%

\noindent Problem 2.4.8  \vspace{0.3cm} Prove the following

\underline{proposition:} If $n$ is an odd integer, then there exists
an integer $k$ such that $n^2=8k+1$.

\vspace{0.4cm}

\fbox{proof} ()



\vspace{1cm}

%%%%%%%%%%%%%%%%%%%%%%%%%%%%%%%%%%%%%%%%%%%%%%%%%%%%%%%%%%%%%%%%%%%

\noindent Problem 2.5.8  Prove or give a counterexample \vspace{0.3cm} 

\underline{proposition:}For each real numbner $p$, there exist real numbers $q$ and $r$ such athat $q\sin(r/5)=p.$

\vspace{0.3cm}


\fbox{proof/counterproof} 

\vspace{0.2cm}



\vspace{1cm}

%%%%%%%%%%%%%%%%%%%%%%%%%%%%%%%%%%%%%%%%%%%%%%%%%%%%%%%%%%%%%%%%%%%%%%%


%%%%%%%%%%%%%%%%%%%%%%%%%%%%%%%%%%%%%%%%%%%%%%%%%%%%%%%%%%%%%%%%%%%%%%%%%%

\noindent Problem 2.5.9   Prove or give a counterexample  \vspace{0.3cm}

\underline{proposition:}For each integer $x$, and for each integer $y$, there exists an integer $z$ such that $$z^2+2xz-y^2=0.$$

\vspace{0.4cm}

\fbox{proof/counterproof}




\end{document}