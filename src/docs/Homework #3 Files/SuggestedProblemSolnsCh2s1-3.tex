\documentclass{article}
\usepackage[dvips]{graphicx}
\usepackage{a4wide}
\usepackage{amsmath}
\usepackage{euscript}
\usepackage{amssymb}
\usepackage{amsthm}
\usepackage{amsopn}

\theoremstyle{definition}
\newtheorem*{definition}{Definition}
\newtheorem{theorem}{Theorem}

\newcommand{\vv}{\ensuremath{\vec{v}}}
\newcommand{\vu}{\ensuremath{\vec{u}}}
\newcommand{\vw}{\ensuremath{\vec{w}}}
\newcommand{\vx}{\ensuremath{\vec{x}}}
\newcommand{\vy}{\ensuremath{\vec{y}}}
\newcommand{\vb}{\ensuremath{\vec{b}}}
\newcommand{\vo}{\ensuremath{\vec{0}}}
\newcommand{\va}{\ensuremath{\vec{a}}}
\newcommand{\ve}{\ensuremath{\vec{e}}}

%%%%%%%%%%%%%%%%%%%%%%%%%%%%%%%%%%%%%%%%%%%%%%%


\begin{document}
    \begin{center}
        \Large{Math 251W: Foundations of Advanced Mathematics}

        \normalsize Solutions to problems from sections 2.1, 2.2, $\&$ 2.3
        \hrule \vspace{0.75cm}
    \end{center}


    \noindent Problem 2.1.1  \vspace{0.3cm}

    Reformulating the given statements into if-then form we get
    something like:

    \begin{enumerate}
        \item  If the radius of a circle is $r$, then the area of the circle
        is $\pi r^2$.
        \item  If $l$ is a line and $P$ is a point not on $l$, then there
        exists exactly one line $m$ containing $P$ and parallel to $l$.
        \item If $\triangle ABC$ is a triangle with sides $a$, $b$, and $c$,
        where side $a$ is opposite angle $A$, side $b$ is opposite angle
        $B$, and side $c$ is opposite angle $C$, then
        $$\frac{a}{\sin(A)}=\frac{b}{\sin(B)}=\frac{c}{\sin(C)}$$
        \item[5] If $f(x)$ is a continuous function on the closed interval
        $[a,b]$ and $F(x)$ is any function for which $F'(x)=f(x)$, then
        $\int_a^b f(x)dx=F(b)-F(a)$.
    \end{enumerate}

    \vspace{1cm}

%%%%%%%%%%%%%%%%%%%%%%%%%%%%%%%%%%%%%%%%%%%%%%%%%%%%%%%%%%%%%%%%%%%%%%%%%%%%

    \noindent Problem 2.2.1  \vspace{0.3cm}

    \begin{enumerate}
        \item[2] \underline{proposition:} If $x$ is a globular integer, then
        $x$ is even.

        \vspace{.2cm}

        Outline of the proof: Assume $x$ is a globular integer. Show that
        this somehow leads to $x=2n$ for some integer $n$, thus proving $x$
        is even by the definition of an even number.

        \item[4] \underline{proposition:} If an integer is odd, then it is
        tactile and filigreed.

        \vspace{.2cm}

        Outline of the proof: Assume $x$ is an odd integer, i.e. $x=2n+1$
        for some integer $n$. Show this leads to $x$ being both tactile and
        filigreed.
    \end{enumerate}

    \vspace{1cm}

%%%%%%%%%%%%%%%%%%%%%%%%%%%%%%%%%%%%%%%%%%%%%%%%%%%%%%%%%%%%%%%%%%%%%%%%

    \noindent Problem 2.2.4  \vspace{0.3cm}


    \underline{proposition:} If $n$ is an even integer, then $n^2$ is
    even. If $n$ is odd, then $n^2$ is odd.

    \vspace{0.4cm}

    \fbox{proof} (Direct Proof)

    \vspace{0.2cm}

    Let $n$ be an even integer. Then by definition, there exists some
    integer $x$ such that $n=2x$. Thus, $$n^2=(2x)^2=(2x)(2x)=2(2x^2)$$
    by the associative and commutative properties of integer
    multiplication. Furthermore, since the integers are closed under
    multiplication, $2x^2$ is equal to some integer $s$. Thus,
    $n^2=2s$, which by definition of an even number, implies $n^2$ is
    even.

    \vspace{0.2cm}

    Let $n$ be an odd integer. By definition of an odd integer, there
    exists some integer $k$ such that $n=2k+1$. Thus,
    $$n^2=(2k+1)^2=(2k+1)(2k+1)=4k^2+4k+1$$ by the associative,
    commutative, and distributive properties of the integers.
    Furthermore, by the distributive property, $4k^2+4k+1=2(2k^2+2k)+1$.
    Since the integers are closed under multiplication and addition,
    $2k^2+2k=t$ for some integer $t$. Thus, $n^2=2t+1$, which by
    definition of an odd integer, implies $n^2$ is odd. $\blacksquare$

    \vspace{1cm}

%%%%%%%%%%%%%%%%%%%%%%%%%%%%%%%%%%%%%%%%%%%%%%%%%%%%%%%%%%%%%%%%%%%%%%%%%%%%%%


%%%%%%%%%%%%%%%%%%%%%%%%%%%%%%%%%%%%%%%%%%%%%%%%%%%%%%%%%%%%%%%%%%%%%%%%%%%%%

    \noindent Problem 2.2.7  \vspace{0.3cm}

    \underline{proposition:} Let $a, b, c,$ and $d$ be integers. If
    $a|b$ and $c|d$ then $ac|bd$.

    \vspace{0.4cm}

    \fbox{proof} (Direct Proof)

    \vspace{0.2cm}

    Suppose $a, b, c,$ and $d$ are integers and that $a|b$ and $c|d$. By
    definition of divisibility, there exist integers $x$ and $y$ such
    that $ax=b$ and $cy=d$. Thus, by the associative and commutative
    properties of integer multiplication, $bd=(ax)(cy)=ac(xy)$. Since
    the integers are closed under multiplication, $xy=s$ for some
    integer $s$. Thus, $bd=acs$, which by definition implies $ac|bd$.
    $\blacksquare$

    \vspace{1cm}

%%%%%%%%%%%%%%%%%%%%%%%%%%%%%%%%%%%%%%%%%%%%%%%%%%%%%%%%%%%%%%%%%%%%%%%%%%%%


%%%%%%%%%%%%%%%%%%%%%%%%%%%%%%%%%%%%%%%%%%%%%%%%%%%%%%%%%%%%%%%%%%%%%%%%%%%%%

    \noindent Problem 2.3.1  \vspace{0.3cm}

    \begin{enumerate}
        \item[2] \underline{proposition:} If $x$ is a globular integer, then
        $x$ is even.

        \vspace{.2cm}

        Outline of the proof by contrapositive: Assume $x$ is odd, i.e.
        $x=2n+1$. Show this implies $x$ is not a globular integer.

        Outline of the proof by contradiction: Assume $x$ is a globular
        integer and $x$ is odd. Show that this leads to a contradiction.

        \item[4] \underline{proposition:} If an integer is odd, then it is
        tactile and filigreed.

        \vspace{.2cm}

        Outline of the proof by contrapositive: Assume $x$ is not tactile
        \underline{or} not filigreed. Show this implies $x$ is even, i.e.
        $x=2n$ for some integer $n$.

        Outline of the proof by contradiction: Assume $x$ is odd, not
        tactile, and not filigreed. Show this leads to a contradiction.
    \end{enumerate}

    \vspace{1cm}

%%%%%%%%%%%%%%%%%%%%%%%%%%%%%%%%%%%%%%%%%%%%%%%%%%%%%%%%%%%%%%%%%%%%%

    \noindent Problem 2.3.2  \vspace{0.3cm}

    \underline{proposition:} If $n$ is an integer and $n^2$ is even,
    then $n$ is even.

    \vspace{0.3cm}

    \underline{SCRATH}

    By the equivalence of the contrapositive, $n^2$ is even implies $n$
    is even iff $n$ being odd implies $n^2$ is odd.

    \vspace{0.4cm}

    \fbox{proof} (Contrapositive)

    \vspace{0.2cm} Suppose $n$ is an odd integer. By definition, there
    exists an integer $m$ such that $n=2m+1$. Thus, by the associative,
    commutative, and distributive properties,
    $n^2=(2m+1)^2=4m^2+4m+1=2(2m^2+2m)+1$. Furthermore, since the
    integers are closed under addition and multiplication, $2m^2+2m=s$
    for some integer $s$. Thus, $n^2=2s+1$. By definition of odd, this
    implies $n^2$ is odd. Thus, by the contrapositive, if $n^2$ is
    even, $n$ must be even. $\blacksquare$

    \vspace{1cm}
    \newpage

%%%%%%%%%%%%%%%%%%%%%%%%%%%%%%%%%%%%%%%%%%%%%%%%%%%%%%%%%%%%%%%%%%%%%%%%%%%%%

    \noindent Problem 2.3.4  \vspace{0.3cm}

    \underline{proposition:} If $y$ is a nonzero rational number and $x$
    is an irrational number, then $xy$ is irrational.

    \vspace{0.4cm}

    \fbox{proof} (Contradiction)

    \vspace{0.2cm} Suppose $y$ is a nonzero rational number, $x$ is an
    irrational number, and $xy$ is a rational number. We will show this
    leads to a contradiction. By definition of rational, since $xy$ and
    $y$ are rational numbers, there exist integers $m, n, l$ and $d$
    such that $xy=\frac{m}{n}$ and $y=\frac{l}{d}$. Thus,
    $$xy=x\frac{l}{d}=\frac{m}{n}.$$  Since $y$ is nonzero, $l\not=0$. Thus multiplication by $\frac{d}{l}$ is well
    defined. Multiplying both sides of $x\frac{l}{d}=\frac{m}{n}$ by
    $\frac{d}{l}$, we find $$x=\frac{m}{n}\frac{d}{l}=\frac{s}{t},$$
    where $s=md$ and $t=nl$. Since the integers are closed under
    multiplication, $s$ and $t$ are both integers. This implies $x$ is
    a rational number, which contradicts our assumption that $x$ is
    irrational. Thus, if $y$ is a nonzero rational number and $x$ is an
    irrational number, their product $xy$ is irrational. $\blacksquare$

    \vspace{1cm}

%%%%%%%%%%%%%%%%%%%%%%%%%%%%%%%%%%%%%%%%%%%%%%%%%%%%%%%%%%%%%%%%%%%%%%%%%%%%


%%%%%%%%%%%%%%%%%%%%%%%%%%%%%%%%%%%%%%%%%%%%%%%%%%%%%%%%%%%%%%%%%%%


%%%%%%%%%%%%%%%%%%%%%%%%%%%%%%%%%%%%%%%%%%%%%%%%%%%%%%%%%%%%%%%%%%%%%%%

    \noindent Problem 2.3.7  \vspace{0.3cm}

    \underline{proposition:} Let $q\ge 2$ be a positive integer. If for
    all integers $a$ and $b$, whenever $q|ab$, $q|a$ or $q|b$, then
    $\sqrt{q}$ is irrational.

    \vspace{0.4cm}

    \fbox{proof} (Contradiction)

    \vspace{0.2cm} Let $q$ be a positive integer greater than or equal
    to 2. Suppose that $\sqrt{q}$ is a rational number, and that for all
    integers $a$ and $b$, whenever $q|ab$, $q|a$ or $q|b$. We will show
    this leads to a contradiction. By definition of a rational number,
    $\sqrt{q}$ can be expressed as $\frac{n}{m}$ where $n$ and $m$ are
    integers, and $m\not=0$. Since $q\not=1$, $m$ and $n$ can be chosen
    in such a way that they do not share any common factors. Given
    $\sqrt{q}=\frac{n}{m}$ it follows that $q=\frac{n^2}{m^2}$.
    Multiplying both sides of this equation by $m^2$, we find
    $n^2=qm^2$. Since the integers are closed under multiplication,
    $m^2=s$ for some integer $s$. Thus, $n^2=qs$, which by definition,
    implies $q|n^2$. By our assumption, since $q|n^2$, $q$ must divide
    $n$. Thus there exists some integer $k$ such that $qk=n$.
    Substituting $qk$ in for $n$ in the equation $n^2=qm^2$, and
    applying the associative and commutative properties of integer
    multiplication, we find $q^2k^2=qm^2$. By the cancelation law,
    $qk^2=m^2.$  Again, by the closure of the integers under
    multiplication, there exists some integer $t=k^2$. Thus $qt=m^2$,
    which by definition, implies $q|m^2$. By our assumptions, $q$ must
    also divide $m$. Thus, $q$ is a common factor of $m$ and $n$, which
    contradicts the fact that $m$ and $n$ were chosen to have no common
    factors. Therefore, if for all integers $a$ and $b$, when $q|ab$,
    $q|a$ or $q|b$, then $\sqrt{q}$ must be irrational. $\blacksquare$

    \vspace{1cm}

%%%%%%%%%%%%%%%%%%%%%%%%%%%%%%%%%%%%%%%%%%%%%%%%%%%%%%%%%%%%%%%%%%%%%%%%%%


\end{document}