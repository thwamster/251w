% Program: PortfolioCh2S1-3.ltx
% Author: Brendan Butler, Teddy Wachtler, Grace Yanzito
% ******************************************************************************

\documentclass{article}
\usepackage[dvips]{graphicx}
\usepackage{a4wide}
\usepackage{amsmath}
\usepackage{euscript}
\usepackage{amssymb}
\usepackage{amsthm}
\usepackage{amsopn}

\theoremstyle{definition}
\newtheorem*{definition}{Definition}
\newtheorem{theorem}{Theorem}

\newcommand{\vv}{\ensuremath{\vec{v}}}
\newcommand{\vu}{\ensuremath{\vec{u}}}
\newcommand{\vw}{\ensuremath{\vec{w}}}
\newcommand{\vx}{\ensuremath{\vec{x}}}
\newcommand{\vy}{\ensuremath{\vec{y}}}
\newcommand{\vb}{\ensuremath{\vec{b}}}
\newcommand{\vo}{\ensuremath{\vec{0}}}
\newcommand{\va}{\ensuremath{\vec{a}}}
\newcommand{\ve}{\ensuremath{\vec{e}}}

\newcommand{\R}{\mathbb{R}}
\newcommand{\Z}{\mathbb{Z}}
\newcommand{\C}{\mathbb{C}}
\newcommand{\N}{\mathbb{N}}
\newcommand{\Q}{\mathbb{Q}}

%%%%%%%%%%%%%%%%%%%%%%%%%%%%%%%%%%%%%%%%%%%%%%%%%%%%%%%%%%%%%%%%%%%%%%%%%%%%%%%

\begin{document}
    \begin{center}
        \Large{Math 251W: Foundations of Advanced Mathematics}

        \normalsize{Portfolio Assignment 3: \S 2.1-3}

        \vspace{0.2cm}

        \hfill {\bf Name:} Brendan Butler, Teddy Wachtler, Grace Yanzito

        \vspace{0.25cm}
        \hrule
    \end{center}

    \vspace{0.3cm}

%%%%%%%%%%%%%%%%%%%%%%%%%%%%%%%%%%%%%%%%%%%%%%%%%%%%%%%%%%%%%%%%%%%%%%%%%%%%%%%

    \noindent Problem 2.2.6 [DISCUSS]
    \vspace{0.3cm}
    \underline{proposition:}

    Let $a, b, c, m,$ and $n$ be integers.
    If $a|b$ and $a|c$ then $a|(bm+cn)$.

    \vspace{0.3 cm}
    \underline{scaffold:} (Direct)

    \[\begin{array}{llr}
          a, b, c, m, n \in \Z \\
          a | b \\
          a | c \\
          \hline
          q_0, q_1 \in \Z \\
          b = aq_0 \mbox{, by properties of integer divisibility} \\
          c = aq_1 \\
          bm + cn = (aq_0)m + (aq_1)n \mbox{, by substitution} \\
          bm + cn = a(q_0 m + q_1 n) \\
          q_2 \in \Z \\
          bm + cn = aq_2 \mbox{, since addition is closed} \\
          a | (bm + cn) \mbox{, $\blacksquare$}
    \end{array}\]

    \vspace{0.4cm}
    \fbox{proof} (Direct Proof)
    \vspace{0.2cm}

    $\blacksquare$

    \vspace{1cm}

%%%%%%%%%%%%%%%%%%%%%%%%%%%%%%%%%%%%%%%%%%%%%%%%%%%%%%%%%%%%%%%%%%%%%%%%%%%%%%%

    \noindent Problem 2.2.8 [DISCUSS]
    \vspace{0.3cm}
    \underline{proposition:}

    If $a$ and $b$ are integers and $a|b$, then $a^n|b^n$.

    \vspace{0.3 cm}
    \underline{scaffold:} (Direct)

    \[\begin{array}{llr}
          a, b \in \Z \\
          a | b \\
          \hline
          q_0 \in \Z \\
          b = aq_0 \mbox{, by properties of integer divisibility} \\
          n \in \Z \\
          (b)^n = (aq_0)^n \\
          (b)^n = (a)^n (q_0)^n \mbox{, by properties of exponents} \\
          q_1 \in \Z \\
          b^n = a^n q_1 \\
          a^n | b^n \mbox{, by properties of integer divisibility, $\blacksquare$}
    \end{array}\]

    \vspace{0.4cm}
    \fbox{proof} ()
    \vspace{0.2cm}

    $\blacksquare$

    \vspace{1cm}

%%%%%%%%%%%%%%%%%%%%%%%%%%%%%%%%%%%%%%%%%%%%%%%%%%%%%%%%%%%%%%%%%%%%%%%%%%%%%%%

    \noindent Problem 2.3.5 [DISCUSS]
    \vspace{0.3cm}
    \underline{proposition:}

    Let $a, b,$ and $c$ be integers.
    If there exists an integer $d$ such that $d|a$ and $d|b$ but $d\not|c$, then $ax+by=c$ has no integer solutions for $x$ and $y$.

    \vspace{0.3 cm}
    \underline{scaffold:} (Contradiction)

    \[\begin{array}{llr}
          a, b, c, d, x, y \in \Z \\
          d | a \\
          d | b \\
          d \not| c \\
          ax + by = c \mbox{, the opposite of the conclusion} \\
          \hline
          q_0, q_1 \in \Z \\
          a = dq_0 \mbox{, by properties of integer divisibility} \\
          b = dq_1 \\
          (dq_0)x + (dq_1)y = c \\
          d(q_0 x + q_1 y) = c \\
          q_2 \in \Z \\
          dq_2 = c\mbox{, since addition is closed} \\
          d | c \mbox{, contradicts with $d \not | c$, $\blacksquare$}
    \end{array}\]

    \vspace{0.4cm}
    \fbox{proof} (Contradiction) \\
    \vspace{0.2cm} \\

    Let a, b, and c be arbitrary integers.
    Let d be a particular integer so that $d|a$, $d|a$, and $d\not | c$.
    Let $x$ and $y$ be particular integers so that $ax + by = c$.
    We will show this leads to a contradiction.
    By the properties of integer divisibility, let $q_0$ and $q_0$ be particular integers so that $a = dq_0$ and $b = dq_1$.
    By substitution, we see $(dq_0)x + (dq_1)y = c$.
    By the distributive law, we see $d(q_0 x + q_1 y) = c$.
    Since addition is closed on the integers, let $q_2$ be a particular integer so that $q_2 = q_0 x + q_1 y$ and $dq_2 = c$.
    By the properties of integer divisibility, we see $d | c$.
    This contradicts with the premise that $d \not | c$.
    Therefore, we see that if $d|a$, $d|a$, and $d\not | c$, $x$ and $y$ must not be integers and solutions to $ax + by = c$.
    $\blacksquare$

    \vspace{1cm}

%%%%%%%%%%%%%%%%%%%%%%%%%%%%%%%%%%%%%%%%%%%%%%%%%%%%%%%%%%%%%%%%%%%%%%%%%%%%%%%
    
    \noindent Problem 2.3.6 [DISCUSS, LATEX or 2.3.8]
    \vspace{0.3cm}
    \underline{proposition:}

    If $c\ge 2$ is a composite integer, then there exists a positive integer $b\ge 2$ such that $b|c$ and $b\le
    \sqrt{c}$.

    \vspace{0.3 cm}
    \underline{scaffold:} (Direct)

    \[\begin{array}{llr}
          c \in \Z \\
          c \geq 2 \\
          c \mbox{ is composite} \\
          \hline
          b, q \in \Z \mbox{, where $b$ is the smaller factor} \\
          2 \leq b < c \wedge 2 \leq q < c \mbox{, by properties of composite numbers} \\
          b \geq 2 \mbox{, $\Box$} \\
          bq = c \mbox{, by properties of composite numbers} \\
          b | c \mbox{, by properties of integer divisibility, $\Box$} \\
          b \leq q \\
          b^2 \leq qb \\
          b^2 \leq c \mbox{, by substitution} \\
          b \leq \sqrt{c} \mbox{, $\blacksquare$}
    \end{array}\]

    \vspace{0.3cm}
    \fbox{proof} ()
    \vspace{0.2cm}

    Let $c$ be an arbitrary integer so that $c \geq 2$ and $c$ is a composite integer.
    By the properties of composite integers, let $b$ and $q$ be particular integers so that $b \leq q$, $2 \leq b < c $,  $2 \leq q < c$, and $bq = c$.
    By the properties of integer divisibility, we see $b | c$.
    In this case, $b$ and $q$ are both factors of $c$, and if $b$ and $q$ are not equal, $b$ is the smaller factor.
    We see $b \leq q$.
    By multiplying each side of the inequality by $b$, we see $b^2 \leq qb$.
    By substitution, we see $b^2 \leq c$.
    By square rooting each side of the inequality, we see $b \leq \sqrt{c}$.
    Therefore, if $c$ is an arbitrary integer so that $c \geq 2$ and $c$ is a composite integer, there exists a particular integer $b$ so that $b \geq 2$, $b | c$, and $b \leq \sqrt{c}$.
    $\blacksquare$

    \vspace{1cm}

%%%%%%%%%%%%%%%%%%%%%%%%%%%%%%%%%%%%%%%%%%%%%%%%%%%%%%%%%%%%%%%%%%%%%%%%%%%%%%%

    \noindent Problem 2.3.8 [DISCUSS, LATEX or 2.3.6]
    \vspace{0.3cm}
    \underline{proposition:}

    Let $q\ge 2$ be a positive integer.
    If for all integers $a$ and $b$, whenever $q|ab$, $q|a$ or $q|b$, then $q$ is prime.

    \vspace{0.3 cm}
    \underline{scaffold:} (Contradiction)

    \[\begin{array}{llr}
          a, b, q \in \Z \\
          abk_0 = q \vee ak_1 = q \vee bk_2 = q \\
          q \mbox{ is composite, opposite of conclusion} \\
          \hline
          c_0, c_1 \in \Z \\
          2 \leq c_0 < q \wedge 2 \leq c_1 < q \\
          q = c_0 c_1 \mbox{, by properties of composite numbers} \\
          q | c_0 \vee q | c_1 \mbox{, by properties of integer divisibility}\\
          q \leq c_0 \vee q \leq c_1 \mbox{, contradiction, $\blacksquare$}
    \end{array}\]

    \vspace{0.4cm}
    \fbox{proof} ()
    \vspace{0.2cm}

    Let $q$ be an arbitrary integer so that $q \geq 2$.
    Let $a$ and $b$ be arbitrary integers.
    For all $a$ and $b$, suppose $q | ab$, $q | a$, or $q | b$.
    Suppose $q$ is a composite integer.
    We will show this leads to a contradiction.
    By the properties of composite numbers, let $c_0$ and $c_1$ be particular integers so that $2 \leq c_0 < q$, $2 \leq c_1 < q$, and $q = c_0 c_1$.
    By properties of integer divisibility, we see $q | c_0$ or $q | c_1$.
    Therefore, we see $q \leq c_0$ or $q \leq c_1$.
    This contradicts with $c_0 < q$ and $c_1 < q$.
    Therefore, we see that if $q | ab$, $q | a$, or $q | b$ for all $a$ and $b$, $q$ must not be composite.
    Therefore, $q$ must be prime.
    $\blacksquare$

    \vspace{1cm}

%%%%%%%%%%%%%%%%%%%%%%%%%%%%%%%%%%%%%%%%%%%%%%%%%%%%%%%%%%%%%%%%%%%%%%%%%%%%%%%

    \noindent Problem 2.2.4
    \vspace{0.3cm}
    \underline{proposition:}

    If $n$ is an even integer, then $n^2$ is even.
    If $n$ is odd, then $n^2$ is odd.

    \vspace{0.4cm}
    \fbox{proof} (Direct Proof)
    \vspace{0.2cm}

    Let $n$ be an even integer. Then by definition, there exists some
    integer $x$ such that $n=2x$. Thus, $$n^2=(2x)^2=(2x)(2x)=2(2x^2)$$
    by the associative and commutative properties of integer
    multiplication. Furthermore, since the integers are closed under
    multiplication, $2x^2$ is equal to some integer $s$. Thus,
    $n^2=2s$, which by definition of an even number, implies $n^2$ is
    even.

    \vspace{0.2cm}

    Let $n$ be an odd integer. By definition of an odd integer, there
    exists some integer $k$ such that $n=2k+1$. Thus,
    $$n^2=(2k+1)^2=(2k+1)(2k+1)=4k^2+4k+1$$ by the associative,
    commutative, and distributive properties of the integers.
    Furthermore, by the distributive property, $4k^2+4k+1=2(2k^2+2k)+1$.
    Since the integers are closed under multiplication and addition,
    $2k^2+2k=t$ for some integer $t$. Thus, $n^2=2t+1$, which by
    definition of an odd integer, implies $n^2$ is odd. $\blacksquare$

    \vspace{1cm}

%%%%%%%%%%%%%%%%%%%%%%%%%%%%%%%%%%%%%%%%%%%%%%%%%%%%%%%%%%%%%%%%%%%%%%%%%%%%%%%

    \noindent Problem 2.2.7
    \vspace{0.3cm}

    \underline{proposition:} Let $a, b, c,$ and $d$ be integers. If
    $a|b$ and $c|d$ then $ac|bd$.

    \vspace{0.4cm}

    \fbox{proof} (Direct Proof)

    \vspace{0.2cm}

    Suppose $a, b, c,$ and $d$ are integers and that $a|b$ and $c|d$. By
    definition of divisibility, there exist integers $x$ and $y$ such
    that $ax=b$ and $cy=d$. Thus, by the associative and commutative
    properties of integer multiplication, $bd=(ax)(cy)=ac(xy)$. Since
    the integers are closed under multiplication, $xy=s$ for some
    integer $s$. Thus, $bd=acs$, which by definition implies $ac|bd$.
    $\blacksquare$

    \vspace{1cm}

%%%%%%%%%%%%%%%%%%%%%%%%%%%%%%%%%%%%%%%%%%%%%%%%%%%%%%%%%%%%%%%%%%%%%%%%%%%%%%%

    \noindent Problem 2.3.2
    \vspace{0.3cm}

    \underline{proposition:} If $n$ is an integer and $n^2$ is even,
    then $n$ is even.

    \vspace{0.4cm}

    \fbox{proof} (Contrapositive)

    \vspace{0.2cm} Suppose $n$ is an odd integer. By definition, there
    exists an integer $m$ such that $n=2m+1$. Thus, by the associative,
    commutative, and distributive properties,
    $n^2=(2m+1)^2=4m^2+4m+1=2(2m^2+2m)+1$. Furthermore, since the
    integers are closed under addition and multiplication, $2m^2+2m=s$
    for some integer $s$. Thus, $n^2=2s+1$. By definition of odd, this
    implies $n^2$ is odd. Thus, by the contrapositive, if $n^2$ is
    even, $n$ must be even. $\blacksquare$

    \vspace{1cm}

%%%%%%%%%%%%%%%%%%%%%%%%%%%%%%%%%%%%%%%%%%%%%%%%%%%%%%%%%%%%%%%%%%%%%%%%%%%%%%%

    \noindent Problem 2.3.4
    \vspace{0.3cm}

    \underline{proposition:} If $y$ is a nonzero rational number and $x$
    is an irrational number, then $xy$ is irrational.

    \vspace{0.4cm}

    \fbox{proof} (Contradiction)

    \vspace{0.2cm} Suppose $y$ is a nonzero rational number, $x$ is an
    irrational number, and $xy$ is a rational number. We will show this
    leads to a contradiction. By definition of rational, since $xy$ and
    $y$ are rational numbers, there exist integers $m, n, l$ and $d$
    such that $xy=\frac{m}{n}$ and $y=\frac{l}{d}$. Thus,
    $$xy=x\frac{l}{d}=\frac{m}{n}.$$ Since $y$ is nonzero, $l\not=0$. Thus multiplication by $\frac{d}{l}$ is well
    defined. Multiplying both sides of $x\frac{l}{d}=\frac{m}{n}$ by
    $\frac{d}{l}$, we find $$x=\frac{m}{n}\frac{d}{l}=\frac{s}{t},$$
    where $s=md$ and $t=nl$. Since the integers are closed under
    multiplication, $s$ and $t$ are both integers. This implies $x$ is
    a rational number, which contradicts our assumption that $x$ is
    irrational. Thus, if $y$ is a nonzero rational number and $x$ is an
    irrational number, their product $xy$ is irrational. $\blacksquare$

    \vspace{1cm}

%%%%%%%%%%%%%%%%%%%%%%%%%%%%%%%%%%%%%%%%%%%%%%%%%%%%%%%%%%%%%%%%%%%%%%%%%%%%%%%

    \noindent Problem 2.3.7
    \vspace{0.3cm}

    \underline{proposition:} Let $q\ge 2$ be a positive integer. If for
    all integers $a$ and $b$, whenever $q|ab$, $q|a$ or $q|b$, then
    $\sqrt{q}$ is irrational.

    \vspace{0.4cm}

    \fbox{proof} (Contradiction)

    \vspace{0.2cm} Let $q$ be a positive integer greater than or equal
    to 2. Suppose that $\sqrt{q}$ is a rational number, and that for all
    integers $a$ and $b$, whenever $q|ab$, $q|a$ or $q|b$. We will show
    this leads to a contradiction. By definition of a rational number,
    $\sqrt{q}$ can be expressed as $\frac{n}{m}$ where $n$ and $m$ are
    integers, and $m\not=0$. Since $q\not=1$, $m$ and $n$ can be chosen
    in such a way that they do not share any common factors. Given
    $\sqrt{q}=\frac{n}{m}$ it follows that $q=\frac{n^2}{m^2}$.
    Multiplying both sides of this equation by $m^2$, we find
    $n^2=qm^2$. Since the integers are closed under multiplication,
    $m^2=s$ for some integer $s$. Thus, $n^2=qs$, which by definition,
    implies $q|n^2$. By our assumption, since $q|n^2$, $q$ must divide
    $n$. Thus there exists some integer $k$ such that $qk=n$.
    Substituting $qk$ in for $n$ in the equation $n^2=qm^2$, and
    applying the associative and commutative properties of integer
    multiplication, we find $q^2k^2=qm^2$. By the cancelation law,
    $qk^2=m^2.$ Again, by the closure of the integers under
    multiplication, there exists some integer $t=k^2$. Thus $qt=m^2$,
    which by definition, implies $q|m^2$. By our assumptions, $q$ must
    also divide $m$. Thus, $q$ is a common factor of $m$ and $n$, which
    contradicts the fact that $m$ and $n$ were chosen to have no common
    factors. Therefore, if for all integers $a$ and $b$, when $q|ab$,
    $q|a$ or $q|b$, then $\sqrt{q}$ must be irrational. $\blacksquare$

    \vspace{1cm}

%%%%%%%%%%%%%%%%%%%%%%%%%%%%%%%%%%%%%%%%%%%%%%%%%%%%%%%%%%%%%%%%%%%%%%%%%%%%%%%

\end{document}