% Program: PortfolioCh1S2-3.ltx
% Author: Carmen Beltran, Teddy Wachtler
% *********************************************************************

\documentclass{article}
\usepackage[dvips]{graphicx}
\usepackage{a4wide}
\usepackage{amsmath}
\usepackage{euscript}
\usepackage{amssymb}
\usepackage{amsthm}
\usepackage{amsopn}

\theoremstyle{definition}
\newtheorem*{definition}{Definition}
\newtheorem{theorem}{Theorem}

\newcommand{\vv}{\ensuremath{\vec{v}}}
\newcommand{\vu}{\ensuremath{\vec{u}}}
\newcommand{\vw}{\ensuremath{\vec{w}}}
\newcommand{\vx}{\ensuremath{\vec{x}}}
\newcommand{\vy}{\ensuremath{\vec{y}}}
\newcommand{\vb}{\ensuremath{\vec{b}}}
\newcommand{\vo}{\ensuremath{\vec{0}}}
\newcommand{\va}{\ensuremath{\vec{a}}}
\newcommand{\ve}{\ensuremath{\vec{e}}}

%%%%%%%%%%%%%%%%%%%%%%%%%%%%%%%%%%%%%%%%%%%%%%%

\begin{document}
    \begin{center}
        \Large{Math 251W: Foundations of Advanced Mathematics}

        \normalsize{Portfolio Assignment 1: \S 1.2-3}

        \vspace{0.2cm}

        \hfill {\bf Name:} Carmen Beltran, Teddy Wachtler

        \vspace{0.25cm}
        \hrule
    \end{center}

    \vspace{0.3cm}

%*****************************************************************

    \noindent Problem 1.3.8: State the {\it inverse, converse,} and {\it contrapositive} of each of the following statements. \emph{Hint: First write the original statement in If... then... form}.
    \begin{itemize}
        \item[(1)] If it's Tuesday, it must be Belgium.
        \item[(2)] I will go home if it is after midnight.
        \item[(3)] Good fences make good neighbors.
        \item[(4)] Lousy food is sufficient for a quick meal.
    \end{itemize}

    \vspace{0.25cm}
    \fbox{Solution}
    \begin{itemize}
        \item[(1)]
        \begin{itemize}
            \item Original: If it is Tuesday, then it must be Belgium.
            \item Inverse: If it is not Tuesday, it must not be Belgium.
            \item Converse: If it is Belgium, it must be Tuesday.
            \item Contrapositive: If it is not Belgium, it must not be Tuesday.
        \end{itemize}
        \item[(2)]
        \begin{itemize}
            \item Original: If it is after midnight, then I will go home.
            \item Inverse: If I will go home, then it is after midnight.
            \item Converse: If it is not after midnight, then I will not go home.
            \item Contrapositive: If I will not go home, then it is not after midnight.
        \end{itemize}
        \item[(3)]
        \begin{itemize}
            \item Original: If the fences are good, then I have good neighbors.
            \item Inverse: If my neighbors are good, then they will make good fences.
            \item Converse: If the fences are not good, then I have not good neighbors.
            \item Contrapositive: If my neighbors are not good, then they will make not good fences.
        \end{itemize}
        \item[(4)]
        \begin{itemize}
            \item Original: If I have lousy food, then I have a quick meal.
            \item Inverse: If I have a quick meal, then I have lousy food.
            \item Converse: If I have not lousy food, then I have a not quick meal.
            \item Contrapositive: If I have a not quick meal, then I have not lousy food.
        \end{itemize}
    \end{itemize}

    \vspace{.5cm}

%*****************************************************************

    \noindent Problem 1.3.10: Negate each of the following statements
    \begin{itemize}
        \item[(1)] $e^5>0.$
        \item[(2)] $3<5$ or $7\ge 8.$
        \item[(3)] $\sin(\pi/2)<0$ and $\tan(0)\ge 0.$
        \item[(4)] If $y=3$ then $y^2=7.$
        \item[(5)] $w-3>0$ implies $w^2+9>6w.$
        \item[(6)] $a-b=c$ iff $a=b+c$
    \end{itemize}

    \vspace{0.25cm}
    \fbox{Solution}
    \begin{itemize}
        \item[(1)] If the original statement is $P$, then a negation is $\neg P$. Thus, a negation is $e^5\le 0$.
        \item[(2)]  If the original statement is $P \vee Q$, then a negation is $\neg P \wedge \neg Q$. Thus, a negation is $3 \ge 5$ and $7 < 8$.
        \item[(3)]  If the original statement is $P \wedge Q$, then a negation is $\neg P \vee \neg Q$. Thus, a negation is $\sin(\pi/2) \ge 0$ or $\tan(0) < 0$.
        \item[(4)]  If the original statement is $P \rightarrow Q$, then a negation is $P \wedge \neg Q$. Thus, a negation is $y=3$ and $y^2 \ne 7$.
        \item[(5)]  If the original statement is $P \rightarrow Q$, then a negation is $P \wedge \neg Q$. Thus, a negation is $w - 3 > 0$ and $w^2 + 9 \le 6w$.
        \item[(6)]  If the original statement is $P \leftrightarrow Q$, then a negation is $(\neg P \wedge Q) \vee (P \wedge \neg Q)$. Thus, a negation is $a - b \ne c$ and $a = b + c$, or, $a - b = c$ and $a \ne b + c$.
    \end{itemize}

    \vspace{.5cm}

\end{document}
