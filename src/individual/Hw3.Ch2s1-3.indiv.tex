% Program: PortfolioCh2S1-3.ltx
% Author: Teddy Wachtler
% *********************************************************************

\documentclass{article}
\usepackage[dvips]{graphicx}
\usepackage{a4wide}
\usepackage{amsmath}
\usepackage{euscript}
\usepackage{amssymb}
\usepackage{amsthm}
\usepackage{amsopn}

\theoremstyle{definition}
\newtheorem*{definition}{Definition}
\newtheorem{theorem}{Theorem}

\newcommand{\vv}{\ensuremath{\vec{v}}}
\newcommand{\vu}{\ensuremath{\vec{u}}}
\newcommand{\vw}{\ensuremath{\vec{w}}}
\newcommand{\vx}{\ensuremath{\vec{x}}}
\newcommand{\vy}{\ensuremath{\vec{y}}}
\newcommand{\vb}{\ensuremath{\vec{b}}}
\newcommand{\vo}{\ensuremath{\vec{0}}}
\newcommand{\va}{\ensuremath{\vec{a}}}
\newcommand{\ve}{\ensuremath{\vec{e}}}

\newcommand{\R}{\mathbb{R}}
\newcommand{\Z}{\mathbb{Z}}
\newcommand{\C}{\mathbb{C}}
\newcommand{\N}{\mathbb{N}}
\newcommand{\Q}{\mathbb{Q}}

%%%%%%%%%%%%%%%%%%%%%%%%%%%%%%%%%%%%%%%%%%%%%%%

\begin{document}
    \begin{center}
        \Large{Math 251W: Foundations of Advanced Mathematics}

        \normalsize{Portfolio Assignment 3: \S 2.1-3}

        \vspace{0.2cm}

        \hfill {\bf Name:} Teddy Wachtler

        \vspace{0.25cm}
        \hrule
    \end{center}

    \vspace{0.3cm}

%%%%%%%%%%%%%%%%%%%%%%%%%%%%%%%%6%%%%%%%%%%%%%%%%%%%%%%%%%%%%%%%%%%%%%%%%%%%%%

    \noindent Problem 2.3.5  \vspace{0.3cm}
    \underline{proposition:}

    Let $a, b,$ and $c$ be integers.
    If there exists an integer $d$ such that $d|a$ and $d|b$ but $d\not|c$, then $ax+by=c$ has no integer solutions for $x$ and $y$.

    \vspace{0.3 cm}
    \underline{scaffold:} (Contradiction)

    \[\begin{array}{llr}
          a, b, c, d, x, y \in \Z \\
          d | a \\
          d | b \\
          d \not| c \\
          ax + by = c \mbox{, the opposite of the conclusion} \\
          \hline
          q_0, q_1 \in \Z \\
          a = dq_0 \mbox{, by properties of divides} \\
          b = dq_1 \\
          (dq_0)x + (dq_1)y = c \\
          d(q_0 x + q_1 y) = c \\
          q_2 \in \Z \\
          dq_2 = c  \mbox{, since addition is closed} \\
          d | c \mbox{, contradicts with $d \not | c$, $\blacksquare$}
    \end{array}\]

    \vspace{0.4cm}
    \fbox{proof} (Contradiction) \\
    \vspace{0.2cm} \\

    Let a, b, and c be arbitrary integers.
    Let d be a particular integer so that $d|a$, $d|a$, and $d\not | c$.
    Let $x$ and $y$ be particular integers so that $ax + by = c$.
    We will show this leads to a contradiction.
    By the properties of integer divisibility, let $q_0$ and $q_0$ be particular integers so that $a = dq_0$ and $b = dq_1$.
    By substitution, we see $(dq_0)x + (dq_1)y = c$.
    By the distributive law, we see $d(q_0 x + q_1 y) = c$.
    Since addition is closed on the integers, let $q_2$ be a particular integer so that $q_2 = q_0 x + q_1 y$ and $dq_2 = c$.
    By the properties of integer divisibility, we see $d | c$.
    This contradicts with the premise that $d \not | c$.
    Therefore, we see that if $d|a$, $d|a$, and $d\not | c$, $x$ and $y$ must not be integers and solutions to $ax + by = c$.
    $\blacksquare$

    \vspace{1cm}

\end{document}