% Program: PortfolioCh2S4-5.ltx
% Author: Michael Cotten, Isaiah Trotter, Teddy Wachtler
%
% Description: Portfolio Assignment 4: Sections 2.4 & 2.5
% *********************************************************

\documentclass{article}
\usepackage[dvips]{graphicx}
\usepackage{a4wide}
\usepackage{amsmath}
\usepackage{euscript}
\usepackage{amssymb}
\usepackage{amsthm}
\usepackage{amsopn}

\theoremstyle{definition}
\newtheorem*{definition}{Definition}
\newtheorem{theorem}{Theorem}

\renewcommand{\qed}{\blacksquare}

\newcommand{\vv}{\ensuremath{\vec{v}}}
\newcommand{\vu}{\ensuremath{\vec{u}}}
\newcommand{\vw}{\ensuremath{\vec{w}}}
\newcommand{\vx}{\ensuremath{\vec{x}}}
\newcommand{\vy}{\ensuremath{\vec{y}}}
\newcommand{\vb}{\ensuremath{\vec{b}}}
\newcommand{\vo}{\ensuremath{\vec{0}}}
\newcommand{\va}{\ensuremath{\vec{a}}}
\newcommand{\ve}{\ensuremath{\vec{e}}}

\newcommand{\R}{\mathbb{R}}
\newcommand{\Z}{\mathbb{Z}}
\newcommand{\C}{\mathbb{C}}
\newcommand{\N}{\mathbb{N}}
\newcommand{\Q}{\mathbb{Q}}

%%%%%%%%%%%%%%%%%%%%%%%%%%%%%%%%%%%%%%%%%%%%%%%%%%%%%%%%%%%

\begin{document}
	\begin{center}
		\Large{Math 251W: Foundations of Advanced Mathematics}

		\normalsize{Portfolio Assignment 4: \S 2.4 \& 2.5}

		\vspace{0.2cm}

		\hfill {\textbf Name:} Michael Cotten, Isaiah Trotter, Teddy Wachtler

		\vspace{0.25cm}
		\hrule
	\end{center}

	\vspace{0.3cm}

%%%%%%%%%%%%%%%%%%%%%%%%%%%%%%%%%%%%%%%%%%%%%%%%%%%%%%%%%%%

	\noindent Problem 2.4.3 [DISCUSS]
	\vspace{0.3cm}

	For the following proposition, give a list of conditional statements you would prove to prove the entire stated equivalence.
	Then prove \textbf{one} of those conditional statements.

	\medskip

	\underline{proposition:} Given $a$ and $b$ are positive integers,
	the following are equivalent:
	\begin{itemize}
		\item[i] $a$ and $b$ are relatively prime.
		\item[ii] $a+b$ and $b$ are relatively prime.
		\item[iii] $a$ and $a+b$ are relatively prime.
	\end{itemize}

	\vspace{0.4cm}

	Statements:
	\begin{itemize}
		\item[i] For all integers $a$ and $b$, if $a$ and $b$ are coprime, then $a$ and $a + b$ are coprime.
		\item[ii] For all integers $a$ and $b$, if $a$ and $a + b$ are coprime, then $a$ and $b$ are coprime.
	\end{itemize}

	\vspace{0.2cm}
	\fbox{proof} (Contradiction)
	\vspace{0.2cm}

	Let $a$ and $b$ be arbitrary integers so that $a$ and $b$ are coprime.
	Suppose $a$ and $a + b$ are not coprime.
	We will show this to be false through a contradiction.
	By the definition of coprime numbers, let $d$ be a particular integer so that $d|a$, $d|(a + b)$ and $d \not = \pm 1$.
	By the definition of integer divisibility, let $f$ and $g$ be particular integers so that $df = a$ and $dg = a + b$.
	By subtracting $b$ from both sides, we see $dg - b = a$.
	By substitution, we see $df = dg - b$.
	By adding $b$ to both sides, we see $b + df = dg$.
	By subtracting $df$ from both sides, we see $b = dg - df$.
	By the distributive law, we see $b = d(g - f)$.
	Since addition is closed on the integers, let $h$ be a particular integer so that $h = g - f$.
	By substitution, we see $b = dh$.
	By the definition of integer divisibility, we see $d|b$.
	This contracts with the fact that $a$ and $b$ are coprime as $d|a$ and $d \not = \pm 1$.
	Therefore, for all integers $a$ and $b$, if $a$ and $b$ are coprime, then $a$ and $a + b$ are coprime.
	$\qed$.

	\vspace{1cm}

%%%%%%%%%%%%%%%%%%%%%%%%%%%%%%%%%%%%%%%%%%%%%%%%%%%%%%%%%%%

	\noindent Problem 2.4.6 [DISCUSS]
	Fill in, and then prove, the following proposition.
	\vspace{0.3cm}

	\medskip

	\underline{proposition}
	The only triple primes (i.e., primes of the form $p, p + 2,$ and $p + 4$ for some integer $p$) are $3, 5$, and $7$.

	\vspace{0.4cm}
	\fbox{proof} (Direct)
	\vspace{0.2cm}

	Let $p$ be an arbitrary prime natural number.
	Let $q$ and $r$ be particular prime natural numbers so that $q = p + 2$ and $r = p + 4$.
	Let $p, q$, and $r$ be considered triple primes.
	Consider that every third natural number must be composite except for $3$ itself because it is divisible by $3$.
	Since $p$ is prime, let $a$ and $b$ be particular integers so that we see $p = 3, p = 3a + 1$, or $p = 3b + 2$.

	Consider the case where $p = 3$.
	If $p = 3$, then by substitution and simplification we see $q = (3) + 2 = 5$ and $r = (3) + 4 = 7$.
	Therefore, $3, 5$ and $7$ are triple primes.

	Consider the case where $p = 3a + 1$.
	Since the smallest prime in this case is $7$, we see that $7 + 2 = 9 \leq q$.
	If $p = 3a + 1$, then substitution, simplification and the distributive law we see $q = (3a + 1) + 2 = 3a + 3 = 3(a + 1)$.
	Since addition is closed on the integers, let $c$ be a particular integer so that $c = a + 1$.
	By substitution, we see $q = 3c$.
	By the properties of integer divisibility, we see $3|q$.
	Therefore, since $q$ must be larger than $3$, we see that $q$ must be composite and this case has no triple primes.

	Consider the case where $p = 3b + 2$.
	Since the smallest prime in this case is $4$, we see that $2 + 4 = 6 \leq r$.
	If $p = 3b + 2$, then substitution, simplification and the distributive law we see $r = (3b + 2) + 4 = 3b + 6 = 3(b + 2)$.
	Since addition is closed on the integers, let $d$ be a particular integer so that $d = b + 1$.
	By substitution, we see $r = 3d$.
	By the properties of integer divisibility, we see $3|r$.
	Therefore, since $r$ must be larger than $3$, we see that $r$ must be composite and this case has no triple primes.

	Therefore, the only case of triple primes occurs when $p = 3$.

	Therefore, for all natural numbers $p$, if $p, p + 2$, and $p + 4$ are all prime, then $p = 3$.
	$\qed$.

	\vspace{1cm}

%%%%%%%%%%%%%%%%%%%%%%%%%%%%%%%%%%%%%%%%%%%%%%%%%%%%%%%%%%%

	\noindent Problem 2.4.8 [INDIVIDUAL]
	Prove the following
	\vspace{0.3cm}

	\underline{proposition:}
	If $n$ is an odd integer, then there exists an integer $k$ such that $n^2=8k+1$.

	\vspace{0.4cm}
	\fbox{proof} (Direct)
	\vspace{0.2cm}

	Let $n$ be an arbitrary odd integer.
	By the definition of odd integers, let $a$ be a particular integer so that $n = 2a + 1$.
	By taking both sides to the power of two, we see $n^2 = (2a + 1)^2$.
	By simplification and the distributive law, we see \[n^2 = 4a^2 + 4a + 1 = n^2 = 4(a^2 + a) + 1.\]
	Consider the parity of $a$.
	Consider $a_{even}$ to represent $a$ in the case that $a$ is even.
	By the properties of even numbers, let $b$ be an integer so that $a_{even} = 2b$.
	By substitution, simplification, and the distributive law we see \[a_{even}^2 + a_{even} = (2b)^2 + (2b) = 4b^2 + 2b = 2(2b^2 + b).\]
	Consider $a_{odd}$ to represent $a$ in the case that $a$ is odd.
	By the properties of odd numbers, let $c$ be an integer so that $a_{odd} = 2c + 1$.
	By substitution, simplification, and the distributive law we see \[a_{odd}^2 + a_{odd} = (2c + 1)^2 + (2c + 1) = 4c^2 + 4c + 1 + 2c + 1 = 4c^2 + 6c + 2 = 2(2c^2 + 3c + 1).\]
	Since addition is closed, let $k$ be an integer so that $k = b^2 + 2b$ or $k = c^2 + 3c + 1$.
	By substitution, we see $a_{odd}^2 + a_{odd} = 2k$ or $a_{even}^2 + a_{even} = 2k$.
	Since an integer must be even or odd, and since the value of $a^2 + a$ is the same irrespective of the parity of $a$, we see $a^2 + a = 2k$.
	By substitution, we see $n^2 = 4(2k) + 1$.
	By simplification, we see $n^2 = 8k + 1$.
	Therefore, for all odd integers $n$, there exists an integer $k$ so that $n^2 = 8k + 1$.
	$\qed$.

	\vspace{1cm}

%%%%%%%%%%%%%%%%%%%%%%%%%%%%%%%%%%%%%%%%%%%%%%%%%%%%%%%%%%%

	\noindent Problem 2.5.8 [GROUP or 2.5.9]
	Prove or give a counterexample
	\vspace{0.3cm}

	\underline{proposition:}
	For each real number $p$, there exist real numbers $q$ and $r$ such that \[q\sin\left(\frac{r}{5}\right)=p.\]

	\vspace{0.3cm}
	\fbox{proof} (Direct)
	\vspace{0.2cm}

	Let $p$ be an arbitrary real number.
	Let $q$ be a particular real number so that $q = p$.
	By the multiplicative identity property, we see $q \cdot 1 = p$.
	Let $r$ be a particular real number so that \[r = \frac{5\pi}{2}.\]
	By dividing both sides by five and simplification, we see \[\frac{r}{5} = \frac{\frac{5\pi}{2}}{5} = \frac{\pi}{2}.\]
	By applying the $\sin$ function to both sides and simplification, we see \[\sin\left(\frac{r}{5}\right) = \sin\left(\frac{\pi}{2}\right) = 1.\]
	By substitution, we see \[q \cdot 1 = q \sin\left(\frac{r}{5}\right) = p.\]
	Therefore, for all real numbers $p$, there exist real numbers $q$ and $r$ so that \[q\sin\left(\frac{r}{5}\right)=p.\]
	$\qed$.

	\vspace{1cm}

%%%%%%%%%%%%%%%%%%%%%%%%%%%%%%%%%%%%%%%%%%%%%%%%%%%%%%%%%%%

	\noindent Problem 2.5.9 [GROUP or 2.5.8]
	Prove or give a counterexample
	\vspace{0.3cm}

	\underline{proposition:}
	For each integer $x$, and for each integer $y$, there exists an integer $z$ such that $z^2+2xz-y^2=0$.

	\vspace{0.4cm}
	\fbox{counterproof}
	\vspace{0.2cm}

	We ask whether for all integers $x$ and $y$ there exists a particular integer $z$ so that $z^2 + 2xz - y^2 = 0$.
	We will show this to be false through a counterexample.
	Suppose $x = 1$ and $y = 1$.
	By substitution and simplification we see $z^2 + 2(1)z - (1)^2 = 0$.
	By simplification, we see $z^2 + 2z - 1 = 0$.
	We will solve this quadratic equation by completing the square.
	By adding $2$ to both sides, we see $z^2 + 2z + 1 = 2$.
	By the distributive law, we see $(z + 1)^2 = 2$.
	By taking the square root of both sides, we see $z + 1 = \pm \sqrt{2}$.
	By subtracting 1 from both sides, we see $z = -1 \pm \sqrt{2}$.
	Since the square root of $2$ is an irrational number, and since the sum of an integer and an irrational number is always an irrational number, we see that $z$ must be irrational.
	Therefore, it is not true that for all integers $x$ and $y$, there exists an integer $z$ so that $z^2 + 2xz - y^2 = 0$.
	$\qed$.

	\vspace{1cm}

\end{document}