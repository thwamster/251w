% Program: PortfolioCh5s1s3.tex
% Author: Erin McNicholas
%
%
% Note: Comments made after a % sign are not read by the compiler.
%       are notational comments in the code.
% *********************************************************************

% *********************************************************************
%     Header Commands: These are commands that format the document and
%         define new command shortcuts.  You can use the \newcommand
%         function to define shortcuts for commonly used commands.
%         If you are just learning LaTeX, you should not need to 
%         modify this portion of the code
% *********************************************************************


\documentclass{article}
\usepackage[dvips]{graphicx}
\usepackage{a4wide}
\usepackage{amsmath}
\usepackage{euscript}
\usepackage{amssymb}
\usepackage{amsthm}
\usepackage{amsopn}

\theoremstyle{definition}
\newtheorem*{definition}{Definition}
\newtheorem{theorem}{Theorem}

\newcommand{\R}{\mathbb{R}}
\newcommand{\Z}{\mathbb{Z}}
\newcommand{\C}{\mathbb{C}}
\newcommand{\N}{\mathbb{N}}
\newcommand{\Q}{\mathbb{Q}}
\newcommand{\D}{\ensuremath{{\cal D}}}
\newcommand{\E}{\mathcal{E}}
\newcommand{\T}{\mathcal{T}}
\newcommand{\M}{\mathcal{M}}
\newcommand{\com}{\psi\circ\phi}
%%%%%%%%%%%%%%%%%%%%%%%%%%%%%%%%%%%%%%%%%%%%%%%

% *********************************************************************
%     \begin{document} starts the portion of the code that will 
%         be translated into text.  This is the portion of the code
%         you will modify to insert your answers in
% *********************************************************************
\begin{document}
\begin{center}
\Large{Math 251W: Foundations of Advanced Mathematics}

\normalsize  Portfolio problems from section 5.2 \& 5.3

\hfill Name: 

\vspace{0.4cm}
\hrule \vspace{0.4cm}


\vspace{0.75cm}
\end{center}




%%%%%%%%%%%%%%%%%%%%%%%%%%%%%%%%%%%%%%%%%%%%%%%%%%%%%%%%%%%%%%%%%%%%%%%%%%


\noindent Problem 5.3.14  \vspace{0.3cm}


Let $A$ be a non-empty set, and let $E_1$ and $E_2$ be equivalence
relations on $A$ with associated partitions $\D _1$ and $\D _2$,
respectively.  Let $E$ be the equivalence relation on $A$ defined by
$E=E_1\cap E_2$, and $\D$ its associated partition.  Let $\D$ denote the partition of $A$ corresponding to $E$.  Define $\D$ in terms of $\D_1$ and $\D_2$ and prove your result.

\underline{proposition:}  $\D =$.

\vspace{.2cm}

\fbox{proof}
\vspace{.15cm}


\vspace{1cm}

%%%%%%%%%%%%%%%%%%%%%%%%%%%%%%%%%%%%%%%%%%%%%%%%%%%%%%%%%%%%%%%%%%%%%%%%%%

\noindent Problem Big Bijection Proof\vspace{0.3cm}

See the big bijection proof in Perusall for a complete explanation of the symbols in the following propositions.

\begin{itemize}
    \item[a.] \underline{proposition:}  $[a]_{\psi(\M)}\subseteq S.$

    \vspace{.2cm}

\fbox{proof}
\item[b.] \underline{proposition:}  $\mathcal{R}\subseteq(\com)(\mathcal{R}).$

\vspace{.2cm}

\fbox{proof}
\end{itemize}

\vspace{1cm}


%%%%%%%%%%%%%%%%%%%%%%%%%%%%%%%%%%%%%%%%%%%%%%%%%%%%%%%%%%%%%%%%%%%%%%%%%%

\noindent Problem 5.2.4  \vspace{0.3cm}

\underline{proposition:}  Let $n,q \in  \N$ and let $a,b\in\Z.$  Suppose $a\equiv b\pmod{n}$ and that $q|n.$  Then $a\equiv b\pmod{q}.$

\vspace{.2cm}

\fbox{proof}

\vspace{1cm}


%%%%%%%%%%%%%%%%%%%%%%%%%%%%%%%%%%%%%%%%%%%%%%%%%%%%%%%%%%%%%%%%%%%%%%%%%%


\noindent Problem 5.2.9  \vspace{0.3cm}

\underline{proposition:}  Given $n\in \N$, one of the following is
true: $n^2\equiv 0 \mod 16$, $n^2\equiv 1 \mod 8$, or $n^2\equiv 4
\mod 8$

\vspace{.2cm}

\fbox{proof}

\vspace{1cm}


%%%%%%%%%%%%%%%%%%%%%%%%%%%%%%%%%%%%%%%%%%%%%%%%%%%%%%%%%%%%%%%%%%%%%%%%%%




\end{document}