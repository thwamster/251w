\documentclass{article}
\usepackage[dvips]{graphicx}
\usepackage{a4wide}
\usepackage{amsmath}
\usepackage{euscript}
\usepackage{amssymb}
\usepackage{amsthm}
\usepackage{amsopn}

\theoremstyle{definition}
\newtheorem*{definition}{Definition}
\newtheorem{theorem}{Theorem}

\newcommand{\vv}{\ensuremath{\vec{v}}}
\newcommand{\vu}{\ensuremath{\vec{u}}}
\newcommand{\vw}{\ensuremath{\vec{w}}}
\newcommand{\vx}{\ensuremath{\vec{x}}}
\newcommand{\vy}{\ensuremath{\vec{y}}}
\newcommand{\vb}{\ensuremath{\vec{b}}}
\newcommand{\vo}{\ensuremath{\vec{0}}}
\newcommand{\va}{\ensuremath{\vec{a}}}
\newcommand{\ve}{\ensuremath{\vec{e}}}

\newcommand{\R}{\mathbb{R}}
\newcommand{\Z}{\mathbb{Z}}
\newcommand{\C}{\mathbb{C}}
\newcommand{\N}{\mathbb{N}}
\newcommand{\Q}{\mathbb{Q}}

%%%%%%%%%%%%%%%%%%%%%%%%%%%%%%%%%%%%%%%%%%%%%%%


\begin{document}
\begin{center}
\Large{Math 251W: Foundations of Advanced Mathematics}

\normalsize Solutions to problems from section  5.2 \hrule
\vspace{0.4cm}

\end{center}




%%%%%%%%%%%%%%%%%%%%%%%%%%%%%%%%%%%%%%%%%%%%%%%%%%%%%%%%%%%%%%%%%%%%%%%%%%


\noindent Problem 5.2.2  \vspace{0.3cm}

\begin{itemize}
	\item[1.] $x=[5]$
	\item[2.] $x=[9]$
	\item[3.] No solutions.  One way to see this is to imagine a solution exists.  Then by definition of modular congruence there is a value for x such that $15|(6x-4)$.  Of course if $15|(6x-4)$ then $3|(6x-4)$.  But since $3|6x$ for all values of $x$, this would suggest that $3$ also divides $4$ which is not true.
	\item[4.] $x=[2]$
	\item[5.] $x=[-1]=[4]$
\end{itemize}

\vspace{1cm}


%%%%%%%%%%%%%%%%%%%%%%%%%%%%%%%%%%%%%%%%%%%%%%%%%%%%%%%%%%%%%%%%%%%%%%%%%%

\noindent Problem 5.2.3  \vspace{0.3cm}

There are many possible answers.  One is $n=4$, $a=1$ and $b=3$.  Thus $$a^2=1=9=b^2\pmod{4}.$$
\vspace{1cm}


%%%%%%%%%%%%%%%%%%%%%%%%%%%%%%%%%%%%%%%%%%%%%%%%%%%%%%%%%%%%%%%%%%%%%%%%%%




\noindent Problem 5.2.6  \vspace{0.3cm}


\begin{enumerate}

\item[1]

\underline{proposition:} Let $a,b,c\in \Z$ and let $n\in \N$.  Then
$a+c\equiv b+c \mod n$ implies $a\equiv b \mod n$.

\vspace{.2cm}

\fbox{proof} (Direct) \vspace{.15cm}

By definition of modular equivalence, $a+c\equiv b+c \mod n$ implies
$n|((a+c)-(b+c))$.  But $(a+c)-(b+c)=a-b$, thus this implies
$n|(a-b)$.  By the definition of modular equivalence, if $n|(a-b)$
then $a\equiv b \mod n$.  $\blacksquare$

\vspace{.2cm}

\item[2]

\underline{proposition:} Let $a,b,c\in \Z$ and let $n\in \N$.  Then
$ac\equiv bc \mod n$ implies $a\equiv b \mod n$.

\vspace{.2cm}

\fbox{counterexample} \vspace{.15cm}

Let $a=7$, $b=4$, $c=2$, and $n=6$.  $$7\cdot 2\equiv 4\cdot 2 \mod
6\qquad \mbox{but}\qquad 7\not\equiv 4\mod 6$$

\end{enumerate}

\vspace{1cm}

\newpage

%%%%%%%%%%%%%%%%%%%%%%%%%%%%%%%%%%%%%%%%%%%%%%%%%%%%%%%%%%%%%%%%%%%%%%%%%%

\noindent Problem 5.2.7  \vspace{0.3cm}

\underline{proposition:} For all $n\in\N$, if $(n-1)!=-1\pmod{n}$ then $n$ is prime.

\vspace{.2cm}

\fbox{proof} (Contradiction) \vspace{.15cm}

Let $n$ be an arbitrary natural number.  Suppose by way of contradiction that $(n-1)!=-1\pmod{n}$ and $n$ is composite. By definition of modular equivalence, $n|((n-1)!-(-1))$.  By definition of divides, there exists an integer $l$ such that $nl=(n-1)!+1.$

By definition of composite, there exists an integer $m$ such that $1<m<n$ and $m|n.$  By definition of divides, there exists an integer $t$ such that $mt=n.$  Since $m$ is greater than $1$ and less than $n$, it is contained in the set $\{1,2,\ldots,n-1\}$, and thus is a factor in the product $(n-1)!=(n-1)\cdot(n-2)\cdots2\cdot 1$.  Thus, $(n-1)!=m\cdot k$ where $k$ is the product of all the integers  in the set $\{1,2,\ldots,n-1\}-\{m\}$.  By the closure of integers under multiplicaiton, $k$ is an integer.

Thus, by substitution, $nl=(n-1)!+1$ implies $(mt)l=mk+1$.  Subtracting $mk$ from both sides of this equation, and using the associative and distributive properties of integer multiplicaiton and addition, we have $m(tl-k)=1$.  By the closure property of integer multiplicaiton and addition $(tl-k)$ is an integer, and thus by definition of divides $m|1$.  However, the only integer divisors of $1$ are $1$ and $-1$, and by assumption, $m>1$.  Thus we have a contradiction.

Therefore, if $(n-1)!=-1\pmod{n}$ then $n$ is prime. $\blacksquare$


\vspace{1cm}


%%%%%%%%%%%%%%%%%%%%%%%%%%%%%%%%%%%%%%%%%%%%%%%%%%%%%%%%%%%%%%%%%%%%%%%%%%


\noindent Problem 5.2.11 \vspace{0.3cm}

Note: the relationship described in this proposition is the model used by several divisibility tests.

\vspace{.2cm}

\underline{proposition:} $\sum_{i=1}^ma_i10^{i-1}=\sum_{i=1}^ma_i\pmod{9}$

\vspace{.2cm}

\fbox{proof}(Direct) \vspace{.15cm}

Consider the sum $\sum_{i=1}^ma_i10^{i-1}\pmod{9}$.  By properties of modular equivalence, if $c=d\pmod{9}$ then $ac=ad\pmod{9}$.  Thus, since $10=1\pmod{9}$ we have $$\sum_{i=1}^ma_i10^{i-1}=\sum_{i=1}^ma_i1^{i-1}=\sum_{i=1}^ma_i1=\sum_{i=1}^ma_i\pmod{9} \qquad \blacksquare$$



\vspace{1cm}


%%%%%%%%%%%%%%%%%%%%%%%%%%%%%%%%%%%%%%%%%%%%%%%%%%%%%%%%%%%%%%%%%%%%%%%%%%






\end{document}
