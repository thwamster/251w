% Program: PortfolioCh5s1s3.tex
% Author: Erin McNicholas
%
%
% Note: Comments made after a % sign are not read by the compiler.
%       are notational comments in the code.
% *********************************************************************

% *********************************************************************
%     Header Commands: These are commands that format the document and
%         define new command shortcuts.  You can use the \newcommand
%         function to define shortcuts for commonly used commands.
%         If you are just learning LaTeX, you should not need to 
%         modify this portion of the code
% *********************************************************************


\documentclass{article}
\usepackage[dvips]{graphicx}
\usepackage{a4wide}
\usepackage{amsmath}
\usepackage{euscript}
\usepackage{amssymb}
\usepackage{amsthm}
\usepackage{amsopn}

\theoremstyle{definition}
\newtheorem*{definition}{Definition}
\newtheorem{theorem}{Theorem}

\newcommand{\vv}{\ensuremath{\vec{v}}}
\newcommand{\vu}{\ensuremath{\vec{u}}}
\newcommand{\vw}{\ensuremath{\vec{w}}}
\newcommand{\vx}{\ensuremath{\vec{x}}}
\newcommand{\vy}{\ensuremath{\vec{y}}}
\newcommand{\vb}{\ensuremath{\vec{b}}}
\newcommand{\vo}{\ensuremath{\vec{0}}}
\newcommand{\va}{\ensuremath{\vec{a}}}
\newcommand{\ve}{\ensuremath{\vec{e}}}
%%%%%%%%%%%%%%%%%%%%%%%%%%%%%%%%%%%%%%%%%%%%%%%

% *********************************************************************
%     \begin{document} starts the portion of the code that will 
%         be translated into text.  This is the portion of the code
%         you will modify to insert your answers in
% *********************************************************************
\begin{document}
\begin{center}
\Large{Math 251W: Foundations of Advanced Mathematics}

\normalsize  Portfolio problems from section 5.1 \& 5.3

\hfill Name: 

\vspace{0.4cm}
\hrule \vspace{0.4cm}


\vspace{0.75cm}
\end{center}




%%%%%%%%%%%%%%%%%%%%%%%%%%%%%%%%%%%%%%%%%%%%%%%%%%%%%%%%%%%%%%%%%%%%%%%%%%

\noindent Problem 5.1.8  \vspace{0.3cm}

Let $A$ be a set.  Let $\{R_i\}_{i\in I}$ be a family of relations
on $A$ indexed by the nonempty set $I$.  

\begin{enumerate}
\item[a] Prove each of the following propositions:
\begin{enumerate}
\item[i] If $R_i$ is reflexive for all $i\in I$ then
$\bigcap_{i\in I} R_i$ is reflexive
\item[ii] If $R_i$ is symmetric for all $i\in I$ then
$\bigcap_{i\in I} R_i$ is symmetric
\item[iii] If $R_i$ is transitive for all $i\in I$ then
$\bigcap_{i\in I} R_i$ is transitive
\end{enumerate}

 \vspace{.2cm}



\vspace{.2cm}

\item[b] Find a counterexample to the proposition: 
 If $R_i$ is transitive for all $i\in I$ then
$\bigcup_{i\in I} R_i$ is transitive


 \vspace{.2cm}


\end{enumerate}
\vspace{1cm}



%%%%%%%%%%%%%%%%%%%%%%%%%%%%%%%%%%%%%%%%%%%%%%%%%%%%%%%%%%%%%%%%%%%%%%%%%%%%%%%%

\noindent Problem 5.1.11  \vspace{0.3cm}

\begin{enumerate}
\item[3] \underline{proposition:} If ${\cal R}$ is a transitive
relation on the set $A$ and $x{\cal R}y$ then $[y]\subseteq [x]$

\vspace{0.2cm}

\fbox{proof}

\vspace{0.2cm}


\end{enumerate}

\vspace{1cm}



% %%%%%%%%%%%%%%%%%%%%%%%%%%%%%%%%%%%%%%%%%%%%%%%%%%%%%%%%%%%%%%%%%%%%%%%%%%
%%%%%%%%%%%%%%%%%%%%%%%%%%%%%%%%%%%%%%%%%%%%%%%%%%%%%%%%%%%%%%%%%%%%%%%%%%

\noindent Problem 5.3.6  \vspace{0.3cm}

Let $A$ be a non-empty set, let $\sim$ be an equivalence
relation on $A$, and let $A/\sim$ be the set of equivalence classes (i.e. $A/\sim\quad=\{[a]|a\in A\}$). Note it is common to use $\sim$ instead of $\mathcal{R}$ when the relation is an equivalence relation. 

\medskip

\underline{proposition:} Let $x,y\in A$.
\begin{enumerate}
\item[a] If $x\sim y$, then $[x]=[y]$
\item[b] If $x\not\sim y$ then $[x]\cap [y]=\emptyset$
\item[c] $\bigcup_{[x]\in A/\sim}[x]=A$
\end{enumerate}

\begin{enumerate}
\item[a]
\fbox{proof} \vspace{.15cm}

\item[b]
\fbox{proof} \vspace{.15cm}


\item[c]
\fbox{proof} \vspace{.15cm}

\end{enumerate}


\vspace{.2cm}






%%%%%%%%%%%%%%%%%%%%%%%%%%%%%%%%%%%%%%%%%%%%%%%%%%%%%%%%%%%%%%%%%%%%%%%%%%



\end{document}