% Program: PortfolioCh4S1-2.tex
% Author: Erin McNicholas
% Date: 
%
% Description: Portfolio Assignment 7: Sections 4.1 and 4.2
%
% Note: Comments made after a % sign are not read by the compiler.
%       are notational comments in the code.
% *********************************************************************

% *********************************************************************
%     Header Commands: These are commands that format the document and
%         define new command shortcuts.  You can use the \newcommand
%         function to define shortcuts for commonly used commands.
%         If you are just learning LaTeX, you should not need to 
%         modify this portion of the code
% *********************************************************************


\documentclass{article}
\usepackage[dvips]{graphicx}
\usepackage{a4wide}
\usepackage{amsmath}
\usepackage{euscript}
\usepackage{amssymb}
\usepackage{amsthm}
\usepackage{amsopn}

\theoremstyle{definition}
\newtheorem*{definition}{Definition}
\newtheorem{theorem}{Theorem}

\newcommand{\vv}{\ensuremath{\vec{v}}}
\newcommand{\vu}{\ensuremath{\vec{u}}}
\newcommand{\vw}{\ensuremath{\vec{w}}}
\newcommand{\vx}{\ensuremath{\vec{x}}}
\newcommand{\vy}{\ensuremath{\vec{y}}}
\newcommand{\vb}{\ensuremath{\vec{b}}}
\newcommand{\vo}{\ensuremath{\vec{0}}}
\newcommand{\va}{\ensuremath{\vec{a}}}
\newcommand{\ve}{\ensuremath{\vec{e}}}
%%%%%%%%%%%%%%%%%%%%%%%%%%%%%%%%%%%%%%%%%%%%%%%

% *********************************************************************
%     \begin{document} starts the portion of the code that will 
%         be translated into text.  This is the portion of the code
%         you will modify to insert your answers in
% *********************************************************************

\begin{document}
\begin{center}
\Large{Math 251W: Foundations of Advanced Mathematics}

\normalsize  Portfolio problems from sections 4.1, $\&$
4.2 

\hfill Name: 

\hrule \vspace{0.4cm}


\vspace{0.75cm}
\end{center}

%%%%%%%%%%%%%%%%%%%%%%%%%%%%%%%%%%%%%%%%%%%%%%%%%%%%%%%%%%%%%%%%%%%%%%%%%%



\vspace{1cm}

\noindent Problem 4.1.8  \vspace{0.3cm}


\underline{proposition:} $A,B\subseteq X$, $\chi_A=\chi_B$ iff
$A=B$

\vspace{.2cm}

\fbox{proof} () \vspace{.15cm}



\vspace{1cm}


%%%%%%%%%%%%%%%%%%%%%%%%%%%%%%%%%%%%%%%%%%%%%%%%%%%%%%%%%%%%%%%%%%%%%%%%%%

\noindent Problem 4.2.5  \vspace{0.3cm}

\begin{enumerate}
\item[2]
\underline{proposition:} Given sets $A\subseteq X$ and $B\subseteq
Y$, $A\times B=\pi_1^*(A)\cap \pi_2^*(B)$

\vspace{.2cm}

\fbox{proof} ( ) \vspace{.15cm}




\item[3]
 Given sets $X$ and $Y$.  Let $P\subseteq X\times Y$.  Does $P=\pi_{1*}(P)\times\pi_{2*}(P)?$  Give a proof or counterexample.

\vspace{.2cm}
\vspace{.15cm}


\end{enumerate}
\vspace{1cm}

%%%%%%%%%%%%%%%%%%%%%%%%%%%%%%%%%%%%%%%%%%%%%%%%%%%%%%%%%%%%%%%%%%%%%%%%%%

\noindent Problem 4.2.6  \vspace{0.3cm}

\begin{enumerate}
\item[viii]
\underline{proposition:}  Given $f:A\rightarrow B$ is a function,
and $\{V_j\}_{j\in J}$ is a family of sets such that for all $j\in
J, V_j\subseteq B$, $f^*\left(\bigcup_{j\in
J}V_j\right)=\bigcup_{j\in J}f^*(V_j)$.

\vspace{.2cm}

\fbox{proof} () \vspace{.15cm}

\end{enumerate}
\vspace{1cm}


%%%%%%%%%%%%%%%%%%%%%%%%%%%%%%%%%%%%%%%%%%%%%%%%%%%%%%%%%%%%%%%%%%%%%%%%%%%%%%

\noindent Problem 4.2.8 \vspace{0.3cm}


\begin{enumerate}
\item[1]   Find an example of a function $f:A\rightarrow B$ together with sets $X\subseteq A$ and $Y\subseteq B$ such that $f_*(X)=Y$ and $X\neq f^*(Y).$

\vspace{.2cm}

 \vspace{.15cm}

\item[2]   Find an example of a function $g:J\rightarrow K$ together with sets $Z\subseteq J$ and $W\subseteq K$ such that $g^*(W)=Z$ and $g_*(Z)\neq W.$

\vspace{.2cm}

 \vspace{.15cm}

\end{enumerate}




\vspace{1cm}



%%%%%%%%%%%%%%%%%%%%%%%%%%%%%%%%%%%%%%%%%%%%%%%%%%%%%%%%%%%%%%%%%%%%%%%%%%
\noindent Problem 4.2.11  \vspace{0.3cm}

\underline{proposition:}  Given $f:A\rightarrow B$ is a function,
and $P, Q\subseteq A$, $f_*(P)-f_*(Q)\subseteq f_*(P-Q)$.

\vspace{.2cm}

\fbox{proof} ( ) \vspace{.15cm}









%%%%%%%%%%%%%%%%%%%%%%%%%%%%%%%%%%%%%%%%%%%%%%%%%%%%%%%%%%%%%%%%%%%%%%%%%%


% \noindent Problem 4.3.10  \vspace{0.3cm}


% \underline{proposition:} Let $f:A\rightarrow B$ be a function.
% \begin{itemize}
% \item[i] If $f$ has two distinct left inverses, it has no right inverse.
% % \item[ii] If $f$ has two distinct right inverses, it has no left
% inverse.
% \end{itemize}

% \vspace{.2cm}

% \fbox{proof} ( ) \vspace{.15cm}



% \vspace{1cm}



%%%%%%%%%%%%%%%%%%%%%%%%%%%%%%%%%%%%%%%%%%%%%%%%%%%%%%%%%%%%%%%%%%%%%%%%%%





\end{document}