\documentclass{article}
\usepackage[dvips]{graphicx}
\usepackage{a4wide}
\usepackage{amsmath}
\usepackage{euscript}
\usepackage{amssymb}
\usepackage{amsthm}
\usepackage{amsopn}

\theoremstyle{definition}
\newtheorem*{definition}{Definition}
\newtheorem{theorem}{Theorem}

\newcommand{\vv}{\ensuremath{\vec{v}}}
\newcommand{\vu}{\ensuremath{\vec{u}}}
\newcommand{\vw}{\ensuremath{\vec{w}}}
\newcommand{\vx}{\ensuremath{\vec{x}}}
\newcommand{\vy}{\ensuremath{\vec{y}}}
\newcommand{\vb}{\ensuremath{\vec{b}}}
\newcommand{\vo}{\ensuremath{\vec{0}}}
\newcommand{\va}{\ensuremath{\vec{a}}}
\newcommand{\ve}{\ensuremath{\vec{e}}}

\newcommand{\R}{\mathbb{R}}
\newcommand{\Z}{\mathbb{Z}}
\newcommand{\C}{\mathbb{C}}
\newcommand{\N}{\mathbb{N}}
\newcommand{\Q}{\mathbb{Q}}

%%%%%%%%%%%%%%%%%%%%%%%%%%%%%%%%%%%%%%%%%%%%%%%


\begin{document}
\begin{center}
\Large{Math 251W: Foundations of Advanced Mathematics}

\normalsize Solutions to some of the suggested problems from chapter 7\hrule
\vspace{0.4cm}


\end{center}


\noindent Problem 7.1.3  \vspace{0.3cm}

\begin{enumerate}
\item Commutative, $3$ is the identity element, $2$ and $3$ have inverses
\item Not commutative ($lj\not=jl$), $k$ is the identity element, $j$ and $k$ have inverses
\item Commutative, No identity element
\item Commutative, $c$ is the identity element, $c, d,$ and $e$ have inverses
\item Not commutative, $i$ is the identity element, all elements have inverses.
\end{enumerate}

\vspace{1cm}

%%%%%%%%%%%%%%%%%%%%%%%%%%%%%%%%%%%%%%%%%%%%%%%%%%%%%%%%%%%%%%%%%%%%%%%%%%%%

\noindent Problem 7.2.1  \vspace{0.3cm}

\begin{enumerate}
\item Not a group (not every element has an inverse)
\item Abelian group
\item Abelian group
\item Not a group, doesn't have inverses or an identity element
\item Not a group.  While $a*0=a$, $0*a=-a\not=a$, thus there is no identity element
\item Not a group, no identity element (remember the identity element much commute with all the other elements, even if the group itself is not commutative)
\item Abelian group, identity element is $-1$, $a$-inverse is $-a-2$.
\item Abelian group, identity element is $0$, $a$-inverse is $\frac{-a}{1+a}$ which is well defined since $a\not=-1$.
\end{enumerate}

\vspace{1cm}

%%%%%%%%%%%%%%%%%%%%%%%%%%%%%%%%%%%%%%%%%%%%%%%%%%%%%%%%%%%%%%%%%%%%%%%%

\noindent Problem 7.2.2  \vspace{0.3cm}

\begin{enumerate}
\item Not a group $z=zx=zz$.  Thus, if $z$ has an inverse, which if must to be a group, $x=z$, but $xy\not=zy$.
\item Abelian group
\item Abelian group
\end{enumerate}

\vspace{1cm}

%%%%%%%%%%%%%%%%%%%%%%%%%%%%%%%%%%%%%%%%%%%%%%%%%%%%%%%%%%%%%%%%%%%%%%%%%%%%%%%%%%%%%%%%%%%%%%

\noindent Problem 7.2.6  \vspace{0.3cm}


\underline{proposition:}  Let $(G,*)$ be a group.  For all $g$ in $G$, $g$ has a unique $*$-inverse.
\vspace{.2cm}

\fbox{proof} (Direct) \vspace{.15cm}

Let $(G,*)$ be an arbitrary group, let $e$ be the identity of the group, and let $g$ be an arbitrary element of $G$.  By definition of a group $g$ has at least one inverse element $g^{-1}$.  All that remains to prove is that this inverse is unique.  Let $h$ be an arbitrary inverse of $g$. By properties of the identity, $h=h*e$.  By properties of inverses and the associative property of group operations we have $h=h*e=h*(g*g^{-1})=(h*g)*g^{-1}=g^{-1}$.  Thus $h=g^{-1}$.  Since $h$ was an arbitrary inverse, it follows that all inverses of $g$ are equal to $g^{-1}$, and thus the inverse is unique.

$\blacksquare$

\vspace{1cm}


%%%%%%%%%%%%%%%%%%%%%%%%%%%%%%%%%%%%%%%%%%%%%%%%%%%%%%%%%%%%%%%%%%%%%%%%%%



%%%%%%%%%%%%%%%%%%%%%%%%%%%%%%%%%%%%%%%%%%%%%%%%%%%%%%%%%%%%%%%%%%%%%%%%%%%%%%

\newpage

\noindent Problem 7.2.7  \vspace{0.3cm}

Let $(G,*)$ be a group, and let $a,b,c$ be arbitrary elements of $G$
\begin{enumerate}
\item[i] \underline{prop:} If $a*c=b*c$ then $a=b$ \newline
\fbox{proof} Let $(G,*)$ form a group, and let $a,b,c$ be arbitrary elements of $G$ such that $a*c=b*c$.  By definition of group, there exists an inverse element $c'\in G$ such that $c*c'=e$ where $e$ is the group identity element.  Thus, by substitution and the associative property of group operation, $$a=a*e=a*(c*c')=(a*c)*c'=(b*c)*c'=b*(c*c')=b*e=b.$$  Therefore, $a*c=b*c$ implies $a=b$. $\blacksquare$
\item[ii] proof follows similarly to $(i)$.
\item[iii] \underline{prop:} $(a')'=a$ \newline
\fbox{proof} Let $(G,*)$ form a group, and let $a$ be an arbitrary element of $G$.  By definition of inverse of $a$, $a'*a=e=a*a'$, thus by definition of inverse, $a$ is an inverse of $a'$.  By theorem 7.2.2 inverse are unique, so $(a')'=a$ is the only inverse of $a'$. $\blacksquare$
\end{enumerate}

\vspace{1cm}

%%%%%%%%%%%%%%%%%%%%%%%%%%%%%%%%%%%%%%%%%%%%%%%%%%%%%%%%%%%%%%%%%%%%%%%%%%%%%%

%%%%%%%%%%%%%%%%%%%%%%%%%%%%%%%%%%%%%%%%%%%%%%%%%%%%%%%%%%%%%%%%%%%%%%%%%%%%%

\noindent Problem 7.3.1  \vspace{0.3cm}

\begin{enumerate}
\item Isomorphism
\item Isomorphism (Automorphism: an isomorphism from a group to itself)
\item Not a homomorphism $m(a+b)=a+b+3\not=a+3+b+3=m(a)+m(b)$
\item Isomorphism
\item Not a homomorphism
\end{enumerate}

\vspace{1cm}

%%%%%%%%%%%%%%%%%%%%%%%%%%%%%%%%%%%%%%%%%%%%%%%%%%%%%%%%%%%%%%%%%%%%%%%%%%%%


\noindent Problem 7.3.6  \vspace{0.3cm}

Given $f:G\rightarrow H$ is a homomorphism on the groups $(G,*)$ and $(H,\cdot)$ with identity elements $e_G$ and $e_H$ respectively.
\begin{enumerate}
\item[i] \underline{prop:} $f(e_G)=e_H$
\newline
\fbox{proof}  Let $h\in H$ be the image of $e_G$ under $f$, i.e. $h=f(e_G)$.  By definition of group homomorphism, $f(a)=f(a*e_G)=f(a)\cdot f(e_G)=f(a)\cdot h$.  Furthermore, since $f(a)\in H$, there exists an inverse element $f(a)'$ such that $e_H=f(a)'\cdot f(a)=f(a)'\cdot f(a)\cdot h$.  Thus, by substitution, we have $e_H=e_H\cdot h$, which by definition of identity element implies $e_H=h$.  Thus, $f(e_G)=e_H$. $\blacksquare$
\end{enumerate}


\vspace{1cm}



%%%%%%%%%%%%%%%%%%%%%%%%%%%%%%%%%%%%%%%%%%%%%%%%%%%%%%%%%%%%%%%%%%%%%%%%%%


%%%%%%%%%%%%%%%%%%%%%%%%%%%%%%%%%%%%%%%%%%%%%%%%%%%%%%%%%%%%%%%%%%%%%%%%%%



\end{document}